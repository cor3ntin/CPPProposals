% !TeX program = luatex
% !TEX encoding = UTF-8


\RequirePackage{luatex85}%
\documentclass{wg21}

\usepackage{xcolor}
\usepackage{soul}
\usepackage{ulem}
\usepackage{fullpage}
\usepackage{parskip}
\usepackage{listings}
\usepackage{enumitem}
\usepackage{makecell}
\usepackage{longtable}
\usepackage{graphicx}
\usepackage{amssymb}
\RequirePackage{fontspec}
\usepackage{newunicodechar}
\usepackage{url}

% \setmonofont{Inconsolatazi4}
\setmainfont{Noto Serif}
\usepackage{csquotes}

\lstset{language=C++,
    keywordstyle=\ttfamily,
    stringstyle=\ttfamily,
    commentstyle=\ttfamily,
    morecomment=[l][]{\#},
    morekeywords={reflexpr,valueof,exprid,typeof, concept,constexpr,consteval,unqualid}
}


\lstdefinestyle{color}{language=C++,
        keywordstyle=\color{blue}\ttfamily,
        stringstyle=\color{red}\ttfamily,
        commentstyle=\color{OliveGreen}\ttfamily,
        morecomment=[l][\color{magenta}]{\#},
        morekeywords={reflexpr,valueof,exprid,typeof, concept,constexpr,consteval,unqualid, static\_assert}
}




\title{\tcode{std::generator}: Synchronous Coroutine Generator for Ranges}
\docnumber{D2168R0}
\audience{LEWG}
\author{Lewis Baker}{lbaker@fb.com }
\authortwo{Corentin Jabot}{corentin.jabot@gmail.com}

\begin{document}
\maketitle


\section{Abstract}

We propose a standard library type \tcode{std::generator} which implement a coroutine generator compatible with ranges.

\section{Example}
\begin{lstlisting}[style=color]

std::generator<int> fib (int max) {
    co_yield 0;
    auto a = 0, b = 1;
    
    for(auto n : std::views::iota(0, max))  {
        auto next = a + b;
        a = b, b = next;
        co_yield next;
    }
}

int answer_to_the_universe() {
    auto coro = fib(7) ;
    return std::accumulate(coro | std::views::drop(5), 0);
}

\end{lstlisting}

\section{Motivation}

C++ 20 had a very minimalist library support for coroutines.
Synchronous generators is an important use case for coroutines, one that cannot be supported without 
the machinery presented in this paper.

 
\section{Design}

While the proposed \tcode{std::generator} is fairly straight-forward, a few decisions are worth pointing out.

\subsection{\tcode{input_view}}

\tcode{std::generator} is a non-copyable \tcode{view} which models \tcode{input_range} and spawn move-only iterators.
As such calling \tcode{begin} multiple times is UB.

\subsection{Header}

Multiple options are available as to where put the \tcode{generator} class.

\begin{itemize}
\item \tcode{<coroutine>}, but \tcode{<coroutine>} is a low level header, and \tcode{generator} depends on bits of \tcode{<type_traits>} and \tcode{<iterator>}.

\item \tcode{<ranges>}

\item A new \tcode{<generator>}

\end{itemize}

\subsection{Value Type}

This proposal supports specifying both the "yielded" type, which is the iterator ""reference"" type ( not need to be a reference) and its corresponding value type.
This allow ranges to handle proxy types and wrapped \tcode{reference}, like this implementation of zip:

\begin{lstlisting}[style=color]
template<typename Rng1, typename Rng2>
generator<
    std::tuple<range_reference_t<Rng1>, range_reference_t<Rng2>,
    std::tuple<range_value_type_t<Rng1>, range_value_type_t<Rng2>
>
zip(Rng1 r1, Rng2 r2) {
    auto it1 = begin(r1);
    auto it2 = begin(r2);
    auto end1 = end(r1);
    auto end2 = end(r2);
    while (it1 != end1 && it2 != end2) {
        co_yield {*it1++, *it2++};
    }
}
\end{lstlisting}

\section{Wording}


\rSec2[ranges.syn]{Header \tcode{<ranges>} synopsis}

[...]

\begin{codeblock}
    
namespace std {

\end{codeblock}  

\begin{addedblock}
\begin{codeblock}  
template<typename Y, typename V  = std::remove_cvref_t<Y>>
class generator;

template <typename Y, typename V>
inline constexpr bool ranges::enable_view<generator<Y, V>> = true;

\end{codeblock}
\end{addedblock}
\begin{codeblock}
    
}
\end{codeblock}

\begin{addedblock}


\rSec2[ranges.generator]{Generator View}

\rSec3[ranges.generator.overview]{Overview}


\tcode{generator} produces an \tcode{input_view} over a synchronous coroutine function yielding values.

\begin{example}
\begin{codeblock}
generator<int> iota(int start = 0) {
    while(true)
        co_yield start++;
}

void f() {
    for(auto i : iota() | views::take(3))
        cout << i << " " ; // prints 0 1 2
}

\end{codeblock}
\end{example}

\rSec3[ranges.generator.class]{Class template \tcode{generator}}

\begin{codeblock}
    
namespace std {

template <typename Y, typename V = std::remove_cvref_t<Y>>
class generator : public ranges::view_base {
    
    class promise_type;
    class iterator;
    class sentinel {};
    
    std::coroutine_handle<promise> coroutine_ = nullptr; // \expos
    
    
    explicit generator(std::coroutine_handle<promise> coroutine) noexcept
    : coroutine_(coroutine) {}
    
public:
    generator() = default;
    generator(const generator &other) = delete;
    generator(generator && other)
        :coroutine_(exchange(other.coroutine_, nullptr)){}
    
    ~generator() {
       if (coroutine_) {
           coroutine_.destroy();
       }
    }
    
    generator &operator=(generator && other) noexcept {
        swap(other);
        return *this;
    }
    
    iterator begin();
    sentinel end() noexcept
    { return {};  }
    
    void swap(generator & other) noexcept {
        std::swap(coroutine_, other.coroutine_);
    }

};
\end{codeblock}

\begin{itemdecl}
iterator begin();
\end{itemdecl}

\begin{itemdescr}
    \effects
    Equivalent to:
    \begin{codeblock}
        if(coroutine_) coroutine_.resume();
        return iterator{std::exchange(coroutine_, nullptr)};
    \end{codeblock}
\end{itemdescr}


\rSec3[ranges.generator.promise]{Class template \tcode{generator::promise_type}}

\begin{codeblock}

template <typename Y, typename V>
class generator::promise_type {
   
    std::add_pointer_t<Y> value_; // \expos
    friend generator;     
    
public:
    using value_type = V;
    using reference = std::add_lvalue_reference_t<Y>;
    
    std::coroutine_handle<promise_type> get_return_object() noexcept;
    
    std::suspend_always initial_suspend() const {
        return {};
    }
    std::suspend_always final_suspend() const {
        return {};
    }

    std::suspend_always
    yield_value(std::remove_reference_t<reference> && value) noexcept;
    
    std::suspend_always
    yield_value(std::remove_reference_t<reference> &value) noexcept;
    
    reference value() const noexcept {
        return *value_;
    }

    template <typename U>
    std::suspend_never await_transform(U &&value) = delete;
    
    void return_void() noexcept {}
    
    void unhandled_exception() { throw; }
};  
\end{codeblock}

\begin{itemdecl}
std::coroutine_handle<promise_type> get_return_object() noexcept;
\end{itemdecl}

\begin{itemdescr}
    \effects
    Equivalent to:
    \begin{codeblock}
        return generator{
            std::coroutine_handle<promise_type>::from_promise(*this)};
    \end{codeblock}
\end{itemdescr}


\begin{itemdecl}
std::suspend_always
yield_value(std::remove_reference_t<reference>) noexcept;
std::suspend_always
yield_value(std::remove_reference_t<reference> &value) noexcept;
\end{itemdecl}


\begin{itemdescr}
\effects
Equivalent to:
\begin{codeblock}
    m_value = std::addressof(value);
    return {};
\end{codeblock}
\end{itemdescr}

\rSec3[ranges.generator.iterator]{Class template \tcode{generator::iterator}}

\begin{codeblock}

template <typename Y, typename V>
class generator::iterator {
private:
    std::coroutine_handle<promise_type> coroutine_ = nullptr;
    
    
public:    
    using iterator_category = std::input_iterator_tag;
    using difference_type = std::ptrdiff_t;
    using value_type = promise_type::value_type;
    using reference = promise_type::reference;
    
    iterator() noexcept = default;
    iterator(const iterator &) = delete;
    
    ~iterator() {
        if (coroutine_) {
            coroutine_.destroy();
        }
    }
    
    iterator(iterator && other) noexcept
    : coroutine_(exchange(other.coroutine_, nullptr)) {}
       
    iterator &operator=(iterator &&other) noexcept {
        coroutine_ = exchange(other.coroutine_, nullptr);
    }
    
    explicit iterator(std::coroutine_handle<promise_type> coroutine) noexcept
    : coroutine_(coroutine) {}
    
    bool operator==(sentinel) const noexcept {
         return !coroutine_ || coroutine_.done();
    }
    
    iterator &operator++();
    void operator++(int);
    
    reference operator*() const noexcept;
    reference operator->() const noexcept requires std::is_reference_v<reference>;
    
}; 

\end{codeblock}

\begin{itemdecl}
iterator &operator++();
\end{itemdecl}

\begin{itemdescr}
    \precondition \tcode{coroutine_ \&\& !coroutine_.done()} is \tcode{true}.
    
    \effects
    Equivalent to:
    \begin{codeblock}
     m_coroutine.resume();
     return *this;
    \end{codeblock}
\end{itemdescr}


\begin{itemdecl}
void operator++(int);
\end{itemdecl}

\begin{itemdescr}
    \precondition \tcode{coroutine_ \&\& !coroutine_.done()} is \tcode{true}.
    
    \effects
    Equivalent to:
    \begin{codeblock}
         (void)operator++();
    \end{codeblock}
\end{itemdescr}


\begin{itemdecl}
reference operator*() const noexcept;
reference operator->() const noexcept requires std::is_reference_v<reference>;
\end{itemdecl}

\begin{itemdescr}
    \precondition \tcode{coroutine_ \&\& !coroutine_.done()} is \tcode{true}.
    
    \effects
    Equivalent to:
    \begin{codeblock}
        return coroutine_.promise().value();
    \end{codeblock}
\end{itemdescr}



\end{addedblock}
    

\section{Implementation experience}

//TODO Add links cppcoro, ranges v3, rangesnext, etc


\section{Acknowledgments}


\section{References}
\renewcommand{\section}[2]{}%
\bibliographystyle{plain}
\bibliography{../wg21}

\begin{thebibliography}{9}

    \bibitem[N4861]{N4861}
    Richard Smith
    \emph{Working Draft, Standard for Programming Language C++}\newline
    \url{https://wg21.link/N4861}

\end{thebibliography}
\end{document}
