% !TeX program = luatex
% !TEX encoding = UTF-8


\RequirePackage{luatex85}%
\documentclass{wg21}

\usepackage{xcolor}
\usepackage{soul}
\usepackage{ulem}
\usepackage{fullpage}
\usepackage{parskip}
\usepackage{listings}
\usepackage{enumitem}
\usepackage{makecell}
\usepackage{longtable}
\usepackage{graphicx}
\usepackage{amssymb}
\RequirePackage{fontspec}
\usepackage{newunicodechar}
\usepackage{url}

\setmonofont{Inconsolatazi4}
\setmainfont{Noto Serif}
\usepackage{csquotes}


\lstset{language=C++,
    keywordstyle=\ttfamily,
    stringstyle=\ttfamily,
    commentstyle=\ttfamily,
    morecomment=[l][]{\#},
    morekeywords={reflexpr,valueof,exprid,typeof, concept,constexpr,consteval,unqualid}
}

\lstdefinestyle{color}{language=C++,
        keywordstyle=\color{blue}\ttfamily,
        stringstyle=\color{red}\ttfamily,
        commentstyle=\color{OliveGreen}\ttfamily,
        morecomment=[l][\color{magenta}]{\#},
        morekeywords={reflexpr,valueof,exprid,typeof, concept,constexpr,consteval,unqualid, static\_assert}
}




\title{\tcode{Comparing pairs and tuples}}
\docnumber{D2165R0}
\audience{LEWG}
\author{Corentin Jabot}{corentin.jabot@gmail.com}

\begin{document}
\maketitle


\section{Abstract}

We propose to make tuples of 2 elements and pairs comparable

\section{Tony tables}
\begin{center}
\begin{tabular}{l|l}
Before & After\\ \hline

\begin{minipage}[t]{0.5\textwidth}
\begin{lstlisting}[style=color]

constexpr std::pair  p {1, 3.0};
constexpr std::tuple t {1.0, 3};
static_assert(std::tuple(p) == t);
static_assert(std::tuple(p) <=> t == 0);

\end{lstlisting}
\end{minipage}
&
\begin{minipage}[t]{0.5\textwidth}
\begin{lstlisting}[style=color]

constexpr std::pair  p {1, 3.0};
constexpr std::tuple t {1.0, 3};
static_assert(p == t);
static_assert(p <=> t == 0);

\end{lstlisting}
\end{minipage}
\\\\ \hline

\end{tabular}
\end{center}

\section{Motivation}

\tcode{pair}s are platonic tuples of 2 elements. \tcode{pair} and \tcode{tuple} share
most of their interface.

Notably a tuple can be constructed and assigned fom a pair.
However, \tcode{tuple} and \tcode{pair} are not comparable.
This proposal fixes that.

This makes tuple more consistent (assignment and comparison usually form a pair, at least in regular-ish types),
and makes the library ever so slightly less surprising.
 
\section{Design}

Because \tcode{tuple} is already constructible from \tcode{pair}, and to avoid inter-dependencies between \tcode{<utility>} and \tcode{<tuple>}, 
we propose to add the comparison operator  in the \tcode{<tuple>} header.

The design of these new operators for comparing a tuple and a pair is similar 
to the operators for comparing a pair of tuples.  

\section{Proposal}


\section{Wording}


\rSec2[tuple.syn]{Header \tcode{<tuple>} synopsis}

[...]
\begin{codeblock}

// [tuple.rel], relational operators
template<class... TTypes, class... UTypes>
constexpr bool operator==(const tuple<TTypes...>&, const tuple<UTypes...>&);
template<class... TTypes, class... UTypes>

@\added{template<class... TTypes, class... UTypes>}@
@\added{requires (sizeof..(UTyes) == 2)}@
@\added{constexpr bool operator==(const tuple<TTypes...>\& t, const pair<UTypes...>\& u);}@

constexpr common_comparison_category_t<@\placeholder{synth-three-way-result}@<TTypes, UTypes>...>
operator<=>(const tuple<TTypes...>&, const tuple<UTypes...>&);

@\added{template<class... TTypes, class... UTypes>}@
@\added{requires (sizeof..(UTyes) == 2)}@
@\added{constexpr common_comparison_category_t<\placeholder{synth-three-way-result}<TTypes, UTypes>...>}@
@\added{operator<=>(const tuple<TTypes...>\& t, const pair<UTypes...>\& u);}@

// [tuple.traits], allocator-related traits
template<class... Types, class Alloc>
struct uses_allocator<tuple<Types...>, Alloc>;
\end{codeblock}


\rSec2[tuple.rel]{Relational operators}

\begin{itemdecl}
template<class... TTypes, class... UTypes>
constexpr bool operator==(const tuple<TTypes...>& t, const tuple<UTypes...>& u);
\end{itemdecl}

\begin{addedblock}
\begin{itemdecl}
template<class... TTypes, class... UTypes>
requires (sizeof..(UTyes) == 2)
constexpr bool operator==(const tuple<TTypes...>& t, const pair<UTypes...>& u);
\end{itemdecl}
\end{addedblock}

\begin{itemdescr}
    \pnum
    \mandates
    For all \tcode{i},
    where $0 \leq \tcode{i} < \tcode{sizeof...(TTypes)}$,
    \tcode{get<i>(t) == get<i>(u)} is a valid expression
    returning a type that is convertible to \tcode{bool}.
    \tcode{sizeof...(TTypes)} equals
    \tcode{sizeof...(UTypes)}.
    
    \pnum
    \returns
    \tcode{true} if \tcode{get<i>(t) == get<i>(u)} for all
    \tcode{i}, otherwise \tcode{false}.
    For any two zero-length tuples \tcode{e} and \tcode{f}, \tcode{e == f} returns \tcode{true}.
    
    \pnum
    \effects
    The elementary comparisons are performed in order from the
    zeroth index upwards.  No comparisons or element accesses are
    performed after the first equality comparison that evaluates to
    \tcode{false}.
\end{itemdescr}

\indexlibrarymember{operator<=>}{tuple}%
\begin{itemdecl}
template<class... TTypes, class... UTypes>
constexpr common_comparison_category_t<@\placeholder{synth-three-way-result}@<TTypes, UTypes>...>
operator<=>(const tuple<TTypes...>& t, const tuple<UTypes...>& u);
\end{itemdecl}

\begin{itemdescr}
    \pnum
    \effects
    Performs a lexicographical comparison between \tcode{t} and \tcode{u}.
    For any two zero-length tuples \tcode{t} and \tcode{u},
    \tcode{t <=> u} returns \tcode{strong_ordering::equal}.
    Otherwise, equivalent to:
    \begin{codeblock}
        if (auto c = @\placeholder{synth-three-way}@(get<0>(t), get<0>(u)); c != 0) return c;
        return @$\tcode{t}_\mathrm{tail}$@ <=> @$\tcode{u}_\mathrm{tail}$@;
    \end{codeblock}
    where $\tcode{r}_\mathrm{tail}$ for some tuple \tcode{r}
    is a tuple containing all but the first element of \tcode{r}.
\end{itemdescr}

\pnum
\begin{note}
    The above definition does not require \tcode{t$_{\mathrm{tail}}$}
    (or \tcode{u$_{\mathrm{tail}}$}) to be constructed. It may not
    even be possible, as \tcode{t} and \tcode{u} are not required to be copy
    constructible. Also, all comparison functions are short circuited;
    they do not perform element accesses beyond what is required to determine the
    result of the comparison.
\end{note}

\begin{addedblock}
\begin{itemdecl}
template<class... TTypes, class... UTypes>
requires (sizeof...(UTyes) == 2)
constexpr common_comparison_category_t<@\placeholder{synth-three-way-result}@<TTypes, UTypes>...>
operator<=>(const tuple<TTypes...>& t, const pair<UTypes...>& u);
\end{itemdecl}

\begin{itemdescr}
    \pnum
    \effects
    Performs a lexicographical comparison between \tcode{t} and \tcode{u}.
    
    Equivalent to:
    \begin{codeblock}
        if (auto c = @\placeholder{synth-three-way}@(get<0>(t), get<0>(u)); c != 0)
            return c;
        return  @\placeholder{synth-three-way}@(get<1>(t), get<1>(u));
    \end{codeblock}
\end{itemdescr}
\end{addedblock}



\section{Acknowledgments}


\section{References}
\renewcommand{\section}[2]{}%
\bibliographystyle{plain}
\bibliography{../wg21}

\begin{thebibliography}{9}

    \bibitem[N4861]{N4861}
    Richard Smith
    \emph{Working Draft, Standard for Programming Language C++}\newline
    \url{https://wg21.link/N4861}

\end{thebibliography}
\end{document}
