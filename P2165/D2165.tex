% !TeX program = luatex
% !TEX encoding = UTF-8


\RequirePackage{luatex85}%
\documentclass{wg21}

\usepackage{xcolor}
\usepackage{soul}
\usepackage{ulem}
\usepackage{fullpage}
\usepackage{parskip}
\usepackage{listings}
\usepackage{enumitem}
\usepackage{makecell}
\usepackage{longtable}
\usepackage{graphicx}
\usepackage{amssymb}
\RequirePackage{fontspec}
\usepackage{newunicodechar}
\usepackage{url}

% \setmonofont{Inconsolatazi4}
\setmainfont{Noto Serif}
\usepackage{csquotes}

\lstset{language=C++,
    keywordstyle=\ttfamily,
    stringstyle=\ttfamily,
    commentstyle=\ttfamily,
    morecomment=[l][]{\#},
    morekeywords={reflexpr,valueof,exprid,typeof, concept,constexpr,consteval,unqualid}
}


\lstdefinestyle{color}{language=C++,
        keywordstyle=\color{blue}\ttfamily,
        stringstyle=\color{red}\ttfamily,
        commentstyle=\color{OliveGreen}\ttfamily,
        morecomment=[l][\color{magenta}]{\#},
        morekeywords={reflexpr,valueof,exprid,typeof, concept,constexpr,consteval,unqualid, static\_assert}
}




\title{Compatibility between \tcode{tuple} and tuple-like objects}
\docnumber{D2165R0}
\audience{LEWG}
\author{Corentin Jabot}{corentin.jabot@gmail.com}

\begin{document}
\maketitle


\section{Abstract}

We propose to make tuples of 2 elements and pairs comparable.
We extend comparison and assignment between tuple and any object following the tuple protocol.

\section{Tony tables}
\begin{center}
\begin{tabular}{l|l}
Before & After\\ \hline

\begin{minipage}[t]{0.5\textwidth}
\begin{lstlisting}[style=color]

constexpr std::pair  p {1, 3.0};
constexpr std::tuple t {1.0, 3};
static_assert(std::tuple(p) == t);
static_assert(std::tuple(p) <=> t == 0);

\end{lstlisting}
\end{minipage}
&
\begin{minipage}[t]{0.5\textwidth}
\begin{lstlisting}[style=color]

constexpr std::pair  p {1, 3.0};
constexpr std::tuple t {1.0, 3};
static_assert(p == t);
static_assert(p <=> t == 0);

\end{lstlisting}
\end{minipage}
\\\\ \hline

\end{tabular}
\end{center}

\section{Motivation}

\tcode{pair}s are platonic tuples of 2 elements. \tcode{pair} and \tcode{tuple} share
most of their interface.

Notably, a tuple can be constructed and assigned from a pair.
However, \tcode{tuple} and \tcode{pair} are not comparable.
This proposal fixes that.

This makes tuple more consistent (assignment and comparison usually form a pair, at least in regular-ish types),
and makes the library ever so slightly less surprising.

Following that reasoning, we can extend support for these operations to any tuple-like object, aka object following
the tuple protocol.
 
\section{Design}

We introduce an exposition only concept \tcode{tuple-like} which can then be used in the definition of tuples
comparison and assignment operators.
That same concept can be used in [ranges] to simplify the specification.

\tcode{pair} is not modified as to not introduce dependencies between \tcode{<pair>} and \tcode{<tuple>}

\subsection{Questions For LEWG}

Should \tcode{tuple-like} and \tcode{pair-like} be named concepts (as opposition to exposition only) ?


\section{Future work}

Tuple comparison operators are good candidates for hidden friends. 

\section{Wording}


\rSec2[tuple.syn]{Header \tcode{<tuple>} synopsis}

[...]
\begin{codeblock}
    
template<class T, class... Types>
constexpr const T& get(const tuple<Types...>& t) noexcept;
template<class T, class... Types>
constexpr const T&& get(const tuple<Types...>&& t) noexcept;   
\end{codeblock}

\begin{addedblock}
\begin{codeblock}
    
template <typename T, std::size_t N>   // \expos
constexpr bool is_tuple_element = requires (T t) {
    typename tuple_element_t<N-1, remove_const_t<T>>;
    { get<N-1>(t) } -> convertible_to<tuple_element_t<N-1, T>&>;
} && is_tuple_element<T, N-1>;

template <typename T>
constexpr bool is_tuple_element<T, 0> = true;

template <typename T>
concept @\placeholder{tuple-like}@ // \expos
    = !is_reference_v<T> && requires  {
    typename tuple_size<T>::type; 
    same_as<decltype(tuple_size_v<T>), size_t>;
} && is_tuple_element<T, tuple_size_v<T>>;

template <typename T>
concept @\placeholder{pair-like}@ // \expos
    = @\placeholder{tuple-like}@<T> && std::tuple_size_v<T> == 2;
 
\end{codeblock}
\end{addedblock}


\begin{codeblock}
// [tuple.rel], relational operators
template<class... TTypes, class... UTypes>
constexpr bool operator==(const tuple<TTypes...>&, const @\changed{tuple}{\placeholder{tuple-like}}@<UTypes...>&);
template<class... TTypes, class... UTypes>

constexpr common_comparison_category_t<@\placeholder{synth-three-way-result}@<TTypes, UTypes>...>
operator<=>(const tuple<TTypes...>&, const @\changed{tuple}{\placeholder{tuple-like}}@<UTypes...>&);

// [tuple.traits], allocator-related traits
template<class... Types, class Alloc>
struct uses_allocator<tuple<Types...>, Alloc>;

}
\end{codeblock}



\begin{codeblock}
namespace std {
template<class... Types>
class tuple {
    public:
    // \ref{tuple.cnstr}, \tcode{tuple} construction
    constexpr explicit(@\seebelow@) tuple();
    constexpr explicit(@\seebelow@) tuple(const Types&...);
     // only if \tcode{sizeof...(Types) >= 1}
    template<class... UTypes>
    constexpr explicit(@\seebelow@) tuple(UTypes&&...);
    // only if \tcode{sizeof...(Types) >= 1}
    
    tuple(const tuple&) = default;
    tuple(tuple&&) = default;
    
    template<class... UTypes>
    constexpr explicit(@\seebelow@) tuple(const @\changed{tuple}{\placeholder{tuple-like}}@<UTypes...>&);
    template<class... UTypes>
    constexpr explicit(@\seebelow@) tuple(@\changed{tuple}{\placeholder{tuple-like}}@<UTypes...>&&);
\end{codeblock}


\begin{removedblock}
\begin{codeblock}
    template<class U1, class U2>
    constexpr explicit(@\seebelow@) 
    tuple(const pair<U1, U2>&);   // only if \tcode{sizeof...(Types) == 2}
    template<class U1, class U2>
    constexpr explicit(@\seebelow@) 
    tuple(pair<U1, U2>&&);        // only if \tcode{sizeof...(Types) == 2}
\end{codeblock}
\end{removedblock}
\begin{codeblock}
     
    // allocator-extended constructors
    template<class Alloc>
    constexpr explicit(@\seebelow@)
    tuple(allocator_arg_t, const Alloc& a);
    template<class Alloc>
    constexpr explicit(@\seebelow@)
    tuple(allocator_arg_t, const Alloc& a, const Types&...);
    template<class Alloc, class... UTypes>
    constexpr explicit(@\seebelow@)
    tuple(allocator_arg_t, const Alloc& a, UTypes&&...);
    template<class Alloc>
    constexpr tuple(allocator_arg_t, const Alloc& a, const tuple&);
    template<class Alloc>
    constexpr tuple(allocator_arg_t, const Alloc& a, tuple&&);
    template<class Alloc, class... UTypes>
    constexpr explicit(@\seebelow@)
    tuple(allocator_arg_t, const Alloc& a, const @\changed{tuple}{\placeholder{tuple-like}}@<UTypes...>&);
    template<class Alloc, class... UTypes>
    constexpr explicit(@\seebelow@)
    tuple(allocator_arg_t, const Alloc& a, @\changed{tuple}{\placeholder{tuple-like}}@<UTypes...>&&);
\end{codeblock}
\begin{removedblock}
\begin{codeblock}
    template<class Alloc, class U1, class U2>
    constexpr explicit(@\seebelow@)
    tuple(allocator_arg_t, const Alloc& a, const pair<U1, U2>&);
    template<class Alloc, class U1, class U2>
    constexpr explicit(@\seebelow@)
    tuple(allocator_arg_t, const Alloc& a, pair<U1, U2>&&);
\end{codeblock}
\end{removedblock}
\begin{codeblock}    
    // \ref{tuple.assign}, \tcode{tuple} assignment
    constexpr tuple& operator=(const tuple&);
    constexpr tuple& operator=(tuple&&) noexcept(@\seebelow@);
    
    template<class... UTypes>
    constexpr tuple& operator=(const @\changed{tuple}{\placeholder{tuple-like}}@<<UTypes...>&);
    template<class... UTypes>
    constexpr tuple& operator=(@\changed{tuple}{\placeholder{tuple-like}}@<<UTypes...>&&);
\end{codeblock}
\begin{removedblock}
\begin{codeblock}    
    template<class U1, class U2>
    constexpr tuple& operator=(const pair<U1, U2>&);
     // only if \tcode{sizeof...(Types) == 2}
    template<class U1, class U2>
    constexpr tuple& operator=(pair<U1, U2>&&);
    // only if \tcode{sizeof...(Types) == 2}
\end{codeblock}
\end{removedblock}
\begin{codeblock}   
    
    // \ref{tuple.swap}, \tcode{tuple} swap
    constexpr void swap(tuple&) noexcept(@\seebelow@);
};

template<class... UTypes>
tuple(UTypes...) -> tuple<UTypes...>;
template<@\changed{class T1, class T2}{class... UTypes}@>
tuple(@\changed{pair<T1, T2>}{\placeholder{tuple-like}<UTypes..>}@) -> tuple<@\changed{T1, T2}{UTypes...}@>;
template<class Alloc, class... UTypes>
tuple(allocator_arg_t, Alloc, UTypes...) -> tuple<UTypes...>;
@\removed{template<class Alloc, class T1, class T2>}@
@\removed{tuple(allocator_arg_t, Alloc, pair<T1, T2>) -> tuple<T1, T2>;}@
template<class Alloc, class... UTypes>
tuple(allocator_arg_t, Alloc, @\changed{tuple}{\placeholder{tuple-like}}@<UTypes...>) -> tuple<UTypes...>;
}
\end{codeblock}

\rSec3[tuple.cnstr]{Construction}

[...]


\indexlibraryctor{tuple}%
\begin{itemdecl}
template<class... UTypes> 
constexpr explicit(@\seebelow@) tuple(const @\changed{tuple}{\placeholder{tuple-like}}@<UTypes...>& u);
\end{itemdecl}

\begin{itemdescr}
    \pnum
    \constraints
    \begin{itemize}
        \item
        \tcode{sizeof...(Types)} equals \tcode{sizeof...(UTypes)} and
        
        \item
        \tcode{is_constructible_v<$\tcode{T}_i$, const $\tcode{U}_i$\&>} is \tcode{true} for all $i$, and
        
        \item
        either
        \tcode{sizeof...(Types)} is not 1, or
        (when \tcode{Types...} expands to \tcode{T} and \tcode{UTypes...} expands to \tcode{U})
        \tcode{is_convertible_v<const tuple<U>\&, T>}, \tcode{is_constructible_v<T, const tuple<U>\&>}, and \tcode{is_same_v<T, U>} are all \tcode{false}.
    \end{itemize}
    
    \pnum
    \effects
    Initializes each element of \tcode{*this}
    with the corresponding element of \tcode{u}.
    
    \pnum
    \remarks
    The expression inside \tcode{explicit} is equivalent to:
    \begin{codeblock}
        !conjunction_v<is_convertible<const UTypes&, Types>...>
    \end{codeblock}
\end{itemdescr}

\indexlibraryctor{tuple}%
\begin{itemdecl}
template<class... UTypes> 
constexpr explicit(@\seebelow@) tuple(@\changed{tuple}{\placeholder{tuple-like}}@<UTypes...>&& u);
\end{itemdecl}

\begin{itemdescr}
    \pnum
    \constraints
    \begin{itemize}
        \item
        \tcode{sizeof...(Types)} equals \tcode{sizeof...(UTypes)}, and
        
        \item
        \tcode{is_constructible_v<$\tcode{T}_i$, $\tcode{U}_i$>} is \tcode{true} for all $i$, and
        
        \item
        either
        \tcode{sizeof...(Types)} is not 1, or
        (when \tcode{Types...} expands to \tcode{T} and \tcode{UTypes...} expands to \tcode{U})
        \tcode{is_convertible_v<tuple<U>, T>}, \tcode{is_constructible_v<T, tuple<U>>},
        and \tcode{is_same_v<T, U>} are all \tcode{false}.
    \end{itemize}
    
    \pnum
    \effects
    For all $i$,
    initializes the $i^\text{th}$ element of \tcode{*this} with
    \tcode{std::forward<$\tcode{U}_i$>(get<$i$>(u))}.
    
    \pnum
    \remarks
    The expression inside \tcode{explicit} is equivalent to:
    \begin{codeblock}
        !conjunction_v<is_convertible<UTypes, Types>...>
    \end{codeblock}
\end{itemdescr}

\begin{removedblock}
\indexlibraryctor{tuple}%
\indexlibraryglobal{pair}%
\begin{itemdecl}
template<class U1, class U2> constexpr explicit(@\seebelow@) tuple(const pair<U1, U2>& u);
\end{itemdecl}

\begin{itemdescr}
    \pnum
    \constraints
    \begin{itemize}
        \item \tcode{sizeof...(Types)} is 2,
        \item \tcode{is_constructible_v<$\tcode{T}_0$, const U1\&>} is \tcode{true}, and
        \item \tcode{is_constructible_v<$\tcode{T}_1$, const U2\&>} is \tcode{true}.
    \end{itemize}
    
    \pnum
    \effects
    Initializes the first element with \tcode{u.first} and the
    second element with \tcode{u.second}.
    
    \pnum
    The expression inside \tcode{explicit} is equivalent to:
    \begin{codeblock}
        !is_convertible_v<const U1&, @$\tcode{T}_0$@> || !is_convertible_v<const U2&, @$\tcode{T}_1$@>
    \end{codeblock}
\end{itemdescr}

\indexlibraryctor{tuple}%
\indexlibraryglobal{pair}%
\begin{itemdecl}
    template<class U1, class U2> constexpr explicit(@\seebelow@) tuple(pair<U1, U2>&& u);
\end{itemdecl}

\begin{itemdescr}
    \pnum
    \constraints
    \begin{itemize}
        \item \tcode{sizeof...(Types)} is 2,
        \item \tcode{is_constructible_v<$\tcode{T}_0$, U1>} is \tcode{true}, and
        \item \tcode{is_constructible_v<$\tcode{T}_1$, U2>} is \tcode{true}.
    \end{itemize}
    
    \pnum
    \effects
    Initializes the first element with
    \tcode{std::forward<U1>(u.first)} and the
    second element with \tcode{std::forward<U2>(u.second)}.
    
    \pnum
    The expression inside \tcode{explicit} is equivalent to:
    \begin{codeblock}
        !is_convertible_v<U1, @$\tcode{T}_0$@> || !is_convertible_v<U2, @$\tcode{T}_1$@>
    \end{codeblock}
\end{itemdescr}

\end{removedblock}

\indexlibraryctor{tuple}%
\begin{itemdecl}
    template<class Alloc>
    constexpr explicit(@\seebelow@)
    tuple(allocator_arg_t, const Alloc& a);
    template<class Alloc>
    constexpr explicit(@\seebelow@)
    tuple(allocator_arg_t, const Alloc& a, const Types&...);
    template<class Alloc, class... UTypes>
    constexpr explicit(@\seebelow@)
    tuple(allocator_arg_t, const Alloc& a, UTypes&&...);
    template<class Alloc>
    constexpr tuple(allocator_arg_t, const Alloc& a, const tuple&);
    template<class Alloc>
    constexpr tuple(allocator_arg_t, const Alloc& a, tuple&&);
    template<class Alloc, class... UTypes>
    constexpr explicit(@\seebelow@)
    tuple(allocator_arg_t, const Alloc& a, const @\changed{tuple}{\placeholder{tuple-like}}@<UTypes...>&);
    template<class Alloc, class... UTypes>
    constexpr explicit(@\seebelow@)
    tuple(allocator_arg_t, const Alloc& a, @\changed{tuple}{\placeholder{tuple-like}}@<UTypes...>&&);

\end{itemdecl}
\begin{removedblock}
\begin{itemdecl}
    template<class Alloc, class U1, class U2>
    constexpr explicit(@\seebelow@)
    tuple(allocator_arg_t, const Alloc& a, const pair<U1, U2>&);
    template<class Alloc, class U1, class U2>
    constexpr explicit(@\seebelow@)
    tuple(allocator_arg_t, const Alloc& a, pair<U1, U2>&&);
\end{itemdecl}
\end{removedblock}

\begin{itemdescr}
    \pnum
    \expects
    \tcode{Alloc} meets the
    \oldconcept{Allocator} requirements (\tref{cpp17.allocator}).
    
    \pnum
    \effects
    Equivalent to the preceding constructors except that each element is constructed with
    uses-allocator construction\iref{allocator.uses.construction}.
\end{itemdescr}

\rSec3[tuple.assign]{Assignment}

\pnum
For each \tcode{tuple} assignment operator, an exception is thrown only if the
assignment of one of the types in \tcode{Types} throws an exception.
In the function descriptions that follow, let $i$ be in the range \range{0}{sizeof...\brk{}(Types)}
in order, $\tcode{T}_i$ be the $i^\text{th}$ type in \tcode{Types},
and $\tcode{U}_i$ be the $i^\text{th}$ type in a
template parameter pack named \tcode{UTypes}, where indexing is zero-based.

\indexlibrarymember{operator=}{tuple}%
\begin{itemdecl}
    constexpr tuple& operator=(const tuple& u);
\end{itemdecl}

\begin{itemdescr}
    \pnum
    \effects
    Assigns each element of \tcode{u} to the corresponding
    element of \tcode{*this}.
    
    \pnum
    \remarks
    This operator is defined as deleted unless
    \tcode{is_copy_assignable_v<$\tcode{T}_i$>} is \tcode{true} for all $i$.
    
    \pnum
    \returns
    \tcode{*this}.
\end{itemdescr}

\indexlibrarymember{operator=}{tuple}%
\begin{itemdecl}
    constexpr tuple& operator=(tuple&& u) noexcept(@\seebelow@);
\end{itemdecl}

\begin{itemdescr}
    \pnum
    \constraints
    \tcode{is_move_assignable_v<$\tcode{T}_i$>} is \tcode{true} for all $i$.
    
    \pnum
    \effects
    For all $i$, assigns \tcode{std::forward<$\tcode{T}_i$>(get<$i$>(u))} to
    \tcode{get<$i$>(*this)}.
    
    \pnum
    \remarks
    The expression inside \tcode{noexcept} is equivalent to the logical \textsc{and} of the
    following expressions:
    
    \begin{codeblock}
        is_nothrow_move_assignable_v<@$\mathtt{T}_i$@>
    \end{codeblock}
    where $\mathtt{T}_i$ is the $i^\text{th}$ type in \tcode{Types}.
    
    \pnum
    \returns
    \tcode{*this}.
\end{itemdescr}

\indexlibrarymember{operator=}{tuple}%
\begin{itemdecl}
    template<class... UTypes> constexpr tuple& operator=(const @\changed{tuple}{\placeholder{tuple-like}}@<UTypes...>& u);
\end{itemdecl}

\begin{itemdescr}
    \pnum
    \constraints
    \begin{itemize}
        \item \tcode{sizeof...(Types)} equals \tcode{sizeof...(UTypes)} and
        \item \tcode{is_assignable_v<$\tcode{T}_i$\&, const $\tcode{U}_i$\&>} is \tcode{true} for all $i$.
    \end{itemize}
    
    \pnum
    \effects
    Assigns each element of \tcode{u} to the corresponding element
    of \tcode{*this}.
    
    \pnum
    \returns
    \tcode{*this}.
\end{itemdescr}

\indexlibrarymember{operator=}{tuple}%
\begin{itemdecl}
    template<class... UTypes> constexpr tuple& operator=(@\changed{tuple}{\placeholder{tuple-like}}@<UTypes...>&& u);
\end{itemdecl}

\begin{itemdescr}
    \pnum
    \constraints
    \begin{itemize}
        \item \tcode{sizeof...(Types)} equals \tcode{sizeof...(UTypes)} and
        \item \tcode{is_assignable_v<$\tcode{T}_i$\&, $\tcode{U}_i$>} is \tcode{true} for all $i$.
    \end{itemize}
    
    \pnum
    \effects
    For all $i$, assigns \tcode{std::forward<$\tcode{U}_i$>(get<$i$>(u))} to
    \tcode{get<$i$>(*this)}.
    
    \pnum
    \returns
    \tcode{*this}.
\end{itemdescr}

\begin{removedblock}

\indexlibrarymember{operator=}{tuple}%
\indexlibraryglobal{pair}%
\begin{itemdecl}
    template<class U1, class U2> constexpr tuple& operator=(const pair<U1, U2>& u);
\end{itemdecl}

\begin{itemdescr}
    \pnum
    \constraints
    \begin{itemize}
        \item \tcode{sizeof...(Types)} is 2 and
        \item \tcode{is_assignable_v<$\tcode{T}_0$\&, const U1\&>} is \tcode{true}, and
        \item \tcode{is_assignable_v<$\tcode{T}_1$\&, const U2\&>} is \tcode{true}.
    \end{itemize}
    
    \pnum
    \effects
    Assigns \tcode{u.first} to the first element of \tcode{*this}
    and \tcode{u.second} to the second element of \tcode{*this}.
    
    \pnum
    \returns
    \tcode{*this}.
\end{itemdescr}

\indexlibrarymember{operator=}{tuple}%
\indexlibraryglobal{pair}%
\begin{itemdecl}
    template<class U1, class U2> constexpr tuple& operator=(pair<U1, U2>&& u);
\end{itemdecl}

\begin{itemdescr}
    \pnum
    \constraints
    \begin{itemize}
        \item \tcode{sizeof...(Types)} is 2 and
        \item \tcode{is_assignable_v<$\tcode{T}_0$\&, U1>} is \tcode{true}, and
        \item \tcode{is_assignable_v<$\tcode{T}_1$\&, U2>} is \tcode{true}.
    \end{itemize}
    
    \pnum
    \effects
    Assigns \tcode{std::forward<U1>(u.first)} to the first
    element of \tcode{*this} and\\ \tcode{std::forward<U2>(u.second)} to the
    second element of \tcode{*this}.
    
    \pnum
    \returns
    \tcode{*this}.
\end{itemdescr}

\end{removedblock}



\rSec2[tuple.rel]{Relational operators}

\begin{itemdecl}
template<class... TTypes, class... UTypes>
constexpr bool operator==(const tuple<TTypes...>& t, @\changed{tuple}{\placeholder{tuple-like}}@ tuple<UTypes...>& u);
\end{itemdecl}

\begin{itemdescr}
    \pnum
    \mandates
    For all \tcode{i},
    where $0 \leq \tcode{i} < \tcode{sizeof...(TTypes)}$,
    \tcode{get<i>(t) == get<i>(u)} is a valid expression
    returning a type that is convertible to \tcode{bool}.
    \tcode{sizeof...(TTypes)} equals
    \tcode{sizeof...(UTypes)}.
    
    \pnum
    \returns
    \tcode{true} if \tcode{get<i>(t) == get<i>(u)} for all
    \tcode{i}, otherwise \tcode{false}.
    For any two zero-length tuples \tcode{e} and \tcode{f}, \tcode{e == f} returns \tcode{true}.
    
    \pnum
    \effects
    The elementary comparisons are performed in order from the
    zeroth index upwards.  No comparisons or element accesses are
    performed after the first equality comparison that evaluates to
    \tcode{false}.
\end{itemdescr}

\indexlibrarymember{operator<=>}{tuple}%
\begin{itemdecl}
template<class... TTypes, class... UTypes>
constexpr common_comparison_category_t<@\placeholder{synth-three-way-result}@<TTypes, UTypes>...>
operator<=>(const tuple<TTypes...>& t, const @\changed{tuple}{\placeholder{tuple-like}}@<UTypes...>& u);
\end{itemdecl}

\begin{itemdescr}
    \pnum
    \effects
    Performs a lexicographical comparison between \tcode{t} and \tcode{u}.
    For any two zero-length tuples \tcode{t} and \tcode{u},
    \tcode{t <=> u} returns \tcode{strong_ordering::equal}.
    Otherwise, equivalent to:
    \begin{codeblock}
        if (auto c = @\placeholder{synth-three-way}@(get<0>(t), get<0>(u)); c != 0) return c;
        return @$\tcode{t}_\mathrm{tail}$@ <=> @$\tcode{u}_\mathrm{tail}$@;
    \end{codeblock}
    where $\tcode{r}_\mathrm{tail}$ for some tuple \tcode{r}
    is a tuple containing all but the first element of \tcode{r}.
\end{itemdescr}

\pnum
\begin{note}
    The above definition does not require \tcode{t$_{\mathrm{tail}}$}
    (or \tcode{u$_{\mathrm{tail}}$}) to be constructed. It may not
    even be possible, as \tcode{t} and \tcode{u} are not required to be copy
    constructible. Also, all comparison functions are short circuited;
    they do not perform element accesses beyond what is required to determine the
    result of the comparison.
\end{note}

\rSec1[range.utility]{Range utilities}
\rSec2[range.subrange]{Sub-ranges}

\pnum
The \tcode{subrange} class template combines together an
iterator and a sentinel into a single object that models the
\libconcept{view} concept. Additionally, it models the
\libconcept{sized_range} concept when the final template parameter is
\tcode{subrange_kind::sized}.

\indexlibraryglobal{subrange}%
\begin{codeblock}
namespace std::ranges {
    template<class From, class To>
    concept @\defexposconcept{convertible-to-non-slicing}@ =                    // \expos
    convertible_to<From, To> &&
    !(is_pointer_v<decay_t<From>> &&
    is_pointer_v<decay_t<To>> &&
    @\exposconcept{not-same-as}@<remove_pointer_t<decay_t<From>>, remove_pointer_t<decay_t<To>>>);
    
\end{codeblock}
\begin{removedblock}
\begin{codeblock}    
    template<class T>
    concept @\defexposconcept{pair-like}@ =                                     // \expos
    !is_reference_v<T> && requires(T t) {
        typename tuple_size<T>::type;                       // ensures \tcode{tuple_size<T>} is complete
        requires @\libconcept{derived_from}@<tuple_size<T>, integral_constant<size_t, 2>>;
        typename tuple_element_t<0, remove_const_t<T>>;
        typename tuple_element_t<1, remove_const_t<T>>;
        { get<0>(t) } -> @\libconcept{convertible_to}@<const tuple_element_t<0, T>&>;
        { get<1>(t) } -> @\libconcept{convertible_to}@<const tuple_element_t<1, T>&>;
    };
\end{codeblock}
\end{removedblock}
\begin{codeblock}      
    template<class T, class U, class V>
    concept @\defexposconcept{pair-like-convertible-from}@ =                    // \expos
    !@\libconcept{range}@<T> && @\exposconcept{pair-like}@<T> &&
    @\libconcept{constructible_from}@<T, U, V> &&
    @\exposconcept{convertible-to-non-slicing}@<U, tuple_element_t<0, T>> &&
    @\libconcept{convertible_to}@<V, tuple_element_t<1, T>>;
\end{codeblock}

\rSec2[range.elements]{Elements view}
\rSec3[range.elements.view]{Class template \tcode{elements_view}}

\indexlibraryglobal{elements_view}%
\indexlibrarymember{base}{elements_view}%
\indexlibrarymember{begin}{elements_view}%
\indexlibrarymember{end}{elements_view}%
\indexlibrarymember{size}{elements_view}%
\begin{codeblock}

namespace std::ranges {
    template<class T, size_t N>
    concept @\defexposconcept{has-tuple-element}@ =                   // \expos
    @\added{\exposconcept{tuple-like}<T> \&\& tuple_size_v<T> < N; }@   
    @\removed{requires(T t) \{ }@ 
    @\removed{    typename tuple_size<T>::type;}@
    @\removed{    requires N < tuple_size_v<T>;}@
    @\removed{     typename tuple_element_t<N, T>;}@
    @\removed{     \{ get<N>(t) \} -> convertible_to<const tuple_element_t<N, T>\&>; }@
@\removed{ \}; }@
\end{codeblock}


\section{Acknowledgments}


\section{References}
\renewcommand{\section}[2]{}%
\bibliographystyle{plain}
\bibliography{../wg21}

\begin{thebibliography}{9}

    \bibitem[N4861]{N4861}
    Richard Smith
    \emph{Working Draft, Standard for Programming Language C++}\newline
    \url{https://wg21.link/N4861}

\end{thebibliography}
\end{document}
