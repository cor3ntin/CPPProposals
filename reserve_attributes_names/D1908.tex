\documentclass{wg21}


\usepackage{xcolor}
\usepackage{soul}
\usepackage{ulem}
\usepackage{fullpage}
\usepackage{parskip}
\usepackage{csquotes}
\usepackage{listings}
\usepackage{enumitem}
\usepackage{makecell}
\usepackage{longtable}


%\DeclareUnicodeCharacter{201F}{‟}


\newcommand{\cc}[1]{\mintinline{c++}{#1}}
%\newminted[cpp]{c++}{}


\title{Reserving Attribute Namespaces for Future Use}
\docnumber{P1908R0}
\audience{EWG}
\author{Corentin Jabot}{corentin.jabot@gmail.com}

\begin{document}
\maketitle

\section{Target}

C++23

\section{Abstract}

The standard does not reserve names for future attributes, such that standardizing attributes might affect users negatively
or prevent the committee to use the most appropriate names in an attempt to avoid naming conflicts.
This paper proposes to reserve attributes with no namespace as well as the \tcode{std} namespace.

\section{Proposed Wording}

\rSec1[dcl.attr]{Attributes}%
\indextext{attribute|(}

\rSec2[dcl.attr.grammar]{Attribute syntax and semantics}

\pnum
For an \grammarterm{attribute-token}
(including an \grammarterm{attribute-scoped-token})
not specified in this document, the
behavior is \impldef{behavior of non-standard attributes}.
Any \grammarterm{attribute-token} that is not recognized by the implementation
is ignored.

\begin{addedblock}
\grammarterm{attribute-token} that are not \grammarterm{attribute-scoped-token} and the \tcode{std} \grammarterm{attribute-namespace}
are reserved.
\end{addedblock}


\begin{note}
	Each implementation should choose a distinctive name for the
	\grammarterm{attribute-namespace} in an \grammarterm{attribute-scoped-token}.
\end{note}


 
\begin{thebibliography}{9}
    
    \bibitem[N4830]{N4830}
    Richard Smith
    \emph{Working Draft, Standard for Programming Language C++}\newline
    \url{https://wg21.link/n4830}
    
\end{thebibliography}

\end{document}