% !TeX program = luatex
% !TEX encoding = UTF-8

\documentclass{wg21}

\title{Range constructor for \tcode{std::string\_view} 2: Constrain Harder}
\docnumber{P1989R2}
\audience{LWG}
\author{Corentin Jabot}{corentin.jabot@gmail.com}


\begin{document}

\maketitle

\section{Abstract}

in Belfast, LWG accepted P1391 partially over concern about the constraints for the range constructor, and as such only the iterator+sentinel
constructor was accepted. Please refer to P1391 for the design of the proposed changed.
(P1391 being now accepted, I needed a new paper number for the range constructor.)


\section{Revisions}

\subsection{R2}
\begin{itemize}
    \item Wording fixes (LWG review).
\end{itemize}


\subsection{R1}
\begin{itemize}
    \item Fix some typos and other issues in section 2.1
    \item Add a link to a libstdc++ implementation of the proposal
    \item Wording tweaks
\end{itemize}


\section{Issues found during wording reviews}

The current idiomatic way to construct a \tcode{string_view} is to define a \tcode{string_view} operator on user-defined classes, 
as does \tcode{std::string}, \tcode{QString} Boost Beast, \tcode{fmt} and other.
With the changes as proposed in P1391R3, the range constructor may be selected over the conversion function.
This is not observable in practice, unless the \tcode{string_view} returned by the conversion function is not the same value as what the range constructor would create.

\begin{colorblock}
struct buffer {
   buffer() {};
   char const* begin() const { return data; }
   char const* end() const { return data + 42; }
   operator basic_string_view<char, s>() const{
      return basic_string_view<char, s>(data, data +2);
   }
private:
   char data[42];
};
\end{colorblock}

To make sure this conversion function keeps getting selected, we had the following constraint

\begin{itemize}

\item \tcode{std::remove_cvref_t<R>} has no \tcode{basic\_string\_view<charT, traits>} conversion operator.

\end{itemize}

With that constraint, any type that has a conversion operator will use that conversion operator.
If a \tcode{const} type has a non-const conversion function the program remains ill-formed.

Conversion between \tcode{string_view} types with different \tcode{charT} or different \tcode{type_traits} are ill-formed.

If a type otherwise satisfying the constraints has a conversion operator to a different \tcode{basic_string_view},
notably \tcode{basic_string_view<charT, some-other-traits-type>}, while not itself defining \tcode{using type_traits =  some-other-traits-type}, a program that was previously ill-formed will call the new range overload. 

\section{Limitations}

\tcode{string_view} can be constructed from a type with a deleted conversion operator to \tcode{string_view} after this proposal.
I think this was briefly discussed (in Kona?) and we decided it wasn't an issue.
We noted durin the LWG review (March 2021) that we could add a \tcode{std::constructible_from_string_view} type trait, defaulted to
true, to support this use case, should we want to.


\subsection{Implementability}


The following overload satisfies the desired set of constraints

\begin{colorblock}


template <typename T, typename Traits>
concept has_compatible_traits = !requires { typename T::traits_type; } 
|| ranges::same_as<typename T::traits_type, Traits>;


template<typename charT, typename traits = std::char_traits<char>>
struct basic_string_view {
	//...
	template <ranges::contiguous_range R>
	requires ranges::sized_range<R>
	  && (!std::is_convertible_v<R, const charT*>) 
	  &&   std::is_same_v<std::remove_cvref_t<ranges::range_reference_t<R>>, charT>
	  &&   has_compatible_traits<R, traits>
	  && (!requires (std::remove_cvref_t<R> & d) {
	      d.operator ::std::basic_string_view<charT, traits>();
	  })
	basic_string_view(R&&);
}

\end{colorblock}

This has been implemented in libstdc++ such that it passes the set of tests \href{https://github.com/gcc-mirror/gcc/compare/master...cor3ntin:corentin/P1989?expand=1}{[Github]}

\section{Proposed wording}

Change in \textbf{[string.view] 20.4.2}:
\begin{quote}
\begin{codeblock}

template<class charT, class traits = char_traits<charT>>
class basic_string_view {
public:
    [...]

    // construction and assignment
    constexpr basic_string_view() noexcept;
    constexpr basic_string_view(const basic_string_view&) noexcept = default;
    constexpr basic_string_view& operator=(const basic_string_view&) noexcept = default;
    constexpr basic_string_view(const charT* str);
    constexpr basic_string_view(const charT* str, size_type len);
    template <class It, class End>
    constexpr basic_string_view(It begin, End end);
    @\added{template <class R>}@
    @\added{constexpr basic_string_view(R\&\& r);}@
    

    [...]
};

@\added{template<class R>}@
@\added{basic_string_view(R\&\&)}@
@\added{    -> basic_string_view<ranges::range_value_t<R>>;}@

template<class It, class End>
basic_string_view(It, End) -> basic_string_view<iter_value_t<It>>;




\end{codeblock}
\end{quote}

Change in \textbf{[string.view.cons] 20.4.2.1}:

Add after 7

\begin{quote}
\begin{addedblock}
\begin{itemdecl}
template <class R>
constexpr basic_string_view(R&& r);

\end{itemdecl}

\begin{itemdescr}
    Let \tcode{d} be an lvalue of type \tcode{remove_cvref_t<R>}.
    
    \constraints
    \begin{itemize}
        \item \tcode{R} models \tcode{ranges::contiguous_range} and \tcode{ranges::sized_range},
        \item \tcode{is_same_v<ranges::range_value_t<R>, charT>} is \tcode{true},
        \item \tcode{is_convertible_v<R, const charT*>} is \tcode{false},
        \item \tcode{d.operator ::std::basic_string_view<charT, traits>()} is not a valid expression, and
        \item if the \emph{qualified-id} \tcode{R::traits_type} is valid and denotes a type,\\ \tcode{is_same_v<remove_reference_t<R>::traits\_type, traits>} is \tcode{true}.
    \end{itemize}


    \effects
    Initializes \tcode{data\_} with \tcode{ranges::data(r)} and \tcode{size\_} with \tcode{ranges::size(r)}.


    \throws
    Any exception thrown by \tcode{ranges::data(r)} and \tcode{ranges::size(r)}.



\end{itemdescr}

\end{addedblock}
\end{quote}


Add to the section [string.view.deduct] the following deduction guides:

\textbf{Note to the editors: Rename the title of the [string.view.deduct] section from "Deduction guide" to "Deduction guides".}


\begin{quote}
\begin{addedblock}

\begin{itemdecl}
template<class R>
basic_string_view(R&&)
-> basic_string_view<ranges::range_value_t<R>>;
\end{itemdecl}
\begin{itemdescr}
\constraints
\tcode{R} satisfies \tcode{ranges::contiguous_range}.
\end{itemdescr}

\end{addedblock}
\end{quote}

\end{document}