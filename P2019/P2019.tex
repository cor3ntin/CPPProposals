% !TeX document-id = {9322a846-f757-4574-9231-a2e85c743b21}
% !TeX program = luatex
% !TEX encoding = UTF-8


\RequirePackage{luatex85}%
\documentclass{wg21}

\usepackage{xcolor}
\usepackage{soul}
\usepackage{ulem}
\usepackage{fullpage}
\usepackage{parskip}
%\usepackage{listings}
\usepackage{enumitem}
\usepackage{makecell}
\usepackage{longtable}
\usepackage{graphicx}
\usepackage{amssymb}
\RequirePackageWithOptions{fontspec}
\usepackage{footnote}

\usepackage{newunicodechar}
\usepackage{url}
\usepackage[pass]{geometry}

% \setmainfont{Noto Sans}



\usepackage{blindtext}
\usepackage[most]{tcolorbox}
\definecolor{block-gray}{gray}{0.95}


\newtcolorbox{quoteblock}[1][]{%
    colback=block-gray,
    grow to right by=-10mm,
    grow to left by=-10mm,
    boxrule=0pt,
    boxsep=0pt,
    breakable,
    enhanced jigsaw,
    borderline west={4pt}{0pt}{gray},
    #1,
}

\newfontfamily{\fallbackfont}{Noto Sans}
\DeclareTextFontCommand{\textfallback}{\fallbackfont}
\newunicodechar{ẞ}{\textfallback{ẞ}}


% \setmonofont{Inconsolatazi4}
\setmainfont{Noto Serif}
\usepackage{csquotes}







\newcommand{\cc}[1]{\mintinline{c++}{#1}}
%\newminted[cpp]{c++}{}


\title{Usability improvements for \tcode{std::thread}}
\docnumber{D2019R0}
\audience{LEWG}
\author{Corentin Jabot}{corentin.jabot@gmail.com}

\begin{document}

\maketitle

\setlength{\arrayrulewidth}{0.2mm}
\setlength{\tabcolsep}{8pt}
\renewcommand{\arraystretch}{1.}

\def\changemargin#1#2{\list{}{\rightmargin#2\leftmargin#1}\item[]}
\let\endchangemargin=\endlist 
\setlength\extrarowheight{5pt}

\lstdefinestyle{MY}{language=C++,
    basicstyle=\small\CodeStyle,
    keywordstyle=\color{blue},
    commentstyle=\color{ForestGreen}\itshape\rmfamily,,
    stringstyle=\color{Sepia},
    xleftmargin=0em,
    showstringspaces=false,
    columns=flexible,
    keepspaces=true,
    keywords=[2]{char32_t, char8_t, char16_t},
    keywords=[3]{codepoint},
    keywords=[4]{noexcept,constexpr, static_assert},
    keywordstyle={\color{blue!80!black}},
    keywordstyle=[2]{\color{red!80!black}},
    keywordstyle=[3]{\color{BurntOrange!50!black}},
    keywordstyle=[4]{\color{green!50!black}},
    texcl=true}

\lstdefinestyle{MYSMALL}{language=C++,
    basicstyle=\footnotesize\CodeStyle,
    keywordstyle=\color{blue},
    commentstyle=\color{ForestGreen}\itshape,
    stringstyle=\color{Sepia},
    showstringspaces=false,
    columns=flexible,
    keepspaces=true,
    keywords=[2]{char32_t, char8_t, char16_t},
    keywords=[3]{codepoint},
    keywords=[4]{noexcept,constexpr, static_assert},
    keywordstyle={\color{blue!80!black}},
    keywordstyle=[2]{\color{red!80!black}},
    keywordstyle=[3]{\color{BurntOrange!50!black}},
    keywordstyle=[4]{\color{green!50!black}},
    texcl=true}


\section{Abstract}

We propose a way to set a thread stack size and name before the start of its execution, both of which are,
as we demonstrate, current practices in many domains.

The absence of these features make \tcode{std::thread} and \tcode{std::jthread} unfit or unsatisfactory for many use cases.

\section{Example}

The following code illustrate the totality of the proposed additions:

\begin{lstlisting}[style=MY]
void f();
int main() {
    std::thread thread(
     std::thread::attributes()
        .name("Worker")
        .stack_size(512*1024),
     f
    );
    thread.join();
    return 0;
}
\end{lstlisting}


This code suggest a thread name as well as a stack size 
the implementation should used when creating a new thread of execution.

Achieving the same result in C++20 requires duplicating the entire \tcode{std::thread}
class, which would be difficult to fit in a Tony table.

\pagebreak

Here is how to set the name and stack size of a thread on most posix implementation

\begin{lstlisting}[style=MYSMALL]
int libcpp_thread_create(libcpp_thread_t *t, void *(*func)(void *),
                        void *arg,
                        size_t stack_size,
                        const libcpp_threadname_char_t* name)
{
    int res = 0;
    if(stack_size != 0) {
        pthread_attr_t attr;
        res = pthread_attr_init(&attr);
        if (res != 0) {
            return res;
        }
         // Ignore errors
        pthread_attr_setstacksize(&attr, stack_size);
        res = pthread_create(t, &attr, func, arg);
        // Ignore errors
        pthread_attr_destroy(&attr);
    }
    else {
        res = pthread_create(t, 0, func, arg);
    }
    if (res == 0) {
        // Ignore errors
        pthread_setname_np(*t, name);
    }
    return res;
}
\end{lstlisting}
\section{Motivation}

\subsection{Threads have a name}

Most operating systems, including real-time operating systems for embedded platforms
provide a way to name thread.

Names of threads are usually stored in the control structure the kernel uses to manage threads or tasks.

The name can be used by:
\begin{itemize}
    \item Debuggers such as GDB, LLDB, WinDBG, and IDEs using these tools
    \item Platforms and third party crash dump and trace reporting tools
    \item System task and process monitors
    \item Other profiling tracing and diagnostic tools
    \item Windows Performance Analyzer and ETW tracing
\end{itemize} 

The \href{https://docs.microsoft.com/en-us/visualstudio/debugger/how-to-set-a-thread-name-in-native-code?view=vs-2019}{Visual Studio documentation for SetThreadDescription} explains:

\begin{quoteblock}
Thread naming is possible in any edition of Visual Studio. Thread naming is useful for identifying threads of interest in the Threads window when debugging a running process. Having recognizably-named threads can also be helpful when performing post-mortem debugging via crash dump inspection and when analyzing performance captures using various tools.
\end{quoteblock}

This non-exhaustive table shows that most platforms do in fact provide a way to set and often query a thread name.

\begin{center}
\begin{scriptsize}
\makesavenoteenv{tabular}
\begin{tabular}{ |c|c|c|c| }
  
    %\hline
    %\multicolumn{5}{|c|}{System APIS for Naming Threads} \\
    \hline
    Platform &  At Creation & After  & Query \\
    \hline
    Linux & \multicolumn{2}{c|}{\tcode{pthread_setname_np}\footnote{GLIBC 2.12+, MUSL}} & \tcode{pthread_getname_np} \\
    QNX & \multicolumn{2}{c|}{\tcode{pthread_setname_np}} & \tcode{pthread_getname_np} \\
    NetBSD & \multicolumn{2}{c|}{\tcode{pthread_setname_np}} & \tcode{pthread_getname_np} \\
    \hline
    Win32 & & \tcode{SetThreadDescription}\footnote{Since Windows 10 1607 - In older versions a name can be set only when a debugger is attached, by throwing an exception from the calling thread. See  \href{https://stackoverflow.com/a/59490438/877556}{Windows Documentation} and  \href{https://randomascii.wordpress.com/2015/10/26/thread-naming-in-windows-time-for-something-better/}{this article by Bruce Dawson}} & \tcode{GetThreadDescription} \\
    \hline
    Darwin & & \tcode{pthread_setname_np}\footnote{Can only be called from the new thread} & \tcode{pthread_getname_np} \\
    \hline
    Fuchsia &  \tcode{zx_thread_create} & & \\
    \hline
    Android & \tcode{JavaVMAttachArgs}\footnote{See \url{https://stackoverflow.com/a/59490438/877556}} & &  \\
    \hline
    FreeBSD & \multicolumn{2}{c|}{\tcode{pthread_setname_np}} &  \\
    OpenBSD & \multicolumn{2}{c|}{\tcode{pthread_setname_np}} &  \\
    \hline
    RTEMS \footnote{since 2017} & \tcode{pthread_setname_np}  & \tcode{pthread_setname_np}  & \tcode{pthread_getname_np}   \\
    \hline
    FreeRTOS & \tcode{xTaskCreate} & & \tcode{pcTaskGetName} \\
    \hline
    VXWORKS &  \tcode{taskSpawn} & & \\
    \hline
    eCos & \tcode{cyg_thread_create} & & \\
    \hline
    Plan 9 &\tcode{threadsetname}\footnote{Can only be called from the new thread}  & \tcode{threadsetname}\footnote{Can only be called from the new thread}&\\
    \hline
    Haiku & \tcode{spawn_thread} & \tcode{rename_thread} &\tcode{get_thread_info} \\
    \hline 
    Keil RTX & \tcode{osThreadNew} & & \tcode{osThreadGetName}  \\
    \hline
    Webassembly & & &  \\
    \hline
\end{tabular}
\end{scriptsize}
\end{center}

Web assembly was the only platform for which we didn't find a way to set a thread name.

A cursory review of programming language reveals that at least the following languages/environments provide a way to set thread names:

\href{https://doc.rust-lang.org/std/thread/struct.Builder.html}{Rust}, \href{https://docs.python.org/3/library/threading.html#thread-objects}{Python}, \href{https://dlang.org/phobos/core_thread_osthread.html#.Thread.name}{D} 
, \href{https://docs.microsoft.com/en-us/dotnet/api/system.threading.thread.name?view=netframework-4.8}{C\#}
, \href{https://docs.oracle.com/javase/7/docs/api/java/lang/Thread.html}{Java}
, \href{https://docs.raku.org/type/Thread#method_name}{Raku}
, Swift
, \href{https://doc.qt.io/qt-5/qthread.html#managing-threads}{Qt}
, \href{https://github.com/facebook/folly/blob/master/folly/system/ThreadName.h}{Folly}

We also found \href{https://stackoverflow.com/questions/10121560/stdthread-naming-your-thread}{multiple} questions \href{https://stackoverflow.com/questions/57477053/how-to-set-custom-name-of-this-thread}{related} to setting \href{https://stackoverflow.com/questions/16486937/name-a-thread-created-by-beginthread-in-c}{name thread} 
on Stackoverfow.

Thread names are also the object of a C proposal \cite{N2419}

All of that illustrates that giving a name to os threads is standard practice.


\subsection{Threads have a stack size}

In the following, non-exhaustive table, we observe that almost all APIs across a wide range
of environments expose a stack size that can either be queried or set.
The necessity for such a parameter results from the unfortunate non-existence of infinite tape.

A stack size refers to the number of bytes an application can use to store variables of static storage duration and other implementation-defined information
necessary to store the sequence of stack entries making the stack.

Because of that, all implementations which let a stack size be set, do so during the creation of the thread
of execution.

We observe fewer variations of APIs across platforms (compared to names) as the parameter is a simple integer
that can be no greater than the total system memory.

\tcode{pthread_attr_setstacksize} is part of the POSIX specification since Issue 5 (1997).
However, platforms vary in the minimum and maximum stack size supported.


\begin{changemargin}{-1.7cm}{0.5cm} 
\begin{center}
\begin{footnotesize}
\begin{tabular}{ |c|c|c| }
    
    %\hline
    %\multicolumn{5}{|c|}{System APIS for Naming Threads} \\
    \hline
    Platform &  At Creation & Query \\
    \hline
    Linux & \tcode{pthread_attr_setstacksize} & \tcode{pthread_attr_getstacksize} \\
    QNX & \tcode{pthread_attr_setstacksize} & \tcode{pthread_attr_getst\textbf{}acksize} \\
    \hline
    Win32 & \tcode{CreateThread} &    \\
    \hline
    Darwin & \tcode{pthread_attr_setstacksize} & \tcode{pthread_attr_getstacksize}  \\
    \hline
    Fuchsia & &  \\
    \hline
    Android & \tcode{pthread_attr_setstacksize} & \tcode{pthread_attr_getstacksize} \\
    \hline
    FreeBSD & \tcode{pthread_attr_setstacksize} & \tcode{pthread_attr_getstacksize}  \\
    OpenBSD & \tcode{pthread_attr_setstacksize} & \tcode{pthread_attr_getstacksize}  \\
    NetBSD & \tcode{pthread_attr_setstacksize} & \tcode{pthread_attr_getstacksize}  \\
    \hline
    RTEMS & \tcode{pthread_attr_setstacksize} & \tcode{pthread_attr_getstacksize}  \\
    \hline
    FreeRTOS & \tcode{xTaskCreate} &  \\
    \hline
    VXWORKS &  \tcode{taskSpawn} &  \\
    \hline
    eCos & \tcode{cyg_thread_create} &  \\
    \hline
    Plan 9 & \tcode{threadcreate}  &  \\
    \hline
    Haiku &  &\tcode{get_thread_info} \\
    \hline 
    Keil RTX & & \tcode{osThreadGetStackSize}  \\
    \hline
    Webassembly & & \tcode{pthread_attr_getstacksize}  \\
    \hline
\end{tabular}
\end{footnotesize}
\end{center}
\end{changemargin}

We observe that \href{https://docs.oracle.com/javase/7/docs/api/java/lang/Thread.html#Thread()}{Java}, \href{https://doc.rust-lang.org/std/thread/struct.Builder.html}{Rust}, 
\href{https://docs.python.org/3/library/threading.html}{Python}, \href{https://docs.microsoft.com/en-us/dotnet/api/system.threading.thread.-ctor?view=netframework-4.8#System_Threading_Thread__ctor_System_Threading_ParameterizedThreadStart_System_Int32_}{C\#}, \href{https://downloads.haskell.org/~ghc/latest/docs/html/users_guide/runtime_control.html}{Haskell} (through a compile time parameter), \href{https://dlang.org/phobos/core_thread_osthread.htm}{D}, \href{https://perldoc.perl.org/threads.html#THREAD-STACK-SIZE}{Perl}, Swift,
\href{https://www.boost.org/doc/libs/1_72_0/doc/html/thread/thread_management.html#thread.thread_management.tutorial.attributes}{Boost},  \href{https://doc.qt.io/qt-5/qthread.html#setStackSize}{Qt}
support constructing threads with a stack size.

There are many reasons why a program may need to set a stack size:
\begin{itemize}
\item Ensuring consistent stack-size across platforms for portability and reliability as some applications are designed to be run with a specific amount of stack size. 

More generally, such inconsistencies are a source of bugs and expensive testing.

\item Ability to use less than the platform default (usually 1MB on windows, 2MB on many Unixes), which, when not used is a waste (on systems without virtual memory), especially if a large number of threads is started.
\item Some applications will set a larger stack trace for the main thread, which is then inherited by spawn threads, which might be undesirable.
\item Some applications, notably big games, and other large applications will require a stack larger than the default.
\end{itemize}

\section{Motivation for standardization}

Libc++ \tcode{std::thread} implementation is (very approximately) 1000 lines of code.
Because stack size needs to be set before thread creation, an application wishing to use a non-default stack size has to duplicate that effort.

We found threads classes supporting names and stack size in many open source projects, including POCO, Chromium, Firefox, LLVM, Bloomberg Basic Development Environment, Folly, Intel TBB, Tensorflow...
In many cases, these classes are very similar to \tcode{std::thread}, except for they support a stack size.

Like thread names, adding this support to \tcode{std::thread} would be standardizing existing practices.

People working on AAA games told us that the lack of stack size support prevented them to use \tcode{std::thread}, which therefore fails to be a vocabulary type.
As such this proposal is more about rounding an existing feature rather than proposing a new one.

\section{FAQ}

In which we try to answer all the question we heard about this proposal

\subsection{What about queries?}

We observe that
\begin{itemize}
\item It is rarely useful to query the stack size (except to assert that it in a range acceptable to the application).\\
Querying the stack size could be done by storing a \tcode{std::size_t} within the \tcode{std::thread} instance, which is rather cheap,
but we still don't think it is worth it.
\item It is rarely useful to query the thread name, nor is there a portable way to do so (some platforms have API to do that). Use cases for querying a thread name includes printing stack traces \cite{P0881R5}\\
\item It is also more difficult to design a query API for the name that would not pull in \tcode{<string>}
\item While less convenient facility, it is at least possible to query available properties after creation from \tcode{native_handle}.
\end{itemize}


\subsection{Threads should have names??? What next, \tcode{mutex} should have name ? \tcode{vector}?}

Naming threads is standard practice across many operating system and environments.
This proposal merely proposes to expose this widely available and use system feature.
We observe that it is common for threads to have names as processes do.

Windows indeed has the concept of named mutexes which are used to share mutexes across processes.
However, \tcode{std::mutex} is not intended to be shared across processes and as such does not need
a name nor should it have one.
A quick review of platforms reveals that it is not standard practice to use a mutex across processes (many UNIX systems rely on lock files).

\tcode{std::vector} and other C++ objects are not visible outside the program, except by debuggers which can identify them
by their identifiers. Giving them a name would make little sense.


\subsection{What about domains? Colors?}

These are not features provides by systems nor it is existing practice or implemented in any of the languages and libraries we surveyed.
While it is true that these features may be offered by some tools they either use a heuristic to set these information or rely on tool-specific methods. 

\subsection{I don't need that and don't want to pay for it}

None of the proposed attributes is stored in the thread object nor any object associated with the thread or its associated \tcode{thread} object.
the proposed \tcode{thread::attribute} object can be destroyed after the thread creation.
The behavior of preexisting constructors remains unchanged.

On many implementations, including Linux, the space for the thread name is allocated regardless of whether it is used or not.

\subsection{It's an ABI break ???}

No.
Because none of the attributes is stored in the thread or it's associated \tcode{std::thread} object, the ABI is not changed.
We proposed adding a single template constructor.


\subsection{We cannot speak about stack size in the standard?}
There exist a POSIX function which makes the wording more palatable.
Setting a stack size insufficient for the correct execution of a well-formed program isn't different
than if the default stack size is insufficient ([intro.compliance])

\subsection{This is not something that the committee have the bandwidth to deal with?}

We spent resources standardizing 2 (!) thread classes, which are not used in many cases.
This proposal will help more people use \tcode{std::thread}.

The author of this proposal is aware of the limited resources of the committee, and that informed the design.
The cost of re-implementing classes similar to \tcode{std::thread} is great for the industry.

\subsection{I cannot implement that on my platform?}
Here is a conforming minimal implementation

\begin{lstlisting}[style=MY]
class thread {
    class attributes  {
    public:  
        attributes& stack_size(std::size_t) noexcept {return *this;}
        attributes& name(std::span<const char>)  {return *this;}
        attributes& name(std::span<const char8_t>)  {return *this;}
    };
template<class F, class... Args>}
explicit thread(const attributes &, F&& f, Args&&... args)
: thread(std::forward<F>(f), std::forward<Args>(args)...) {}
};
\end{lstlisting}

\subsection{This belongs in a library?}

Because the proposed attributes may need to be set during the thread creation, a library would have no choice but
to reimplement all of \tcode{std::thread}.
Besides the cost of doing that implementation, it poses composability challenges (cannot put a \tcode{custom_thread} in a \tcode{std::thread_poool} for example)

\subsection{What about GPUs threads?}

While \tcode{std::thread} has no mechanism to specify an execution context, an implementation that wishes
to use \tcode{std::thread} on a GPU or other hardware could ignore all attributes or the ones not relevant on their platform.

\subsection{What about other properties}

Depending on platforms, threads may have

\begin{itemize}
\item A CPU affinity such that they are only executed on a given CPU or set of CPU
\item A CPU preference such that they preferentially executed on a given CPU
\item A priority compared to the thread in the process
\item A priority compared to threads in the system
\end{itemize}

The meaning of each value and parameter has more variation across implementations,
as it is tied to the scheduler or the system.

It is also less generally useful and mostly used in HPC and embedded platforms, where there is the greatest variety of implementation. 

As such, thread priorities and other properties are not proposed in this paper.
However, the API is designed to allow adding support for more properties in the future.

Note that priorities can often be changed after the thread creation making it easier for third
parties libraries to support thread priorities.


\section{Proposed design}

\subsection{Constraints}

\begin{itemize}
\item Some environments do not support naming threads.
\item Thread names can be either narrow encoded or, in the case of win32, Unicode (UTF-16) encoded
\item There is a platform-specific limit on thread name length (15(+1) on Linux, 32K on windows)
\item All platforms expect names to be null-terminated.
\item Some platforms set the name during the thread creation, while on Darwin (and plan 9) it can only be set in the thread which name is set.
\item The stack size is always set prior to the thread creation.
\item Platforms have minimum and maximum stack size that are not always possible to expose
\item Implementation may allocate more stack size than requested (it is usually aligned on a memory page) 
\item Implementation may ignore stack size requests
\item On some platforms, the thread stack size is not configurable.
\item On some platforms, the thread stack size is not query-able.
\item Defining these features in terms of wording may be challenging.
\item \textbf{Users who do not care about these features should not have to pay for it}
\end{itemize}

\subsection{Design}

We propose adding a \tcode{thread::attribute} on which can be set a stack size and a name.
This class can then be passed to the constructor of \tcode{std::thread} and \tcode{std::jthread}.

The setters \tcode{name} and \tcode{stack_size} can be used to set each attributes, which are ignored
if not set.

\tcode{name} can be either utf-8 or narrow encoding. This is because most posix implementations are UTF-8 (but take a narrow encoding parameter) and  windows is UTF-16 (\tcode{wchar_t*}).
\tcode{name} take a span whose header is cheaper than \tcode{string_view} and it otherwise makes no difference.

The name has to be copied by implementations which do support that attribute, however, that name is short,
and limited to 16 or 32 bytes in many implementations.
Still, this is why \tcode{name} is not \tcode{noexcept}

The setters are preferred to constructors arguments as it makes it easier to add other attributes in the
future, such as thread priority, while maintaining a small surface API.

\section{Implementation}

A \href{https://github.com/cor3ntin/llvm-project/tree/thread_name}{prototype implementation} for libc++ (supporting only POSIX) threads has been created to validate the design.

\section{Alternatives considered}

\subsection{P0320 \cite{P0320R1}}

P0320R1 proposes a mechanism such as the \tcode{thread::attributes} class proposed in this paper, except
all attributes would be implementation-defined (the standard would specify no setter).
This puts the burden on the user to check which attributes are present, presumably using \tcode{\#ifdef}.
We feel very strongly that such an approach fails to improve portability and only improves the status quo marginally.
There is little value in standardizing a class without standardizing its members.

Moreover, it proposes a \tcode{get_attributes()} function which would returned an implementation-defined object
with all the supported attributes of that platform.
The problem is that not all attributes that can be set can be queried (and reciprocally), and that interface would force
and implementation to return all the attributes it supports, which is wasteful (would have to allocate for the name if a user wants to check the stack size).

\subsection{P0484 \cite{P0484R1}}

P0484 proposes several solutions in the same design space:

\begin{itemize}
\item  A constructor taking a native handle as parameter
\begin{codeblock}
std::thread thread::thread(native_handle_type h);
\end{codeblock}

This is probably a good idea, regardless of the attributes presented here, 
to interface with C libraries or third-party code.

This solves the problem of having to rewrite an entire thread class just to set a stack size.
However, it would still be painful to do so portably, as described in P0484.
A standard library that targets a limited number of platforms can set the attributes more easily than a library that may desire to work on an environment where C++ is deployed. 

\item A factory function for creating a thread with attributes
\begin{codeblock}
template <class F, class ... Args>
unicorn<std::thread, ??> make_thread(thread::attributes, F && f, Args && ... args);
\end{codeblock}
\end{itemize}

We think this is trying to solve two problems:

\begin{itemize}
\item  Threads cannot be used without exceptions support
\item  Some users want the stack size to be guaranteed
\end{itemize}

We are sympathetic to the first concern, however, it seems orthogonal to thread attributes.
If a unicorn type (\tcode{expected}?) or a cheaper exception mechanism is ever standardized, 
such a factory function will be welcome, but it doesn't prevent a thread constructor to support attributes.
As for guaranteed stack size:

\begin{itemize}
\item  Some platforms do no support stack size at all - doesn't mean they won't use the desired amount
\item  Some platforms may ignore stack size requests silently
\item  Some platforms may allocate more than request to align with memory pages
\item  Trying to check after the thread has started is not possible (aka it would throw an exception even though the new thread has started) 
\end{itemize}

As such, we allow but do not require and implementation to throw when a stack size request cannot be fulfilled.


\section{Wording}

For illustrative purposes only


\rSec3[thread.thread.class]{Class \tcode{thread}}


\begin{codeblock}
namespace std {
    class thread {
        public:
        // types
        class id;
        @\added{class attributes;}@
        using native_handle_type = @\impdefnc@;
        
        // construct/copy/destroy
        thread() noexcept;
        template<class F, class... Args> explicit thread(F&& f, Args&&... args);
        @\added{template<class F, class... Args>}@ 
        @\added{explicit thread(const attributes \& attrs, F\&\& f, Args\&\&... args);}@
        
        ~thread();
        //...
    };
}
\end{codeblock}

\begin{addedblock}

\rSec3[thread.thread.attr]{Class \tcode{thread::attributes}}


\begin{codeblock}

class thread::attributes  {
public:
    attributes() noexcept = default;
    attributes(const attributes&);
    attributes(attributes&& other) = default;
    
    attributes& stack_size(std::size_t size) noexcept;
    attributes& name(std::span<const char> name);
    attributes& name(std::span<const char8_t> name);
    
    @\emph{implementation-defined}@ __name; // \expos
    std::size_t __size; // \expos
};
\end{codeblock}

\pnum
An object of type \tcode{thread::attributes} provides a way to
set different parameters that may be used by the constructor of \tcode{std::thread} and \tcode{std::jhread}.

The behavior of each attribute is implementation-defined.
Attributes that are not handled by the implementation are ignored.

\begin{center}
\begin{footnotesize}
%\def\arraystretch{1.5}%
\begin{tabular}{ll}
    \hline
    Name & Description\\
    \hline
    \tcode{stack\_size} & \parbox{10cm}{
        \vspace{.5\baselineskip}
        \begin{note}
        The intent is to configure a stack size as if by POSIX \mbox{\tcode{pthread_attr_setstacksize()}}.\\
        \end{note}\\
        If a non-zero value is outside of the range of values supported by the implementation, \tcode{thread} and \tcode{jthread} constructors
        may throw \tcode{system_error}.\\
        \vspace{.5\baselineskip}   
    }\\
    \hline
    \tcode{name} & \parbox{10cm}{\vspace{.5\baselineskip}
        \tcode{name} represents a string convertible to the native character encoding which may be used to non-uniquely identify a thread of execution.\\\\
        When the value of \tcode{name} is outside of the range of values supported by the implementation, an implementation\\
        can ignore the attribute entirely or use any \tcode{subspan} of the value to non-uniquely identify the thread of execution.
        \vspace{.5\baselineskip}
}\\
        
    \hline
\end{tabular}
\end{footnotesize}
\end{center}


\begin{itemdecl}
attributes& stack_size(std::size_t size) noexcept;
\end{itemdecl}
\begin{itemdescr}
\effects Initialize the \tcode{stack_size} attribute with the value of \tcode{size}.

\returns \tcode{*this;}
\end{itemdescr}


\begin{itemdecl}
attributes& name(std::span<const char> name);
\end{itemdecl}
\begin{itemdescr}
\preconditions \tcode{name} represents a valid string of character in the narrow literal encoding

\effects Initialize the \tcode{name} attribute with the value of \tcode{name}.

\throws \tcode{bad\_alloc}

\returns \tcode{*this;}


\end{itemdescr}


\begin{itemdecl}
attributes& name(std::span<const char8_t> name);
\end{itemdecl}
\begin{itemdescr}
\effects Initialize the \tcode{name} attribute with the value of \tcode{name}.

\returns \tcode{*this;}

\throws \tcode{bad\_alloc}
\end{itemdescr}
\end{addedblock}

\rSec3[thread.thread.constr]{Constructors}

\begin{itemdecl}
template<class F, class... Args> 
explicit thread(F&& f, Args&&... args);
\end{itemdecl}
\begin{addedblock}
\begin{itemdecl}
template<class F, class... Args> 
explicit thread(const thread::attributes & attrs, F&& f, Args&&... args);
\end{itemdecl}
\end{addedblock}

\begin{itemdescr}
\pnum
\constraints
\begin{itemize}
\begin{addedblock}
\item \tcode{remove_cvref_t<F>} is not the same type as \tcode{thread::attributes}.
\end{addedblock}
\item \tcode{remove_cvref_t<F>} is not the same type as \tcode{thread}.
\end{itemize} 
\pnum
\mandates
The following are all \tcode{true}:
\begin{itemize}
    \item \tcode{is_constructible_v<decay_t<F>, F>},
    \item \tcode{(is_constructible_v<decay_t<Args>, Args> \&\& ...)},
    \item \tcode{is_move_constructible_v<decay_t<F>>},
    \item \tcode{(is_move_constructible_v<decay_t<Args>> \&\& ...)}, and
    \item \tcode{is_invocable_v<decay_t<F>, decay_t<Args>...>}.
\end{itemize}

\pnum
\expects
\tcode{decay_t<F>} and each type in \tcode{decay_t<Args>} meet the
\oldconcept{MoveConstructible} requirements

\pnum
\effects
The new thread of execution executes
\begin{codeblock}
    invoke(@\placeholdernc{decay-copy}@(std::forward<F>(f)), @\placeholdernc{decay-copy}@(std::forward<Args>(args))...)
\end{codeblock}
with the calls to
\tcode{\placeholder{decay-copy}} being evaluated in the constructing thread.
Any return value from this invocation is ignored.
\begin{note}
    This implies that any exceptions not thrown from the invocation of the copy
    of \tcode{f} will be thrown in the constructing thread, not the new thread.
\end{note}
If the invocation of \tcode{invoke} terminates with an uncaught exception,
\tcode{terminate} is called.

\begin{addedblock}
In the second form, \tcode{attrs} can be used to specify implementation-defined thread attributes (see [thread.thread.attr]).
\end{addedblock}

\pnum
\sync
The completion of the invocation of the constructor
synchronizes with the beginning of the invocation of the copy of \tcode{f}.

\pnum
\ensures
\tcode{get_id() != id()}. \tcode{*this} represents the newly started thread.

\pnum
\throws
\tcode{system_error} if unable to start the new thread.

\pnum
\errors
\begin{itemize}
    \item \tcode{resource_unavailable_try_again} --- the system lacked the necessary
    resources to create another thread, or the system-imposed limit on the number of
    threads in a process would be exceeded.
\end{itemize}
\end{itemdescr}



\section{Acknowledgments}

Thanks to Martin Hořeňovský, Kamil Rytarowski, Clément Grégoire, Bruce Dawson, Patrice Roy, Ronen Friedman, Billy Baker and others for their valuable feedback.


\bibliographystyle{plain}
\bibliography{../wg21}


\renewcommand{\section}[2]{}%
\begin{thebibliography}{9}
    \nocite{N4830}
    
    \bibitem[N2419]{N2419}
    Kamil Rytarowski
    \emph{Add methods for setting and getting the thread name} (WG14 proposal)\newline
    \url{http://www.open-std.org/jtc1/sc22/wg14/www/docs/n2419.htm}
\end{thebibliography}



\end{document}
