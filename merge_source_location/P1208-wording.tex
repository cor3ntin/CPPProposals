\documentclass[a4paper,article]{article}
               % See geometry.pdf to learn the layout options. There are lots.

\usepackage[final]
           {listings}     % code listings
\usepackage{color}        % define colors for strikeouts and underlines
\usepackage{underscore}   % remove special status of '_' in ordinary text
\usepackage{xspace}
\usepackage{xcolor}
\usepackage{chngcntr}

\input{macros.tex}
\title{Adopt source\_location from Library Fundamentals V3 for C++20}
\author{Robert Douglas, Corentin Jabot}
\date{2019-03-11}



\begin{document}
\maketitle
\begin{tabular}[t]{|l|l|}\hline
Document Number: &  P1208R4\\\hline
Audience: & LWG\\\hline
Date: &  2019-03-11 \\\hline
Project: & Programming Language C++\\\hline
\end{tabular}

\section{Class \tcode{source_location} [reflection.src_loc]}
\subsection{Header \tcode{\textless source_location\textgreater}  Synopsis [reflection.src_loc.intro]}
\begin{codeblock}
namespace std {
  struct source_location {
    constexpr source_location() noexcept;

    constexpr uint_least32_t line() const noexcept;
    constexpr uint_least32_t column() const noexcept;
    constexpr const char* file_name() const noexcept;
    constexpr const char* function_name() const noexcept;

    static consteval source_location current() noexcept;
  };
}
\end{codeblock}
[\emph{Note:} The intent of \tcode{source_location} is to have a small size and efficient copying.-- \emph{end note}]

\begin{itemdecl}
constexpr source_location() noexcept;
\end{itemdecl}
\begin{itemdescr}
\pnum
\effects
Constructs an object of class \tcode{source_location}.

\pnum
\note
The values are implementation-defined.
\end{itemdescr}

\begin{itemdecl}
constexpr uint_least32_t line() const noexcept;
\end{itemdecl}
\begin{itemdescr}
\pnum
\returns
The presumed line number (16.8) represented by this object.
\end{itemdescr}

\begin{itemdecl}
constexpr uint_least32_t column() const noexcept;
\end{itemdecl}
\begin{itemdescr}
\pnum
\returns
An implementation-defined value representing some offset from the start of the line represented by this object.
\end{itemdescr}

\begin{itemdecl}
constexpr const char* file_name() const noexcept;
\end{itemdecl}
\begin{itemdescr}
\pnum
\returns
The presumed name of the current source file (14.2) represented by this object as an NTBS.
\end{itemdescr}

\begin{itemdecl}
constexpr const char* function_name() const noexcept;
\end{itemdecl}
\begin{itemdescr}
\pnum
\returns
If this object represents a position in the body of a function, returns an implementation-defined NTBS that should correspond to the function name.  Otherwise, returns an empty string.

\end{itemdescr}

\begin{itemdecl}
static consteval source_location current() noexcept;
\end{itemdecl}
\begin{itemdescr}
\pnum
\returns
When invoked by a function call whose \techterm{postfix-expression} is a (possibly parenthesized) \techterm{id-expression} naming \tcode{current}, returns a \tcode{source_location} with an implementation-defined value. The value should be affected by \tcode{\#line} (14.4) in the same manner as for \tcode{__LINE__} and \tcode{__FILE__}. If invoked in some other way, the value returned is unspecified.
\end{itemdescr}
\begin{itemdescr}
\pnum
\note
When a \techterm{brace-or-equal-initializer} is used to initialize a non-static data member, any calls to \tcode{current} should correspond to the location of the constructor or aggregate initialization that initializes the member.

%Member access expression is also precluded
% source_location l; l.current();
%binding a ref to it in one trans unit and using ref in another does not trigger wording for unspecified.

%If \tcode{current} is used other than as the \techterm{postfix-expression} of a function call (5.2.2) the program is ill-formed.

\pnum
[\emph{Note:} When used as a default argument (9.3.6), the value of the \tcode{source_location} will be the location of the call to \tcode{current} at the call site. -- \emph{end note}]
\end{itemdescr}

[\emph{Example:}
\begin{codeblock}
struct s {
  source_location member = source_location::current();
  int other_member;
  s(source_location loc = source_location::current())
    : member(loc) // values of \tcode{member} will be from call-site
  {}
  s(int blather) : // values of \tcode{member} should be hereabouts
    other_member(blather)
  {}
  s(double) // values of \tcode{member} should be hereabouts
  {}
};
void f(source_location a = source_location::current()) {
  source_location b = source_location::current(); // values in \tcode{b} represent this line
}

void g() {
  f(); // \tcode{f}'s first argument corresponds to this line of code

  source_location c = source_location::current();
  f(c); // \tcode{f}'s first argument gets the same values as \tcode{c}, above
}
\end{codeblock}
-- \emph{end example}]

\section{Feature macro}
We recommend the feature macro \tcode{__cpp_lib_source_location} for this feature.
%\end{section}

\end{document}