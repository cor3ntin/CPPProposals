% !TeX program = luatex
% !TEX encoding = UTF-8


\documentclass{wg21}

\title{Improve the wording for Universal Character Names in identifiers}
\docnumber{P2620R2}
\audience{CWG}
\author{Corentin Jabot}{corentin.jabot@gmail.com}

\begin{document}
\maketitle

\section{Abstract}

Reword how \grammarterm{universal-character-names} in identifiers are specified.


\section{Revisions}

\subsection*{Revision 2}

The previous revision of this paper proposed to allow UCNs outside of string literals to refer to
basic character sets elements.
However, we realized that we would not have a good model for UCNs in identifiers that are keywords.
Neither allowing UCNs in keywords, or disallowing them seem desirable because of implementation burden and undue complexity.

It gets more challenging when considering macro names, contextual keywords and preprocessing directives.
However, there was consensus that the other wording changes operated by this paper, ie moving the description of the UCNs handling in [lex.name] was desirable, and this paper now only contain that wording change.
\textbf{This paper no longer contains any design change} and is retargeted at CWG.


\subsection*{Revision 1}
\begin{itemize}
  \item Fix typos
  \item Improve the wording by removing handling of UCNs from phase [lex.charset].
\end{itemize}

%\section {Motivation}
%
%There are restrictions on the constitution of \grammarterm{universal-character-names} that seem artificial,
%and we should lift them\footnote{This is a small cleanup that isn't worth doing unless we can spend very little time on it,
%classified as low priority bucket 72.}!
%
%\begin{colorblock}
%\N{LATIN CAPITAL LETTER I} = 42; // ERROR: I is in the basic character set
%\N{LATIN CAPITAL LETTER I WITH DOT ABOVE} = 42 ;// Ok
%\end{colorblock}
%
%This is by no mean a major issue in C++, as we don't put restrictions on \grammarterm{universal-character-names} in string literals (unlike C),
%but it is somewhat inconsistent with the lexing model.
%
%Instead of restricting \grammarterm{universal-character-names} values,
%we can instead mandate that they are part of valid identifiers outside of strings.
%
%\section{Comparison With C}
%
%C does not allow \grammarterm{universal-character-names} to designate elements of the basic character set:
%
%\begin{quoteblock}
%2 A universal character name shall not designate a codepoint where the hexadecimal value is:
%- less than 00A0 other than 0024 (\$), 0040 (@), or 0060 (` );
%\end{quoteblock}
%
%This has been a pain point for users who would like to consistently use \tcode{\textbackslash{u}} in string literals as
%part of code generation processes.
%
%\begin{itemize}
%\item \href{https://github.com/llvm/llvm-project/issues/36392}{LLVM issue: Unicode string literals}
%\item \href{https://stackoverflow.com/questions/20158472/why-c99-has-such-an-odd-restriction-for-universal-character-names}{Why C99 has such an odd restriction for universal character names?}
%\item \href{https://stackoverflow.com/questions/62759943/restrictions-to-unicode-escape-sequences-in-c11}{Restrictions to Unicode escape sequences in C11}
%\end{itemize}
%
%I hope that both languages regain consistency by:
%\begin{itemize}
%\item Not restricting UCNs in string literals
%\item Not putting restrictions on UCNs in identifiers beyond what naturally falls out of the grammar of identifiers.
%\end{itemize}

\section{Breaking change}

\section{Wording}

\rSec1[lex.separate]{Separate translation}


\pnum{4.} The source file is decomposed into preprocessing
tokens\iref{lex.pptoken} and sequences of whitespace characters
(including comments). A source file shall not end in a partial
preprocessing token or in a partial comment.
Each comment is replaced by one space character. New-line characters are
retained. Whether each nonempty sequence of whitespace characters other
than new-line is retained or replaced by one space character is
unspecified.
\removed{As characters from the source file are consumed
to form the next preprocessing token
(i.e., not being consumed as part of a comment or other forms of whitespace),
except when matching a
\grammarterm{c-char-sequence},
\grammarterm{s-char-sequence},
\grammarterm{r-char-sequence},
\grammarterm{h-char-sequence}, or
\grammarterm{q-char-sequence},
\grammarterm{universal-character-name}s are recognized and
replaced by the designated element of the translation character set.}

The process of dividing a source file's
characters into preprocessing tokens is context-dependent.
\begin{example}
    See the handling of \tcode{<} within a \tcode{\#include} preprocessing
    directive.
\end{example}

\rSec1[lex.charset]{Character sets}

A \grammarterm{universal-character-name}
designates the character in the translation character set
whose UCS scalar value is the hexadecimal number represented by
the sequence of \grammarterm{hexadecimal-digit}s
in the \grammarterm{universal-character-name}.
The program is ill-formed if that number is not a UCS scalar value.
If a \grammarterm{universal-character-name} outside
the \grammarterm{c-char-sequence}, \grammarterm{s-char-sequence}, or
\grammarterm{r-char-sequence} of
a \grammarterm{character-literal} or \grammarterm{string-literal}
(in either case, including within a \grammarterm{user-defined-literal})
corresponds to a control character or
to a character in the basic character set, the program is ill-formed.
\begin{note}
    A sequence of characters resembling a \grammarterm{universal-character-name} in an
    \grammarterm{r-char-sequence}\iref{lex.string} does not form a
    \grammarterm{universal-character-name}.
\end{note}

\rSec1[lex.name]{Identifiers}

\indextext{identifier|(}%
\begin{bnf}
    \nontermdef{identifier}\br
    identifier-start\br
    identifier identifier-continue
\end{bnf}

\begin{bnf}
    \nontermdef{identifier-start}\br
    nondigit\br
    \textnormal{an element of the translation character set of class XID_Start}\\
    \added{universal-character-name \textnormal{designating an element of the translation character set of class XID_Start}}
\end{bnf}

\begin{bnf}
    \nontermdef{identifier-continue}\br
    digit\br
    nondigit\br
    \textnormal{an element of the translation character set of class XID\_Continue}\\
    \added{universal-character-name \textnormal{designating an element of the translation character set of class XID\_Continue}}
\end{bnf}

\begin{bnf}
    \nontermdef{nondigit} \textnormal{one of}\br
    \terminal{a b c d e f g h i j k l m}\br
    \terminal{n o p q r s t u v w x y z}\br
    \terminal{A B C D E F G H I J K L M}\br
    \terminal{N O P Q R S T U V W X Y Z _}
\end{bnf}

\begin{bnf}
    \nontermdef{digit} \textnormal{one of}\br
    \terminal{0 1 2 3 4 5 6 7 8 9}
\end{bnf}

\pnum
\indextext{name!length of}%
\indextext{name}%
The character classes XID_Start and XID_Continue
are Derived Core Properties as described by UAX \#44.

\begin{addedblock}
Each \grammarterm{universal-character-name} is replaced by the designated element of the translation character set.

%A \grammarterm{universal-character-name} matching \grammarterm{identifier-start} or \grammarterm{identifier-continue} shall designate an element %of the translation character with the XID_Start or XID_Continue property respectively.
\end{addedblock}

The program is ill-formed
if an \grammarterm{identifier} does not conform to
Normalization Form C as specified in ISO/IEC 10646.
\begin{note}
    Identifiers are case-sensitive.
\end{note}
\begin{note}
    In translation phase 4,
    \grammarterm{identifier} also includes
    those \grammarterm{preprocessing-token}s\iref{lex.pptoken}
    differentiated as keywords\iref{lex.key}
    in the later translation phase 7\iref{lex.token}.
\end{note}

\end{document}
