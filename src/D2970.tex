% !TeX program = luatex
% !TEX encoding = UTF-8


\documentclass{wg21}

\title{Partial application of concepts in template arguments}
\docnumber{P2970R0}
\audience{EWG}
\author{Corentin Jabot}{corentin.jabot@gmail.com}
\authortwo{Gašper Ažman}{gasper.azman@gmail.com}
\usepackage[dvipsnames]{xcolor}


\begin{document}
\maketitle

\section{Abstract}

Concept template parameters were introduced in \paper{P2841R7}. We propose to extend this feature with partially applied concepts.

This feature was introduced in \paper{P2841R0}, and is re-presented here, adding wording and notes on implementation experience.

\section{Motivation}


The first argument of a concept is the entity for which we are checking satisfaction.
Concepts can take additional arguments parameterizing the concept.

For example, \tcode{invocable<T, int>} checks whether an object of type \tcode{T} can be called
with an \tcode{int} argument.

In specific contexts, a \tcode{type-constraint} can appear in a template head, or before \tcode{auto} to constrain --- or assert ---
a type. In that case, the compiler injects the type to which the concept is applied, rather than this type being specified by the user.

For example, the following two declarations are equivalent:

\begin{colorblock}
template<invocable<int> T>
void f();

template<typename T>
void f() requires invocable<T, int>;
\end{colorblock}

Such short-hand is helpful because a great many concepts are binary or n-ary. In the \tcode{concepts} header, half of the concepts are not unary.


When passing concepts as template arguments, this becomes quickly apparent.
Let us consider a \tcode{range_of} concept constraining the range's value type with a concept template-parameter:

\begin{colorblock}
template <typename R, template <typename> concept C>
concept range_of = std::ranges::range<R> && C<std::ranges::range_value_t<R>>;
\end{colorblock}

We can then have an algorithm taking a range of \tcode{integral}:

\begin{colorblock}
auto median(range_of<std::integral> auto&&);
\end{colorblock}


We may want to be more specific. Can we accept only a range of \tcode{int}{s}?
Given that our \tcode{range_of} expects a concept (it should really accept a universal template parameter as described in \paper{P2989R2}!),
maybe we could use \tcode{same_as}.
But \tcode{same_as} requires arguments. Can we make something like that work?

\begin{colorblock}
auto f(range_of<same_as<int>> auto&&);
\end{colorblock}


We can rewrite our \tcode{range_of} concept to take an extra argument pack forwarded to the concept parameter:

\begin{colorblock}
template <typename R, template <typename...> concept C,  typename... Args>
concept range_of = std::ranges::range<R> && C<std::ranges::range_value_t<R>, Args...>;
void f(range_of<std::same_as, int> auto &&);
\end{colorblock}


This is not the clearest interface. And crucially, it doesn't compose. For example, a range of \tcode{regular} types also convertible to \tcode{int} is not expressible.

One workaround would be to lift the concept and its argument into a type:

\begin{colorblock}
template <template <typename...> concept C, typename... Args>
struct packed_concept {
    template <typename T>
    static constexpr bool apply  = C<T, Args...>;
};

template <typename R, typename... PackedConcept>
concept range_of = std::ranges::range<R>
                   && (PackedConcept::template apply<std::ranges::range_value_t<R>> && ...);
void f(range_of<packed_concept<std::convertible_to, int>,
                packed_concept<std::regular>> auto &&);
\end{colorblock}

This works.
It is also not particularly usable or readable, and all hope of subsumption is lost.
Concepts cannot be class members for a very good reason: It would fundamentally break subsumption. So \tcode{apply} is a \tcode{bool} variable, an atomic constraint.

Can we still support making something similar in the language? Could we make something like this work?

\begin{colorblock}
void f(range_of<@\textbf{concept convertible_to<int>}@> auto)
\end{colorblock}

This is what we are proposing.

% [\href{https://compiler-explorer.com/z/36sqoGsr7}{Run this example on Compiler Explorer}]

We can indeed create a new kind of template argument that carries additional arguments that are injected when the corresponding concept template-parameter is specialized.
\begin{colorblock}
// Given this declaration,
template <typename @\textbf{\textcolor{OliveGreen}{T}}@, template <typename> concept @\textbf{\textcolor{WildStrawberry}{C}}@>
constexpr bool b = @\textbf{\textcolor{WildStrawberry}{C}}@<@\textbf{\textcolor{OliveGreen}{T}}@>;

// this specialization
b<@\textbf{\textcolor{OliveGreen}{double}}@, concept std::convertible_to<@\textbf{\textcolor{Plum}{int}}@>>;

// is rewritten by injecting the arguments passed to the concept argument.
template<>
constexpr bool b<@\textbf{\textcolor{OliveGreen}{double}}@, concept @\textbf{\textcolor{WildStrawberry}{convertible_to}}@<@\textbf{\textcolor{Plum}{int}}@>> = @\textbf{\textcolor{WildStrawberry}{convertible_to}}@<@\textbf{\textcolor{OliveGreen}{double}}@, @\textbf{\textcolor{Plum}{int}}@>;
\end{colorblock}

Notice that in this example, \tcode{C} is a concept template-parameter that accepts exactly one argument,
even though \tcode{convertible_to} needs two.
The other arguments will be filled in automatically.

In effect, it's like we created a new concept:

\begin{colorblock}
template <typename T>
concept convertible_to_int = convertible_to<T, int>;
\end{colorblock}

but we did it inline, directly in the template argument.
This feature (which we call partial application) expresses intent very clearly and works with subsumption.

\subsection{Syntax of partially applied concepts}

Partially applied concepts can appear as template arguments matching a concept template-parameter (and nowhere else).
You might have noticed the \tcode{concept} keyword prefixing template arguments in previous examples and wonder why \tcode{range_of<concept convertible_to<int>>} rather than simply \tcode{range_of<convertible_to<int>>}?

Of course, there is an undeniable parallel with the syntax of concept declaration, which is nice, but the keyword is there to avoid ambiguity.

If the concept accepts a variable number of arguments (because it has defaulted or variadic parameters),
whether \tcode{ConceptName<Args>} is a partial application or a complete specialization
(which is then a boolean expression) can be ambiguous.

This has not been an issue with \grammarterm{type-constraint}{s} because they only ever appear in a context where the first parameter is always injected.

However, the ambiguity between a partial concept application and a boolean expression can appear in a couple of cases.
\begin{enumerate}
    \item In the presence of universal template parameters: Should a given argument matching a UTP be considered a \tcode{bool} or a concept template-parameter?
    \item In an overload set, the concept could be matched to both a \tcode{bool} and a concept, as in this example courtesy of Barry Revzin:
\begin{colorblock}
template <bool B> void f();
template <template <typename> concept C> void f();
f<invocable<int>>(); // boolean or partially applied concept?
\end{colorblock}
\end{enumerate}

We explored various solutions to this ambiguity, including always treating something that \emph{could} be a concept as a concept
and forcing an explicit cast to bool to force a boolean expression or making the concept keyword optional when the concept has a fixed number of arguments.

Both approaches suffer the same issue: They could change the meaning of code when the concept is modified by adding defaulted or variadic
parameters. It could also break pre-C++26 code.
So they do not appear to be robust solutions.

We also refrained from any solution that would make the nature of template arguments somehow deduced from the corresponding parameter during overload resolution, for this would be madness (and it would not really solve the question for universal template parameters).

Ultimately, the \tcode{concept} keyword is a great way to show intent and mirrors concept declaration. The following syntax would be valid:

\begin{colorblock}
some_template_name <
    std::regular,  /// can pass the name directly
    concept regular<>, // no reason this should not work
    invocable<int>, // that is a bool - it will get diagnosed
                    // if no overload or specialization expect a bool
    concept invocable<int> // ok
>;
\end{colorblock}

\subsubsection{Generalized partially applied template template-parameters?}

A question that arises when considering this partial application of concepts is whether it generalizes to other kinds of template-parameters.

Concepts are unique in that the first parameter has a special meaning, and as such, that you could pass the first parameter at a different time than all other parameters does make a lot of sense for other kinds of template names.
After all, this is why \tcode{type-constraint}{s} exist.

To generalize partial application
to other template template-parameters, we would need to be able to provide --- or not --- arbitrary parameters;
We could imagine, for example, being able to write some kind of code where some but not all template arguments are provided.

\begin{colorblock}
Foo<std::map<?, ? , my_comparator>>;
\end{colorblock}

However, this is more complicated than what we need for concepts, and whether it would be useful is debatable.
How arguments and parameters would be matched is less clear.
Besides, unlike concepts, aliases and variable templates can be created at class scope.

Therefore, the motivation for partially applied concepts does not necessarily generalize terribly well.
We did consider whether this sort of placeholder syntax would be a good fit for concepts in isolation, but
we think \tcode{concept \grammarterm{type-constraint}} is more consistent, less novel, and
therefore more teachable than a placeholder syntax.

\subsection{Packs of partial concepts?}

Arguably, one could imagine a scenario in which a pack of partial concepts could be constructed:

\begin{colorblock}
template <typename T, template <typename> concept... Concepts>
concept AllOf  = (Concepts<T> && ...);

template <typename T, template <typename...> concept... Concepts>
concept Foo = AllOf<T, concept Concepts<double>...>;
\end{colorblock}

But it seems difficult to imagine a scenario where this would be useful; i.e., when do different concepts accept the same arguments (besides the first one)?
Supporting that does not seem like a good time investment.


\subsection{Implementation}

A prototype has been implemented in a \href{https://github.com/cor3ntin/llvm-project/commit/7fb869e9b6571629258081674de03279ad60ffd3}{branch of Clang}.
Given that this was done last year, the branch is not actively maintained or available on compiler-explorer, while we focus on both improving the implementation of Clang and upstreaming an implementation of \paper{P2841R7}.
Note that this implementation effort did not cover some aspects, particularly mangling.

\section{Wording}

\rSec1[temp.names]{Names of template specializations}


\begin{bnf}
\nontermdef{template-argument}\br
constant-expression\br
type-id\br
\opt{nested-name-specifier} template-name\br
nested-name-specifier \terminal{template} template-name\br
\begin{addedblock}
partially-applied-concept-argument
\end{addedblock}
\end{bnf}


\rSec1[temp.arg]{Template arguments}

\rSec2[temp.arg.template]{Template template arguments}

\ednote{Add a subsection at the end of [temp.arg.template]} .

\begin{addedblock}

\rSec3[temp.arg.concept.partial]{Partially applied concept arguments}


\begin{bnf}
    \nontermdef{partially-applied-concept-argument}\br
    \terminal{concept} \opt{nested-name-specifier} concept-name \terminal{<} \opt{template-argument-list} \terminal{>}
\end{bnf}


\pnum
The component names of a \grammarterm{partially-applied-concept-argument} are
its \grammarterm{concept-name} and
those of its \grammarterm{nested-name-specifier} (if any).

If \grammarterm{concept-name} denotes a template parameter pack, the program is ill-formed.

A \grammarterm{partially-applied-concept-argument} names an invented concept \tcode{X} defined as

\begin{codeblock}
template <@\tcode{Parameter}@ Injected>
concept X = @\grammarterm{concept-name}@<Injected, @\grammarterm{template-argument-list}@>;
\end{codeblock}

Where \tcode{Parameter} is the first \grammarterm{template-parameter} in the \grammarterm{template-head} \tcode{H} of the concept designated by \grammarterm{concept-name}, without the ellipsis (if any).

If \tcode{H} declares a single non-pack template parameter, or if the \grammarterm{constraint-expression} of \tcode{X} is not valid, the program is ill-formed.

\begin{example}
\begin{colorblock}
template <typename T, template <typename> concept... Concepts>
concept all_of  = (Concepts<T> && ...);

template <typename, auto>
concept A = true;

template <typename T, typename>
concept C = true;

template <typename... T>
concept D = true;

template <typename>
concept E = true;

void f(all_of<concept C<0>, concept C<int>, concept D<int>> auto); // ok
void f(all_of<concept E<int>> auto); // error: E declares a single non-pack template parameter
void f(all_of<concept C<int, int>> auto); // error: the constraint-expression of the invented concept would be C<T, int, int>, which is not a valid expression.

\end{colorblock}
\end{example}

\end{addedblock}

\rSec1[temp.type]{Type equivalence}

\pnum
\indextext{equivalence!template type}%
Two \grammarterm{template-id}{s} are the same if
\begin{itemize}
\item
their \grammarterm{template-name}{s},
\grammarterm{operator-function-id}{s}, or
\grammarterm{literal-operator-id}{s}
refer to the same template, and

\item
their corresponding type \grammarterm{template-argument}{s}
are the same type, and

\item
the template parameter values determined by
their corresponding constant template arguments\iref{temp.arg.nontype}
are template-argument-equivalent (see below), and

\item
their corresponding template \grammarterm{template-argument}{s}
refer to the same template\added{ or are partial-concept-argument-equivalent}.
\end{itemize}

\ednote{Add a subsection at the end of [temp.type]}

\begin{addedblock}
Two template \grammarterm{template-argument}{s} are \defn{partial-concept-argument-equivalent} if:
\begin{itemize}
\item They are both \grammarterm{partially-applied-concept-argument},
\item They have equivalent \grammarterm{template-argument-list}
\item Their \grammarterm{concept-name} refer to the same concept.
\end{itemize}
\end{addedblock}



\subsection{Feature test macro}

\ednote{In [tab:cpp.predefined.ft], bump the value of \tcode{__cpp_template_parameters} to the
    date of adoption} .



\section{Acknowledgments}

\bibliographystyle{plain}
\bibliography{wg21, extra}

\renewcommand{\section}[2]{}%

\begin{thebibliography}{9}


\bibitem[N5008]{N5008}
Thomas Köppe
\emph{Working Draft, Standard for Programming Language C++}\newline
\url{https://wg21.link/N5008}


\end{thebibliography}

\end{document}
