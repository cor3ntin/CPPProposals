% !TeX program = luatex
% !TEX encoding = UTF-8

\documentclass{wg21}


\title{Reflection on Attributes}
\docnumber{P1887R1}
\audience{SG-7}
\author{Corentin Jabot}{corentin.jabot@gmail.com}

\begin{document}
\maketitle



\paperquote{The notion of static types and compile time type checking is central to the effective use of C++\\
    - Bjarne Stroustrup}


\section{Target}

C++23 (Or whenever reflection ships).
At this stage, this paper aims at giving an overview of the feature and does not constitute a complete proposal.

\section{Abstract}

Current reflection proposals do not propose to reflect on attributes. 
However we think it is a powerful feature to have as attributes combined with reflection can enable important use cases, as filtering and augmenting entities with additional information.

Unfortunately, attributes are only identified by their names, and if they accept parameters, the parameters form
a \emph{balanced token sequence}, which makes it impossible for individual parameters to be extracted from within a program.

We introduce the notion of user-defined attributes and a typesafe mechanism to reflect on these attributes and their parameters
at compile time.



\pagebreak

\section{Example}

\begin{colorblock}
namespace Catch {
    struct [[decorator]] test_case {
        constexpr test_case(const char* name)
            : name(name) { }
        const char* name;
    };
}

namespace test {
    void g() {}

    [[+Catch::test_case("Printing something")]]
    void f() {
        g();
        std::cout << "Hello World\n";
    }
}

int main () {
    static constexpr auto r = meta::range(reflexpr(test));
    for...(constexpr auto member :  r) {
        static constexpr bool b = meta::has_attribute<Catch::test_case>(member);
        if constexpr(meta::is_function(member) && b) {
            constexpr auto data = meta::attribute<Catch::test_case>(member);
            std::cout << "Running test " << data.name << "\n";
            auto ptr = &n::unqualid(meta::name_of(member));
            ptr();
        }
    }
}


\end{colorblock}

\section{Proposal}

We propose that any \tcode{regular} type that can be constructed at compile time can appertain to any C++ entity that can be reflected upon through the same syntax as attributes.

We further propose a couple of apis to enable reflecting on these user-defined attributes

\subsection{User-Defined Attribute Declaration}

A user-defined attribute is any class which meets the requirements of \tcode{regular}, and which has \tcode{constexpr}
constructors, copy-constructor and is equality comparable in constant expressions.

The \tcode{[[decorator]]} attribute is proposed to mark types intended to be used as user defined attributes.
This attribute can be helpful for documentation purposes as well as to allow an implementation to offer better diagnostics but is not necessary nor does it affect the behavior of a program.

\subsection{Using User-Defined Attributes}

User-Defined Attributes can be used at the same places other attributes can be used,
and use the same syntax as a qualified-call to a constructor.

\begin{colorblock}
namespace n {
    struct [[decorator]] a {
        constexpr a(int);
    };
    [[+n::a(42)]] void f(); // constructor
}
\end{colorblock}

It is also always possible to invoke the copy constructor


\begin{colorblock}
namespace n {
    struct [[decorator]] a {
        constexpr a(int);
    };
    a get_attribute();
    
    [[+n::a(get_attribute())]] void f(); // copy constructor
}
\end{colorblock}


\subsection{Reflecting on User-Defined Attributes}

The primary use of user defined attributes is to be reflected upon.
(User defined attributes can still get picked up by tooling, including documentation generators, etc).

To that end, we propose a couple of methods to query the presence and the value of a user defined attribute.

\begin{colorblock}
namespace std::meta {
    template <typename Attr>
    consteval bool has_attribute(info x);
    
    template <typename Attr>
    consteval Attr attribute(info x);
}

\end{colorblock}

\begin{note}
These interfaces are inspired by other reflection proposals and may have to change to reflect any design evolution of these proposals.
\end{note}


\section{Design consideration}

\subsection{Multiple definitions of attributes}

If the same user-defined attribute appears multiple times on the same declaration, the program is ill-formed.

Attributes must appertain to the declaration that is being reflected upon.
If different declarations of the same entity do not have the same set of user-defined attributes or have the same attributes with different values, the program is ill-formed, no diagnostic required.

It is ill-formed to redeclare user-defined attributes on definitions if there exists a previous declaration of the same entity
in the same translation units.

\subsection{Syntax}

Because there exist many standard and implementation defined attributes, we need a syntax to disambiguate attributes (which can be unknown from the compiler)
from user-defined attribute.
This paper suggest \tcode{[[ + \emph{constructor-expression}]]} as a strawman syntax.
In Belfast 2019 and in previous versions of this paper, alternatives were explored, including:

\begin{itemize}
\item Having a known-set of namespaces in which attributes would not be considered user defined
\item Delaying the parsing of attributes as user defined until they are reflected upon
\end{itemize}

Ultimately, we came to the disappointing conclusion that attributes and user defined attributes need separate syntaxes. 

\section{Use cases}

\subsection{Filtering}

One major use case for this feature is to tag entity that must be filtered in or out while reflecting upon a member or a class.
Consider the following meta-class:

\begin{colorblock}
class(qobject) my_class {
public:
    [[+qt::slot()]] void show();
    [[+qt::signal()]] void stateChanged();
};

\end{colorblock}

Not all methods are signals, and not all methods are slots, we need a way to filter them.
In the \cite{P0707}, Herb Sutter suggests using the return type of functions to distinguish signal from slots.

\begin{colorblock}
class(qobject) my_class {
    qt::signal mySignal();
    qt::slot mySlot();
};
\end{colorblock}

This would have to look like that because slots can have a non-void return type, which means the qobject meta class implementation would have to extract the different pieces of the function signature to recreate it.

\begin{colorblock}
class(qobject) my_clas {
    qt::signal<void> mySignal();
    qt::slot<void> mySlot();
};
\end{colorblock}

It also makes the intent less clear to the reader as they can't know that the \tcode{qt::signal} type is merely a tag that will not be a part of the generated class.
More critically it makes it almost impossible to combine several tags.

\begin{colorblock}
class(qobject) my_class {
    [[+qt::signal()]]
    [[+qt::notify_property("state")]]; 
    void stateChanged();
};

\end{colorblock}

As showed on pages 2, this could be used by test and bench-marking frameworks, and reduces the scenarios in which the use of macros
is still necessary.


\subsection{Mapping and other information}

A primary use case for reflection is serialization and command line parser generators.
However, it is often beneficial that the name of the serialized field differs from the C++ variable name, as well as to provide extra information to the generator.

\begin{colorblock}
struct my_object {
    [[+json_serializer::field_name("case")]]
    [[+json_serializer::default_value(42)]]
    int caseNumber;
    std::mutex mutex;
};
\end{colorblock}


\begin{colorblock}
struct my_options {
    [[+command_line_parser::short_name("L")]]
    [[+command_line_parser::help("Specify the compression level")]]
    [[+command_line_parser::default_value(5)]]
    int compression_level;
};
\end{colorblock}

\section{Future work}

Along with a way to reflect on user defined attribute, it would be beneficial to copy/inject
standard attributes (including \tcode{maybe_unused}, \tcode{noreturn}, \tcode{nodiscard}, \tcode{deprecated}),
when working with code injection and meta classes.
In the absence of code injection, none of the existing standard attributes seems worth reflecting upon.


\section{Other approaches considered}

One of the early ideas was to have a special syntax to declare attributes, and have the different parameters exposed as a tuple.
This approach added too much complexity and did not provide any benefit compared to the current approach or relying on declared types. 


\section{Implementation}

I have implemented a basic prototype in a fork of clang supporting reflection, proving the viability of the idea,
But the implementation is not yet complete. 


\section{Acknowledgments}

Many thanks to Peter Bindels for
reviewing this work and providing valuable feedback.

\begin{thebibliography}{9}
    
    \bibitem[N4830]{N4830}
    Richard Smith
    \emph{Working Draft, Standard for Programming Language C++}\newline
    \url{https://wg21.link/n4830}
    
    \bibitem[P0707]{P0707}
    Herb Sutter
    \emph{Metaclasses}\newline
    \url{https://wg21.link/p0707}
     
\end{thebibliography}

\end{document}
