% !TeX program = luatex
% !TEX encoding = UTF-8


\documentclass{wg21}

\title{Pack Indexing for Template Names}
\docnumber{P3670R1}
\audience{EWG}
\author{Corentin Jabot}{corentin.jabot@gmail.com}


\usepackage{color, colortbl}
\begin{document}
\maketitle

\section{Revisions}

\subsection{R1}

\begin{itemize}
\item Fix example.
\end{itemize}

\section{Motivation}

We added the ability to index packs of types and expressions in C++26 through \paper{P2662R3}.
(\paper{P2662R3} is now implemented in Clang and GCC, and we got very positive feedback).

However, \paper{P2662R3} does not allow the indexing of a pack of templates.
There is no good reason for that. The intent was always to be able to index all packs.

Both \paper{P2841R7} and \paper{P2989R2} were in flight, and it was not clear to me if either
these papers would impact the indexing of packs of \grammarterm{template-name}{s}.
So, I punt that question to the present paper.
It turns out that \paper{P2841R7} has no impact on the design of this paper - except that indexing a pack of
concept template parameter just works - and \paper{P2989R2} was not approved for C++26.

In short, we are proposing to complete the design of pack indexing.

\section{Design}

The syntax for indexing a pack of template-name is similar to the syntax
to the syntax used to index a pack of types or expressions.

\begin{colorblock}

template < template <typename> typename... TT>
struct S {
    template <typename T>
    using First = TT...[0]<T>;
};
\end{colorblock}

The indexed pack is a \grammarterm{template-name} and can be used anywhere any \grammarterm{template-name}
would be usable.
All packs of template template parameters can be indexed (type, variable, concepts).

\section{Implementation}

This paper has not been implemented, but I am confident this can be implemented in Clang without trouble.
I can't comment on other implementations.

\section{Wording}

\rSec1[temp.names]{Names of template specializations}

\pnum
A template specialization\iref{temp.spec} can be referred to by a
\grammarterm{template-id}:

\begin{bnf}
    \nontermdef{simple-template-id}\br
    template-name \terminal{<} \opt{template-argument-list} \terminal{>}
\end{bnf}

\begin{bnf}
    \nontermdef{template-id}\br
    simple-template-id\br
    operator-function-id \terminal{<} \opt{template-argument-list} \terminal{>}\br
    literal-operator-id \terminal{<} \opt{template-argument-list} \terminal{>}
\end{bnf}

\begin{bnf}
    \nontermdef{template-name}\br
    \removed{identifier} \br
    \added{simple-template-name}\br
    \added{pack-index-template-name}
\end{bnf}

\begin{addedblock}
\begin{bnf}
    \nontermdef{pack-index-template-name}\br
    simple-template-name \terminal{...} \terminal{[} constant-expression \terminal{]}
\end{bnf}

\begin{bnf}
    \nontermdef{simple-template-name}\br
    identifier
\end{bnf}
\end{addedblock}

\begin{bnf}
    \nontermdef{template-argument-list}\br
    template-argument \opt{\terminal{...}}\br
    template-argument-list \terminal{,} template-argument \opt{\terminal{...}}
\end{bnf}

\begin{bnf}
    \nontermdef{template-argument}\br
    constant-expression\br
    type-id\br
    \opt{nested-name-specifier} template-name\br
    nested-name-specifier \terminal{template} template-name
\end{bnf}

\pnum
\indextext{component name}%
The component name of a
\grammarterm{simple-template-id},
\grammarterm{template-id}, or
\grammarterm{template-name}
is the first name in it.

\begin{addedblock}

\pnum
The \grammarterm{simple-template-name} $P$ in a \grammarterm{pack-index-template-name}
shall denote a pack.

\pnum
The \grammarterm{constant-expression} shall be
a converted constant expression \iref{expr.const} of type \tcode{std::size_t}
whose value $V$, termed the index,
is such that $0 \le V < \tcode{sizeof...($P$)}$.

\pnum
A \grammarterm{pack-index-template-name} is a pack expansion \iref{temp.variadic}.

\pnum
\begin{note}
    The \grammarterm{pack-index-template-name} denotes
    the type of the $V^\text{th}$ element of the pack.
\end{note}

\end{addedblock}


\pnum
A \tcode{<} is interpreted as the delimiter of
a \grammarterm{template-argument-list}
if it follows a name that is not a \grammarterm{conversion-function-id} and
\begin{itemize}
    \item
    that follows the keyword \keyword{template} or a \tcode{\~}
    after a \grammarterm{nested-name-specifier} or
    in a class member access expression, or
    \item
    for which name lookup finds the injected-class-name of a class template or
    finds any declaration of a template, or
    \item
    that is an unqualified name
    for which name lookup either finds one or more functions or finds nothing, or
    \item
    that is a terminal name
    in a \grammarterm{using-declarator}\iref{namespace.udecl},
    in a \grammarterm{declarator-id}\iref{dcl.meaning}, or
    in a type-only context
    other than a \grammarterm{nested-name-specifier}\iref{temp.res}.
\end{itemize}


\rSec2[temp.variadic]{Variadic templates}

In a template parameter pack that is a pack expansion \iref{temp.param}:
\begin{itemize}
\item In a \tcode{sizeof...} expression\iref{expr.sizeof}; the pattern is an
\grammarterm{identifier}.

\item In a \grammarterm{pack-index-expression};
the pattern is an \grammarterm{identifier}.

\item In a \grammarterm{pack-index-specifier};
the pattern is a \grammarterm{typedef-name}.

\begin{addedblock}
\item In a \grammarterm{pack-index-template-name};
the pattern is a \grammarterm{simple-template-name}.

\end{addedblock}

\item In a \grammarterm{fold-expression}\iref{expr.prim.fold};
the pattern is the \grammarterm{cast-expression}
that contains an unexpanded pack.

\item In a fold expanded constraint\iref{temp.constr.fold};
the pattern is the constraint of that fold expanded constraint.
\end{itemize}

\ednote{[...]}


\pnum
The instantiation of a pack expansion considers
items $\tcode{E}_1, \tcode{E}_2, \dotsc, \tcode{E}_N$,
where
$N$ is the number of elements in the pack expansion parameters.
Each $\tcode{E}_i$ is generated by instantiating the pattern and
replacing each pack expansion parameter with its $i^\text{th}$ element.
Such an element, in the context of the instantiation, is interpreted as
follows:

\begin{itemize}
\item
if the pack is a template parameter pack, the element is
\begin{itemize}
    \item
    a \grammarterm{typedef-name} for a type template parameter pack,
    \item
    an \grammarterm{id-expression} for a constant template parameter pack, or
    \item
    a \grammarterm{template-name} for a template template parameter pack
\end{itemize}
designating the $i^\text{th}$ corresponding
type, constant, or template template argument;

\item
if the pack is a function parameter pack, the element is an
\grammarterm{id-expression}
designating the $i^\text{th}$ function parameter
that resulted from instantiation of
the function parameter pack declaration;

\item
if the pack is an \grammarterm{init-capture} pack,
the element is an \grammarterm{id-expression}
designating the variable introduced by
the $i^\text{th}$ \grammarterm{init-capture}
that resulted from instantiation of
the \grammarterm{init-capture} pack declaration;
otherwise

\item
if the pack is a structured binding pack,
the element is an \grammarterm{id-expression}
designating the $i^\textrm{th}$ structured binding in the pack
that resulted from the structured binding declaration.
\end{itemize}
When $N$ is zero, the instantiation of a pack expansion
does not alter the syntactic interpretation of the enclosing construct,
even in cases where omitting the pack expansion entirely would
otherwise be ill-formed or would result in an ambiguity in the grammar.

\pnum
The instantiation of a \tcode{sizeof...} expression\iref{expr.sizeof} produces
an integral constant with value $N$.

\pnum
When instantiating a \grammarterm{pack-index-expression} $P$,
let $K$ be the index of $P$.
The instantiation of $P$ is the \grammarterm{id-expression} $\tcode{E}_K$.

\pnum
When instantiating a \grammarterm{pack-index-specifier} $P$,
let $K$ be the index of $P$.
The instantiation of $P$ is the \grammarterm{typedef-name} $\tcode{E}_K$.

\begin{addedblock}
When instantiating a \grammarterm{pack-index-template-name} $P$,
let $K$ be the index of $P$.
The instantiation of $P$ is the \grammarterm{simple-template-name} $\tcode{E}_K$.
\end{addedblock}


\ednote{[...]}

\rSec1[temp.type]{Type equivalence}

\pnum
\indextext{equivalence!template type}%
Two \grammarterm{template-id}{s} are the same if
\begin{itemize}
\item
their \grammarterm{template-name}{s},
\grammarterm{operator-function-id}{s}, or
\grammarterm{literal-operator-id}{s}
refer to the same template, and

\item
their corresponding type \grammarterm{template-argument}{s}
are the same type, and

\item
the template parameter values determined by
their corresponding constant template arguments\iref{temp.arg.nontype}
are template-argument-equivalent (see below), and

\item
their corresponding template \grammarterm{template-argument}{s}
refer to the same template.
\end{itemize}
Two \grammarterm{template-id}{s} that are the same
refer to the same class, function, or variable.

\ednote{[...]}

\pnum
If an expression $e$ is type-dependent \iref{temp.dep.expr},
\tcode{decltype($e$)}
denotes a unique dependent type. Two such \grammarterm{decltype-specifier}{s}
refer to the same type only if their \grammarterm{expression}{s} are
equivalent\iref{temp.over.link}.
\begin{note}
    However, such a type might be aliased,
    e.g., by a \grammarterm{typedef-name}.
\end{note}

\pnum
For a type template parameter pack \tcode{T},
\tcode{T...[}\grammarterm{constant-expression}\tcode{]} denotes
a unique dependent type.

\pnum
If the \grammarterm{constant-expression} of a \grammarterm{pack-index-specifier}
is value-dependent,
two such \grammarterm{pack-index-specifier}s refer to the same type
only if their \grammarterm{constant-expression}s are equivalent \iref{temp.over.link}.
Otherwise, two such \grammarterm{pack-index-specifier}s refer to the same type
only if their indexes have the same value.

\begin{addedblock}
If the \grammarterm{constant-expression} of a \grammarterm{pack-index-template-name}
is value-dependent,
two such \grammarterm{pack-index-template-name}{s} refer to the same template
only if their \grammarterm{constant-expression}{s} are equivalent \iref{temp.over.link}.
Otherwise, two such \grammarterm{pack-index-template-name}{s} refer to the same template
only if their indexes have the same value.
\end{addedblock}



\rSec1[gram.key]{Keywords}

\pnum
\indextext{keyword}%
New context-dependent keywords are introduced into a program by
\tcode{typedef}\iref{dcl.typedef},
\tcode{namespace}\iref{namespace.def},
class\iref{class}, enumeration\iref{dcl.enum}, and
\keyword{template}\iref{temp}
declarations.

\begin{ncbnf}
    typedef-name:\br
    identifier\br
    simple-template-id
\end{ncbnf}

\begin{ncbnf}
    namespace-name:\br
    identifier\br
    namespace-alias
\end{ncbnf}

\begin{ncbnf}
    namespace-alias:\br
    identifier
\end{ncbnf}

\begin{ncbnf}
    class-name:\br
    identifier\br
    simple-template-id
\end{ncbnf}

\begin{ncbnf}
    enum-name:\br
    identifier
\end{ncbnf}

\begin{ncbnf}
    template-name:\br
    \removed{identifier} \br
    \added{simple-template-name}\br
    \added{pack-index-template-name}
\end{ncbnf}


\section{Feature test macros}

\ednote{Bump \tcode{__cpp_pack_indexing} to the date of adoption}.


\renewcommand{\section}[2]{}%
\bibliographystyle{plain}
\bibliography{wg21, extra}

\begin{thebibliography}{9}

\bibitem[N5008]{N5008}
Thomas Köppe
\emph{Working Draft, Standard for Programming Language C++}\newline
\url{https://wg21.link/N5008}


\end{thebibliography}

\end{document}
