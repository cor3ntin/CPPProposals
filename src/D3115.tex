% !TeX program = luatex
% !TEX encoding = UTF-8


\documentclass{wg21}

\title{Data Members, Variables and Alias Declarations Can Introduce A Pack}
\docnumber{D3115R0}
\audience{EWG}
\author{Corentin Jabot}{corentin.jabot@gmail.com}


\usepackage{color, colortbl}
\begin{document}
\maketitle

\section{Abstract}

\section{Motivation}

\begin{itemize}
    \item tuple
    \item variant
    \item SOA Vector
    \item mdspan
    \item integer sequence
    \item MP11
    \item reflection
    \item P2300
    \item ranges
\end{itemize}

\section{The grand organic plan}

Before C++11, if you wanted a tuple, it might have looked like that
\begin{colorblock}
// boost::tuple
template <
class T0 = null_type, class T1 = null_type, class T2 = null_type,
class T3 = null_type, class T4 = null_type, class T5 = null_type,
class T6 = null_type, class T7 = null_type, class T8 = null_type,
class T9 = null_type>
class tuple;
\end{colorblock}

WG21 saw that this could be improved and \paper{N2242} added variadic template parameters and function parameter packs in C++11.
This was a huge success, and it is used in virtually every C++ library from tuples to JSON parsers, CTRE, fmt, Qt, testing frameworks,
HPC applications, std::exec, TensorFlow, embed applications, and everything in between.

In C++17, we realized that if a base specifier could be a pack, then we needed to allow using declarations to be packs too (\paper{P0195R2}).
This enables efficient implementation of the least-terrible way to visit a variant.

In C++20, we fixed that lambdas could not capture a function parameter pack (\paper{P0780R2}).

And more recently we made packs indexable (\paper{P2662R3}), a feature that library vendors promptly requested to be made available to them
in older language modes, as it has the potential to simplify standard library implementations as well as compile times.

We are about to allow befriending packs (\paper{P2893R1}), so we clearly like them.

I think this success can be explained by making metaprogramming less arcane and allowing a surprisingly expressible functional style to build
types and expressions in linear times.

Fold expressions in particular, even if their syntax takes some getting used to are a very powerful way to express some ideas and are less bug-prone than trying to write the same computation manually.

And then there is "Structured Bindings can introduce a Pack" (\paper{P1061R6}), currently making its way through the committee.
This paper is of particular significance because
\begin{itemize}
    \item Packs are no longer tied to templates in some way (lambda \grammarterm{init-capture}{s} already did that arguably, but lambdas are weird).
    \item Consequently, expansions can now appear in non-templated entities, which is novel.
\end{itemize}

Untangling packs from parameters in implementation is going to require some work, if only because some implementations might refer internally to parameter packs :).
However, it does create a model and the groundwork to support packs outside of templates.

In summary, pack declarations can appear in:
\begin{itemize}
  \item Template parameters
  \item Function parameter
  \item Init capture
  \item Structured bindings
\end{itemize}

That list feels rather incomplete, and in this paper, we explore extending it to data members, variables, and alias declarations.

It should be noted that this paper is entirely based on existing work, and the ideas presented here have been discussed since at least 2016.
Notably, in the trailblazing \paper{P0341R0} and excellent \paper{P1858R2}.
The Circle C++ Compiler extensively experimented with most of the same ideas.

But before talking about exciting new features, let's build a common understanding of packs as they exist today.

\section{A primer on pack syntax}

A pack is a name that represents a sequence of entities (variables, types, template template parameters).

Packs can be part of a wider pattern, and once the pack is no longer dependent, it is replaced by some transformation of the pattern, ie an expansion.
Any substitution of the pattern is an expansion (including \tcode{sizeof...} and pack indexing).

So there are really 2 fundamental categories of operations that can be done to a pack:
\begin{itemize}
\item Designating an entity as being a pack.
\item Expanding a pack, potentially as part of a wider pattern.
\end{itemize}

That's really all there is to it.  However
\begin{itemize}
\item Expansion can be arbitrarily complex (for example, they may involve lambdas).
\item A number of proposals (this one, expansion statements (\paper{P1306R1}), the generalized unpacking mechanism proposed in \paper{P1858R2}) introduce
constructs that, if not very carefully designed, could lead to a host of grammar ambiguities.
\end{itemize}

The syntactic model that we have is, for some pack \tcode{T}:

\begin{itemize}
\item \tcode{...T} introduced a pack
\item \tcode{T...} expands (consummes) the pack.
\end{itemize}

We have been pretty good at following that model, even if \tcode{sizeof...(T)} is somewhat of an outlier (but that does not lead to any ambiguity concern, so it is fine),
and even if we had to operate some breakage to arcane/unused/unimplemented syntax to ensure pack indexing is consistent with that model.

Consistency is important for teaching, implementability, and making sure we don't close evolutionary doors or introduce hard-to-fix ambiguities.

Multiple packs can be expanded at once, in lockstep, as long as they have the same size.

Keeping that in mind, what are we actually proposing?
We are going to present Alias Declaration Packs, Variable Packs, and Member Packs in that order because the notions, design, and implementation challenges roughly build up in that order.

\pagebreak
\section{Alias Declaration Packs}
If an alias declaration introduces a type, an alias pack declaration introduces a pack of type, and the defining-type-id is the pattern that will be substituted when the type is used.

\begin{center}
\begin{tabular}{l|l}
\begin{minipage}[t]{0.5\textwidth}
\begin{colorblock}
template <typename T>
struct S{
    using value_type = std::remove_cvref_t<T>;
};
\end{colorblock}
\end{minipage}
&
\begin{minipage}[t]{0.5\textwidth}
\begin{colorblock}
template <typename ...Ts>
struct S{
    using ...value_types = std::remove_cvref_t<Ts>;
};
\end{colorblock}
\end{minipage}
\\
\end{tabular}
\end{center}

This brings the bulk of the design and implementation questions of this proposal, the rest are incremental additions.

\subsection{Packs outside of templates}

Because alias declarations can appear at namespace and class scope, they can be named outside of template entities.

\begin{colorblock}
template <typename ...Ts>
void g();
void f() {
    g<S<int, int>::value_types...>();
}
\end{colorblock}


Packs outside of templates are not a new feature. At least, assuming \paper{P1306R1} gets adopted, the specification
and implementation will be able to support them.

But EDG in particular might rely on the fact that \paper{P1306R1} can only declare packs at function scope, in a statement that appears
before any expression might expand the pack.
Depending on that implementation choice, dealing with a pack that appears out of thin air in an expansion might be quite a bit harder to
support than the limited scenario where a pack is previously introduced.
This problem has been described in \paper{P2277R0}.

We have roughly two options here:

\begin{itemize}
\item Not allowing some packs to be expanded outside of templates (users would have to whip out a function / generic lambdas)
\item Bite the bullet understanding it will take significant implementation efforts.
\end{itemize}

I did briefly consider a third alternative: Adding some syntax to advertise the presence of packs in the function.
However this would be a transient solution pushing the burden of implementation toward users, and neither transient solutions,
not putting undue burden on users seems compelling.

Plus, if we wanted a transient syntax to advertise the possibility of pack expansion in a function... making the function a template ought to be good enough.

\begin{colorblock}
template <std::same_as<void> T = void> // not great, not terrible
auto get_deduplicated_variant()  {
    auto variant = get_variant();
    return std::variant <
        unique_types_ts<decltype(variant)...
    >(std::visit(std::identity{}, variant));
}
\end{colorblock}


But ultimately, I think we should bite the bullet.
Knowing this is going to require significant effort for some implementations, knowing our resources are scarce and our time is limited, we should bite the bullet.

We have been having that discussion for many years now, making incremental progress in that direction with \paper{P1306R1} for example,
and there is a lot of interest in the features presented here, so we have an understanding the effort would justify itself...

But mostly, I think we have a shared understanding getting there is unavoidable.
Indeed, a range of meta info is a pack, and so reflection needs this feature to

Citing \paper{P1306R1}:

\begin{quoteblock}
However, range splicing of dependent arguments is at least an order of magnitude harder to implement than ordinary splicing. We think that not including range splicing gives us a better chance of having reflection in C++26. Especially since, as this paper’s examples demonstrate, a lot can be done without them.
\end{quoteblock}

It would be a missed opportunity to design reflection without the ability to expand ranges of reflections, and we risk introducing mismatched facilites and inconsistencies.
Concepts, Modules, constexpr cmaths, parallel algorithms, etc were all standardized with the understanding of being long-term efforts.

We should note that the implementation concerns for packs outside of templates discussed here would overlap with module implementation concerns.
Implementations retaining an internal representation of template entities during translations are not impacted by that specific concern.

\subsection{Nested Packs? Dependent Pack Names}

Given

\begin{colorblock}
template <typename ...Ts>
struct S {
    using ...types = Ts;
};

using Type = S<int, int>;

template <typename T>
int f() {
    do_something<Type::types...>(); // # 1
    return do_something<typename T::types...>(); // # 2
}
int a = f<Type>();
\end{colorblock}

In \tcode{\#1}, \tcode{Type::types} is a pack of type, which the compiler knows, so that does not present any particular issue.
In \tcode{\#2}, \tcode{T::types} is a dependent name. The compiler doesn't know it is a type, which it needs to in order to resolve parsing ambiguities,
hence the use of \tcode{typename} as disambiguator in \tcode{typename T::types}.

But we don't know that \tcode{T::types} names a pack either.
This is something we need to communicate so that we can diagnose invalid expansion in non-instantiated templates...
but more crucially, we need to be able to identify where the packs are so that we can extract their size prior to extraction.

We use the \tcode{...} \textbf{prefix} to designate a dependent name as a pack.
This is very much purposefully consistent with the declaration of packs; In both places, we are telling the compiler a name is that of a pack.
and the syntax ought to be the same.

The \tcode{...} prefix should be allowed everywhere a nested pack is named, but required only for dependent names.
Unlike \tcode{typename}, (and more like \tcode{template}), the \tcode{...} applies to a name, rather than a fully qualified type.
This is because a qualified name could contain more than one pack.

\subsection{Nested packs? No}

Consider:

\begin{colorblock}
template <typename ...Ts>
struct A {
    using ...inner = Ts;
};

template <typename ...Ts>
struct B {
    using ...outer = Ts;
};

using Matryoshka = B<A<int>>;

\end{colorblock}

Both \tcode{Matryoshka::outer} and \tcode{Matryoshka::outer::inner} are packs.
And this is perfectly fine.

However, we do not have a good model for expanding multiple levels of packs at once,
although this is something that was explored by Barry Revzin.

But it is something that would require a lot more thought and implementation experience.
In particular the generalization of that idea is a cartesian product in the language,
which seems unwieldy.

Ideally, simple ideas should be easy to express and complex ideas should be possible.
If multiple levels of pack expansion are necessary, then other techniques might be more legible.

Maybe there is an argument to be made for the specific case of turning a pack of tuple-like
into a pack of their elements.

So for this paper, In a pack expansion, if the pattern would expand multiple component names, the program is ill-formed.

\begin{colorblock}
DoSomething<Matryoshka>(); // Ok
DoSomething<Matryoshka::...outer...>(); // Ok
DoSomething<Matryoshka::...outer...[0]::...types...>(); // Ok
DoSomething<Matryoshka::...outer::...types...>(); // Error
\end{colorblock}

Like any other packs, packs presented in this paper can also be expanded in step with other (non-nested) packs of the same size.

\subsubsection{Dependent packs need to be substituted before they can be expanded}

\begin{colorblock}
template <typename ...Ts>
struct A {
    using ...types = Ts;
};
template <typename T>
void f() {
  std::variant<T::...types...> v;  // #1
}

f<A<int, int>>();
\end{colorblock}

Currently, the size of packs is inferred from template arguments or (in the case of init captures, structured bindings),
from entities already instantiated.
The presence of packs in dependent names introduces the need to substitute dependent packs before they can be expanded.
This represents extra work for the compiler compared to expanding a pack whose size is known.
However, this is still a lot less work than expressing the same idea without access to nested packs.

\subsubsection{packs in type requirements}

\begin{colorblock}

template <typename T>
concept tuple_like = requires {
    typename T::...elements...;
};

\end{colorblock}

\subsection{Variable Packs}

[TODO]
Initialization?

=> useful init sequences

\subsection{Member Packs}

\begin{center}
\begin{tabular}{l|l}
\begin{minipage}[t]{0.5\textwidth}
\begin{colorblock}
template <typename T>
struct monotype {
    [[no_unique_address]] T elem;
};
\end{colorblock}
\end{minipage}
&
\begin{minipage}[t]{0.5\textwidth}
\begin{colorblock}
template <typename ...Ts>
struct tuple{
    [[no_unique_address]] Ts ... elems;
};
\end{colorblock}
\end{minipage}
\\
\end{tabular}
\end{center}

\subsubsection{Member initializers}

\subsubsection{CTAD}

\subsubsection{Member Expressions, Prvalues, and Packs}

A member expression can name a pack produced by a pr-value.

How many times should \tcode{f} be evaluated?

\begin{colorblock}
template <typename... Ts>
struct S {
    S(Ts... ts) : elems(ts)... {}
    Ts... elems;
};

template <typename... Ts>
S(Ts...) -> S<Ts...>;

auto f (auto... args) {
    std::puts("f");
    return S{args...};
}

int main() {
    return (f(1, 2, 3).elems + ... );
}
\end{colorblock}

\section{Implementations}

\subsubsection{Circle}

\subsubsection{Clang}

Done:

\begin{itemize}
\item Alias packs
\item Member packs
\item Member initializers
\end{itemize}

Todo:

\begin{itemize}
\item Lots of bugs, missing diags
\item Variables
\item Members of anonymous unions
\item Implicit CTAD
\item Member expression materialized temporary reuse
\end{itemize}


\section{Next Steps}

\section{Relation with Reflection}

\section{Future work}

\subsection{Unpacking}

\subsection{Pack Literals}

\subsection{Slicing}

\subsection{Splicing}

\section{Wording}

\section{References}

\renewcommand{\section}[2]{}%
\bibliographystyle{plain}
\bibliography{wg21, extra}

\begin{thebibliography}{9}

\bibitem[N4971]{N4971}
Thomas Köppe
\emph{Working Draft, Standard for Programming Language C++}\newline
\url{https://wg21.link/N4971}


\end{thebibliography}

\end{document}
