% !TeX program = luatex
% !TEX encoding = UTF-8


\documentclass{wg21}

\title{Updated wording and implementation experience for P1481 (constexpr structured bindings)}
\docnumber{P2686R0}
\audience{EWG}
\author{Corentin Jabot}{corentin.jabot@gmail.com}

\usepackage{color, colortbl}
\begin{document}
\maketitle

\section{Abstract}

\paper{P1481R0} proposed to allow reference to constant expressions to be themselves constant expressions,
as a means to support \tcode{constexpr} structured bindings.
This paper reports implementation experience on this proposal and provides updated wording.

\subsection{Context}

The context for this paper can be found in \paper{P1481R0}.
I was not aware to reach the original author, nor do I have the possibility to reproduce the original paper.

The gist of it is that the original author proposed to support \tcode{constexpr} structured binding by making

\begin{colorblock}
constexpr auto[a] = std::tuple(1);
\end{colorblock}

Equivalent to
\begin{colorblock}
constexpr auto __sb = std::tuple(1);
const int& __a = std::get<0>(__sb);
\end{colorblock}

\subsection{Additional motivation}

In addition to the original motivation, if we believe structured bindings are useful (they are, great feature!)
and we also believe in constexpr (as a means to increase type safety, improve runtime performance, etc),
then both features ought to work together.

In addition to that, Expansion Statements (\paper{P1306R1}) aim to add a new kind of for loop
with the express purpose to loop over tuples at compile time.

\begin{colorblock}
auto tup = std::make_tuple(0, ‘a’, 3.14);
// ill-formed without this paper
template for (constexpr auto [idx, member] : std::views::enumerate(meta::data_members_of(^T)) )
    fmt::print("{} {}", idx, foo.[:member:]);
\end{colorblock}


\subsection{History}

Interestingly, this paper was last seen in Kona in 2020.
The concerns were
\begin{itemize}
    \item Lack of implementation
    \item It was presented late in the C++20 cycle
\end{itemize}


\begin{quoteblock}
Encourage further work on this proposal

\begin{tabular}{|c|c|c|c|c|}
    \hline
    SF & F & N & A & SA\\
    \hline
    9 & 16 & 4 & 0 & 0 \\
    \hline
\end{tabular}
\end{quoteblock}

This paper thereby provides an implementation.
I've also update the wording as CWG rewrote the impacted section, and added the wording to support the constexpr keyword
on structured bindings declarations.

\section{Implementation}

\subsection{Circle}

Circle implements constexpr structured bindings - and generally supports initializing references with constant expressions,
and Sean Baxter was not aware that the standard didn't support it.
Sean further observed that this is a core language syntactic sugar and as such, users could expect it to work everywhere.

\subsection{Clang}
I implemented a prototype implementation in the hope to weed out issues.
It is available on \href{https://godbolt.org/z/dWWxcEEf9}{Compiler Explorer}.

Please note that by lack of time, I have not yet published the last version of the implementation, but that should hopefully be
done before Kona.


I don't think the implementation revealed particular issues (my own inaptitudes non-withstanding), I, however, believe \tcode{[basic.odr]}
might need to be tweaked.

\begin{quoteblock}
A variable \tcode{x} that is named by a potentially-evaluated expression \tcode{E} is odr-used by \tcode{E} unless x is a reference that is usable in constant expressions ([expr.const]).
\end{quoteblock}

I don't think this is sufficient. Consider for example,

\begin{colorblock}
void foo() {
    const int a = 1;
    const int& b = a;
    auto l = [] { return b; }; // we should not capture b implicitly here,
                               // even if b is usable in constant expressions
}
\end{colorblock}

In my prototype, I check that the initializer of the reference is itself a constant expression, and that seems to work.

\section{Wording}

\rSec1[expr.const]{Constant expressions}%

\pnum
A variable is \defn{potentially-constant} if
it is constexpr or
it has reference or const-qualified integral or enumeration type.

\pnum
A constant-initialized potentially-constant variable $V$ is
\defn{usable in constant expressions} at a point $P$ if
$V$'s initializing declaration $D$ is reachable from $P$ and
\begin{itemize}
    \item $V$ is constexpr \added{or it is of reference type initialized with a core constant expression},
    \item $V$ is not initialized to a TU-local value, or
    \item $P$ is in the same translation unit as $D$.
\end{itemize}
An object or reference is \defn{usable in constant expressions} if it is
\begin{itemize}
    \item a variable that is usable in constant expressions, or
    \item a template parameter object\iref{temp.param}, or
    \item a string literal object\iref{lex.string}, or
    \item a temporary object of non-volatile const-qualified literal type
    whose lifetime is extended\iref{class.temporary}
    to that of a variable that is usable in constant expressions, or
    \item a non-mutable subobject or reference member of any of the above.
\end{itemize}

\rSec1[dcl.struct.bind]{Structured binding declarations}%
\indextext{structured binding declaration}%
\indextext{declaration!structured binding|see{structured binding declaration}}%

\pnum
A structured binding declaration introduces the \grammarterm{identifier}{s}
$\tcode{v}_0$, $\tcode{v}_1$, $\tcode{v}_2, \dotsc$
of the
\grammarterm{identifier-list} as names
of \defn{structured binding}{s}.
Let \cv{} denote the \grammarterm{cv-qualifier}{s} in
the \grammarterm{decl-specifier-seq} and
\placeholder{S} consist of the \added{\tcode{constexpr} and }\grammarterm{storage-class-specifier}{s} of
the \grammarterm{decl-specifier-seq} (if any).
A \cv{} that includes \tcode{volatile} is deprecated;
see~\ref{depr.volatile.type}.
First, a variable with a unique name \exposid{e} is introduced. If the
\grammarterm{assignment-expression} in the \grammarterm{initializer}
has array type \cvqual{cv1} \tcode{A} and no \grammarterm{ref-qualifier} is present,
\exposid{e} is defined by
\begin{ncbnf}
    \opt{attribute-specifier-seq} \placeholder{S} \cv{} \terminal{A} \exposid{e} \terminal{;}
\end{ncbnf}
and each element is copy-initialized or direct-initialized
from the corresponding element of the \grammarterm{assignment-expression} as specified
by the form of the \grammarterm{initializer}.
Otherwise, \exposid{e}
is defined as-if by
\begin{ncbnf}
    \opt{attribute-specifier-seq} decl-specifier-seq \opt{ref-qualifier} \exposid{e} initializer \terminal{;}
\end{ncbnf}
where
the declaration is never interpreted as a function declaration and
the parts of the declaration other than the \grammarterm{declarator-id} are taken
from the corresponding structured binding declaration.
The type of the \grammarterm{id-expression}
\exposid{e} is called \tcode{E}.
\begin{note}
    \tcode{E} is never a reference type\iref{expr.prop}.
\end{note}

\pnum
If the \grammarterm{initializer} refers to
one of the names introduced by the structured binding declaration,
the program is ill-formed.

\pnum
If \tcode{E} is an array type with element type \tcode{T}, the number
of elements in the \grammarterm{identifier-list} shall be equal to the
number of elements of \tcode{E}. Each $\tcode{v}_i$ is the name of an
lvalue that refers to the element $i$ of the array and whose type
is \tcode{T}; the referenced type is \tcode{T}.
\begin{note}
    The top-level cv-qualifiers of \tcode{T} are \cv.
\end{note}
\begin{example}
    \begin{codeblock}
        auto f() -> int(&)[2];
        auto [ x, y ] = f();            // \tcode{x} and \tcode{y} refer to elements in a copy of the array return value
        auto& [ xr, yr ] = f();         // \tcode{xr} and \tcode{yr} refer to elements in the array referred to by \tcode{f}'s return value
    \end{codeblock}
\end{example}


\rSec2[dcl.constexpr]{The \keyword{constexpr} and \keyword{consteval} specifiers}%
\indextext{specifier!\idxcode{constexpr}}
\indextext{specifier!\idxcode{consteval}}

\pnum
The \keyword{constexpr} specifier shall be applied only to
the definition of a variable or variable template\added{, a structured binding, } or
the declaration of a function or function template.
The \keyword{consteval} specifier shall be applied only to
the declaration of a function or function template.
A function or static data member
declared with the \keyword{constexpr} or \keyword{consteval} specifier
is implicitly an inline function or variable\iref{dcl.inline}.
If any declaration of a function or function template has
a \keyword{constexpr} or \keyword{consteval} specifier,
then all its declarations shall contain the same specifier.


\section{Feature test macros}

\ednote{ In \tcode{[tab:cpp.predefined.ft]}, bump \tcode{__cpp_structured_bindings} to the date of adoption}.


\section{Acknowledgments}

We would like to thank Bloomberg for sponsoring this work.
Thanks to Nicolas Lesser for the original work on \paper{P1481R0}.

\section{References}

\renewcommand{\section}[2]{}%
\bibliographystyle{plain}
\bibliography{wg21, extra}

\begin{thebibliography}{9}

\bibitem[N4885]{N4885}
Thomas Köppe
\emph{Working Draft, Standard for Programming Language C++}\newline
\url{https://wg21.link/N4885}


\end{thebibliography}

\end{document}
