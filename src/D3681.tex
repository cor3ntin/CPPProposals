% !TeX program = luatex
% !TEX encoding = UTF-8


\documentclass{wg21}

\title{\tcode{char\_traits}: Stop the bleeding!}
\docnumber{P3681R0}
\audience{SG-16}
\author{Corentin Jabot}{corentin.jabot@gmail.com}


\usepackage{color, colortbl}
\begin{document}
\maketitle

\section{Abstract}

We propose to deprecate the use of user-defined types for the Traits template parameter of \tcode{std::basic_string},  \tcode{std::basic_string_view}
and iostream-related types. More importantly, we argue that \tcode{std::zstring_view} should not be encumbered by char traits.


\section{Motivation}

Classes such as \tcode{basic_string}, \tcode{basic_string_view}, and the stream types can be customized with a user-provided \grammarterm{CharTraits} type.
This customization has limited use cases, a non-negligible impact on compile times, symbol sizes, and diagnostics messages.

It is also not designed to properly handle Unicode and Unicode character types.

While getting rid of \tcode{char\_traits} or the \tcode{Traits} template parameters is unrealistic, we should consider that this customization
is not worth its cost, and unburden in-flight and future proposals such as \tcode{zstring_view} \paper{P3655R0}.

\section{Problems}

\subsection{\tcode{CharTraits} does not handle multi-bytes encodings}

Every \grammarterm{CharTraits} \tcode{requirement} is specified to \href{https://eel.is/c++draft/strings#tab:char.traits.req}{operate on single values}, \emph{even} for member functions operating on
sequences such as \tcode{compare} and \tcode{copy}.
Consequently, using it to transform multibyte strings will lead to incorrect results.
Comparisons might also be incorrect for shift-state encodings and other non-UTF multibyte encodings.

\subsection{\grammarterm{CharTraits} provides functionalities unrelated to code units}

It is hard to imagine use cases for a non-default implementation of \tcode{assign}, \tcode{length}, \tcode{move}, \tcode{copy}.
\tcode{not_eof}, \tcode{eof} and other functions are only useful for \tcode{iostream}-related facilities and conflate byte reading and encoding concerns.
It is also hard to see how they would be implemented differently from \tcode{std::char_traits}.
Presumably, a user-provided \tcode{Traits} type could circumvent \paper{LWG2959}. However, that this issue has not been fixed (which would be an ABI break),
illustrates the lack of interest in specializing the stream types for \tcode{charN_t}.


\subsection{\tcode{CharTraits} encourages bloat}
Consider:
\begin{colorblock}
void f(std::unordered_map<std::zstring_view, std::zstring_view>);
\end{colorblock}

The presence of \tcode{char_traits} would make the mangled
name 25\% larger using the MSVC mangling scheme. Which, for widely used types, adds up.
This would be justifiable if \tcode{char_traits} provided any value whatsoever... but it does not.

Note that this is less of a problem for the Itanium ABI, which compresses repeated type names.

\subsection{\grammarterm{CharTraits} worsen teachability of text-related notions}

By being incompatible with modern encodings and with its design conflating text encoding, \grammarterm{CharTraits}
\grammarterm{CharTraits} is too often used to encourage incorrect text handling. This is also inconsistent with
the rest of \tcode{std::string} and \tcode{string}-like types, which expose an interface that is otherwise only designed to manipulate sequences of values, without text semantics or encoding considerations.

\section{A plan of action}

\subsection{1. No \tcode{Traits} parameters in new types}

\tcode{zstring_view} and any other such future types or interface should forgo the \tcode{Traits} customization.
The consistency argument is not a very strong one. Never acknowledging mistakes is not useful for the teachability of the language.
We have an opportunity for \tcode{zstring_view} to produce diagnostics and be faster to compile.

There is, however, the question of the conversion from \tcode{std::basic_string} to \tcode{std::basic_zstring_view} (for example).
For which, we have two adequate solutions:
\begin{itemize}
\item Silently drop the user-defined trait type, recognizing that case-insensitivity is a property of a transformation and not an intrinsic property of the string (and any other use of user-provided \tcode{Traits} would be equally assumed to not affect the invariant of the string).
\item Make such conversion ill-formed (forcing users to manually construct the string_view with, for example, \tcode{zstring_view(str.data(), str.size() + 1)}.
\end{itemize}

\subsection{2. \tcode{char\_traits} should not be necessary}

The one intrinsic value of \tcode{char\_traits} is to provide the size of a null-terminated sequence.
\tcode{char_traits<T>::length(str)} is not the most intuitive way to spell \tcode{std::strlen}, so we should add \tcode{constexpr}
overloads of \tcode{strlen} for all character types.
Note that the abandoned \paper{P1944R0} proposed to make \tcode{strlen} constexpr, but that paper was abandoned, and it does not solve
the lack of \tcode{strlen} overloads for \tcode{char8_t}, \tcode{char16_t}, \tcode{char32_t}.
To be clear, we are not proposing to deprecate \tcode{std::char_traits} in this paper, but we also want to ensure its use is unnecessary.

\subsection{3. Replacing off-label \grammarterm{CharTraits} use cases (Casing)}

The one use case we often see purported for user-defined CharTraits types is to let string classes do case-insensitive comparisons.
Which, of course, only works for non-multibyte encoding.
This is partly because there is no easy way to do that in C++.
Because case-insensitive comparisons are a very reasonable thing to want to do, we should facilitate them. \cite{P3688R0} proposes facilities
to compare ranges of ASCII characters. We could also add function objects to ease use with associative containers.
Independently, we should pursue casing and \href{https://www.w3.org/TR/charmod-norm/#definitionCaseFolding}{folding} Unicode views in the C++29 time frame (upper casing or lower casing are not correct ways to compare Unicode sequences).

\subsection{4. Deprecating user-provided \tcode{Traits}}

We should deprecate specialization of any standard type that has a template parameter defaulted to a specialization of \tcode{char_traits},
when the corresponding template argument differs from the default value.

In other words, \tcode{basic_string<char, MyTrait<char>>} should be deprecated.
This is something that implementation can diagnose with depreciation warnings (at least as long as the specialization of \tcode{basic_string} is complete).

\subsection{5. Longer-term prospects}

We could, in a future version of C++, make \tcode{basic_string<char, MyTrait<char>>} ill-formed.
This would allow an implementation not to use Traits in its implementations.

Because the \tcode{Traits} parameter is often not the last parameter in standard types that use it, it would be a serious breaking change to remove it.
And because of ABI concerns, it's unlikely to ever be removed from implementations.

However, if it is neither used nor user-provided, we solve most problems created by char traits, except that:
\begin{itemize}
\item Providing an allocator to string will remain a bit cumbersome (which we could solve by adding an alias template)
\item Symbol size will not improve.
\end{itemize}

However, if an implementation decided to change its ABI one day, or provide a way to opt-out of ABI stability, knowing the \tcode{Trait} parameter need not be mangled
could lead to improvements.


\bibliographystyle{plain}
\bibliography{wg21, extra}

\renewcommand{\section}[2]{}%

\begin{thebibliography}{9}

 \bibitem[P3688R0]{P3688R0}
Jan Schultke, Corentin Jabot
\emph{ASCII character utilities}\newline
\url{https://wg21.link/P3688R0}

\bibitem[N5008]{N5008}
Thomas Köppe
\emph{Working Draft, Standard for Programming Language C++}\newline
\url{https://wg21.link/N5008}


\end{thebibliography}

\end{document}
