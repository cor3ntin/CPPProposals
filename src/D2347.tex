% !TeX document-id = {9322a846-f757-4574-9231-a2e85c743b21}
% !TeX program = luatex
% !TEX encoding = UTF-8


\RequirePackage{luatex85}%
\documentclass{wg21}

\usepackage{luatexja-fontspec}
\RequirePackageWithOptions{fontspec}
\usepackage{newunicodechar}

\setmainfont{Noto Sans}

\newfontfamily{\fallbackfont}{Noto Sans}
\DeclareTextFontCommand{\textfallback}{\fallbackfont}
\newunicodechar{ẞ}{\textfallback{ẞ}}


\newcommand{\UnicodeLetter}[1]{\textbf{\textcolor{BrickRed}{\Large\tcode{#1}}}}


\title{Argument type deduction for non-trailing parameter packs}
\docnumber{D2347R2}
\audience{EWG}
\author{Corentin Jabot}{corentin.jabot@gmail.com}
\authortwo{Bruno Manganelli}{bruno.manga95@gmail.com}

\begin{document}
\maketitle

\section{Abstract}

We propose that, during template argument deduction, a single non-trailing parameters pack be deduced solely based on the arity of the list of arguments.
This paper proposes the mechanism for function calls, but a generalization is explored.


\section{Revisions}

\subsection{Revision 2}

Exploration and implementation experience of argument deduction in more contexts (class, function address, aliases, partial specialization).

\subsection*{Revision 1}
Fix many typos in the wording, remove incorrect wording examples.


\section{Status and target vehicle}

The proposal has been implemented, and we propose some wording, although that wording is only for function calls,
and is insufficient.
As requested by EWG, the paper also explores non-trailing pack deduction for type deduction outside of function calls,
and this has been implemented too, but there are some open questions, some risks (the authors are not sure they covered all the bases),
and no wording.

We can either:
\begin{itemize}
\item Target the whole proposal at C++26
\item Try to ship type deduction for function calls in C++23, as this covers the original \tcode{source_location} use case and generalize later.
\end{itemize}

This is a scenario where incremental improvements are possible.

\section{Motivation}

\subsection{\tcode{source_location}}

Our primary motivation for this change is to improve the usability of \tcode{source_location}.
Many loggers, especially those using fmt, such as \tcode{spdlog} offer a log function of the following form:

\begin{colorblock}
void log(string_view formatString, auto&&...args);
\end{colorblock}
Which can then be called with arbitrary arguments: \tcode{log("Hello {}!", "world");}

Naturally, we would like to extend this function to support source location, and offer a more useful logging framework:

\begin{colorblock}
void log(string_view formatString, auto&&...args, source_location loc = source_location::current());
\end{colorblock}

Unfortunately, this is not possible because non-trailing packs cannot be deduced!
Folks on \href{https://stackoverflow.com/questions/57547273/how-to-use-source-location-in-a-variadic-template-function}{Stackoverflow} have found several clever workarounds, all of which involves using extra types or templates.
So, even if the use case can be somewhat covered, it relies on rather arcane solutions that are worse for diagnostics, compile-time, etc.

\subsection{Accessing the last argument of a pack}

It is sometimes useful to handle the last parameter differently.
The following example is taken from \paper{P0478R0}:

\begin{colorblock}
template <class... Args, class Last>
void signal(Args... args, Last last) {
    // callback expects 5 arguments, and we only want to pass it the first 5
    if constexpr(sizeof... (Args) > 5) {
        return signal(args...);
    } else if constexpr (sizeof... (Args) == 4) {
        callback(args..., last);
    } else {
        callback(args...);
    }
}
\end{colorblock}

Or consider that function which prints its arguments:

\begin{colorblock}
void print(auto&&... args, auto && last) {
    if constexpr(sizeof...(args) > 0)
        ((std::cout << args << ", "),...);
    std::cout << last << "\n";
}
\end{colorblock}

This is currently rather difficult.

Or a usage of \tcode{apply} that handles the last argument differently:

\begin{colorblock}

std::apply([](auto&&..., auto && last) {
    assert(last == 3);
}, std::tuple{1, 2, 3});
\end{colorblock}

A \tcode{apply_last} function can be written, albeit it's a bit cumbersome.

\begin{colorblock}
template <class F, class Tuple>
constexpr decltype(auto) apply_last(F &&f, const Tuple &t) {
    return [&]<auto... I>(std::index_sequence<I...>) {
        return f(std::get<std::tuple_size_v<std::remove_reference_t<Tuple>> - 1>(t),
        std::get<I>(t)...);
    }(std::make_index_sequence<std::tuple_size_v<std::remove_cvref_t<Tuple>> -1>{});
}
\end{colorblock}

\subsection{Consistent interfaces with variadic arguments}

We might consider providing N ranges overloads to \tcode{std::transform}, \tcode{std::merge} and similar algorithms,
such that they are consistent with the order of parameters of existing 1 and 2 ranges overloads.

We would also argue that \tcode{visit} would be more intuitive if the variants were the first parameters.

Lifting the limitations on where a parameter pack can appear gives more flexibility in API design and usage.

\subsection{keys-value interfaces}

Another great use, suggested by the author of sol2, is an interface that takes a number of keys (corresponding to a path or a nested addressing scheme of some sort) and a value.
Users' expectations and logic would dictate that the value would be the last parameter. This proposal, therefore, allows for less surprising APIs.

\begin{colorblock}
// given a lua table (or json doccument, or something of that nature), set("a", "b", "c", 42)
// table["a"]["b"]["c"] becomes 42
void set_value_in_lua_table(auto&& table, auto... keys, auto value);
\end{colorblock}


\section{Design}

We propose that if there is one (and only one) parameter pack in a function,
the arity of that parameter pack, when deduced, is the number of not yet deduced function arguments, minus the number of non-defaulted
parameters following the pack.

The general idea is to deduce a single pack and to deduce the size of that pack such that once expanded,
the argument list matches the size of the parameter list, excluding any defaulted parameter.

\begin{colorblock}
void f(auto a, auto...b, auto c, auto d);
void g(auto a, auto...b, auto c, int d = 0);
void h(auto a, auto...b, int c = 0);


f(0, 0, 0, 0);      // size of b is deduced to be 1
f(0, 0, 0, 0, 0);   // size of b is deduced to be 2
f(0, 0, 0);         // size of b is deduced to be 0

g(0, 0);           // size of b is deduced to be 0
g(0, 0, 0, 0);     // size of b is deduced to be 2

h(0, 0);           // size of b is deduced to be 1
h(0, 0, 0);        // size of b is deduced to be 2

\end{colorblock}

Unlike \paper{P0478R0}, we do not propose that the compiler should try to deduce a valid overload with or without default parameter or apply
a more clever logic.
This proposal is based solely on the arity of the arguments.
This is why we consider this paper lifts a restriction rather than introducing a new feature.
We do not propose any changes to overload resolution nor the ordering of function templates.

As such, a limitation of this proposal is that if a parameter pack is immediately followed by a parameter P with a default value,
it is not possible for the caller to provide a value for P.

\begin{colorblock}
void f(auto...a, int c = 42);
f()     // a is empty, c == 42
f(1)    // a is of size 1,  c == 42
f(1, 2) // a is of size 2,  c == 42

\end{colorblock}

We found that trying to be clever here is not likely to be worth it:

\begin{itemize}
\item Generating automatically extra overloads for each defaulted parameter has a cost in compile times.
\item It would blur the lines between template argument deduction and overload resolution.
\end{itemize}

If one really needs a defaulted argument immediately following a pack,
it is always possible to manually craft an overloads set that would allow a parameter to be both provided and defaulted, for example:

\begin{colorblock}
template <typename... T>
void f(T&&... args, source_location loc = {})
requires (!(std::same_as<T, source_location>||...));

void f(auto&&... args, source_location loc);
\end{colorblock}


\section{Interesting and breaking changes}

\subsection{Non-trailing packs in function are no longer non-deduced}

\begin{colorblock}
void ambiguous(auto..., auto);    // #1
void ambiguous(auto, auto...);    // #2
void ambiguous(auto..., auto...); // #3

ambiguous(1, 2);

\end{colorblock}

In C++20, \#1 and \#3 have undeduced packs, so the only viable candidate after overload resolution is \#2.
With this proposal, all arguments of both \#1 and \#2 can be deduced. None of the arguments of \#3 can be deduced (multiple packs are never deduced),
and after overload resolution, both  \#1 and \#2 are viable overloads, and the call is ambiguous.
A program that was well-formed in C++20 would be ill-formed with this proposal.

Note that in that scenario, it is not possible to resolve the ambiguity with a cast, for example.
This is not a new problem; functions with undeduced packs cannot be explicitly named when there are ambiguities in C++20.

\subsection{Multi levels templates}

The implementation in clang revealed this interesting case.

\begin{colorblock}
template<typename...T>
struct S {
    template<typename...U>
    void g(U &&...u, T &&...t) {}
};

void test_nested_packs() {
    S<int>().g(0, 1);  // #1
}
\end{colorblock}
After instantiation of \tcode{S}, \tcode{g} is
\begin{colorblock}
template<typename...U>
void g(U &&...u, int);
\end{colorblock}

In C++20, \tcode{U...} is an undeduced context, so \#1 is ill-formed.
With is proposal, \tcode{U} can be deduced to \tcode{int} and the program is well-formed.


\section{Deduction in other contexts: Implementation and use cases}

In addition to function calls, type deduction (where a pack can appear) can occur when
\begin{itemize}
    \item Instantiating template classes
    \item Taking the address of a function template
    \item Partial ordering
\end{itemize}

EWG asked that these things are explored.
This section is a report on that exploration.
\textbf{Note that type deduction for function arguments could be standardized independently of everything else and would provide benefits for users
of \tcode{source_location} in the C++23 time frame.}
The exploration shows that this can be expanded later with a consistent design.


\subsection{Type deduction during class instantiation}

A motivating use case is a meta-type that expand to the last argument of a template.
\begin{colorblock}
template <typename...>
struct mp_list{};

// Extracting the last element in a type list is made much easier
template<class L>
struct pop_back_impl{};
template<template<class...> class L, class... T, class Last>
struct pop_back_impl<L<T..., Last>> {
    using type = L<T...>;
};

template<class L>
using pop_back = typename pop_back_impl<L>::type;
\end{colorblock}

This has been implemented in \href{https://godbolt.org/z/9zs3KEWWx}{Compiler Explorer}.
Note that currently, non-trailing parameter packs in class template heads are ill-formed, whereas they are only non-deduced for function calls.
Unlike function calls, this has, therefore, no risk of affecting existing code.

This works exactly like for function calls.
Each time deduction happens, we compare a list of arguments to a list of parameters, and so we can deduce the size of a single pack
from the arity of a single pack.
There cannot be multiple packs for this to work \href{https://godbolt.org/z/rTjhdYrY6}{Compiler Explorer}.
This is one of the reasons this paper does not replace \paper{P1858R2}.

The same logic would apply anywhere a type or an alias is instanciated, for example:

\begin{colorblock}
template<class..., class T>
using  Last = T;
static_assert(std::same_as<Last<int, int, float>, float>);
\end{colorblock}

\subsection{Function address}

The address of a function template can be deduced in the same way as being only a special case of argument deduction,
where the list of arguments is extracted from the type of the LHS declaration.
[\href{https://godbolt.org/z/d68bxfPWT}{Compiler Explorer}]

\begin{colorblock}
auto f(auto... a, int x = 0) {
    return(a +...) * x;
}
int g() {
    using F = int(int, int, int);
    F& func = f;

    return func(20, 1, 2);
}
\end{colorblock}


\subsection{Partial specialization}

Partial specialization also follows regular type deduction rules
[\href{https://godbolt.org/z/18znK877c}{Compiler Explorer}].

\subsection{CTAD}

Class Template Arguments Deduction also follows naturally from the same rule.
Nothing specific had to be modified in Clang to make this work [\href{https://godbolt.org/z/hMEav8z7j }{Compiler Explorer}]

CTAD for aggregate has not been explored.

\subsection{Defaulted template parameters}

There is a design question of whether to allow template heads of the form
\begin{colorblock}
template <typename...T, typename U = void>
\end{colorblock}

It might be useful SFINAE tricks or to declare verbose types used subsequently in the declaration,
but like non-trailing defaulted parameters, they could never be deduced and would always have their default value.
The two design choices are
\begin{itemize}
\item Ill-formed (this is what the implementation is doing)
\item Well-formed and always hold the default value.
\end{itemize}

\section{Previous works}

\paper{P0478R0} was first presented in Issaquah and offered a more complicated approach to some of the problems presented here.
Concerns were expressed mostly because \paper{P0478R0} proposed to modify rules around overload resolution, which the current paper does not. It is, therefore, a lot simpler.

We also consider that \tcode{source_location} demands that this question be revisited.

\subsection{State of the implementation}

There was no major issue implementing what is covered in this paper in clang; although I didn't probably cover all scenarios, this is very much a prototype.
I am hoping that the existence of that prototype might let us uncover scenarios that we might think about.
There is probably nothing worth reporting on the implementation.
It was probably a week worth of work, which tells more about how bad a compiler writer I am than it does about
the actual amount of work required for this feature.

It was necessary for clang to keep track of the index of the template parameter each argument was expanded to.
Some work went into properly handling explicit template arguments.


\section{Alternatives and future evolutions}

\subsection{Generalized pack manipulation facilities}

Several proposals, including \paper{P1858R2} and \paper{P1306R1}, would make manipulating pack simpler,
and we hope these papers progress.
However, neither of these could address the \tcode{source_location} issue (which we realize is rather specific) and are not as elegant in some use cases.

\subsection{Pack separators}

Circle provides a syntax to denotes the end of a pack, which allows separating a pack from subsequent defaulted arguments,
and also to support the deduction of multiple packs \href{https://github.com/seanbaxter/QDA/blob/master/README.md#parameter-packs} as described in the {Circle documentation}. We are not proposing a similar feature, but it is something that we could consider in the future in a backward-compatible manner.

\subsection{Injecting multiple overloads for different combinations of defaulted parameters}

This is discussed in a previous section, and it is a direction we rejected because of its cost and complexity.
It is important to note that adopting such a feature in the future would be a breaking change in regard to this paper.

%\subsection{Non-trailing template parameters of class and alias template}
%
%Similar restrictions on the positioning of parameter packs exist for class templates and alias.
%We could lift these restrictions using the same heuristic - aka using the number of template arguments
%to deduce the size of a single non-trailing parameter pack.
%
%\begin{colorblock}
%template <typename... T, typename> // ill-formed
%struct S;
%\end{colorblock}
%
%This is not proposed in this paper.

\section{Implementation}

%Both this paper and \paper{P0478R0} have been implemented in clang with no difficulties.
% In particular, this proposal required a very small amount of work.

\section{Wording (For function templates ONLY!)}

\rSec2[temp.deduct]{Template argument deduction}

%\rSec3[temp.deduct.general]{General}
%
%\pnum
%When a
%function template
%specialization is referenced, all of the
%template arguments shall have values.
%The values can be
%explicitly specified or, in some cases, be deduced from the use
%or obtained from default
%\grammarterm{template-argument}{s}.
%\begin{example}
%    \begin{codeblock}
%        void f(Array<dcomplex>& cv, Array<int>& ci) {
%            sort(cv);                     // calls \tcode{sort(Array<dcomplex>\&)}
%            sort(ci);                     // calls \tcode{sort(Array<int>\&)}
%        }
%    \end{codeblock}
%    and
%    \begin{codeblock}
%        void g(double d) {
%            int i = convert<int>(d);      // calls \tcode{convert<int,double>(double)}
%            int c = convert<char>(d);     // calls \tcode{convert<char,double>(double)}
%        }
%    \end{codeblock}
%\end{example}
%
%\pnum
%When an explicit template argument list is specified, if the
%given \grammarterm{template-id} is not valid\iref{temp.names},
%type deduction fails.
%Otherwise, the specified template argument values are substituted for the
%corresponding template parameters as specified below.
%
%\pnum
%After this substitution is performed, the function parameter type
%adjustments described in~\ref{dcl.fct} are performed.
%\begin{example}
%    A parameter type of ``\tcode{void (const int, int[5])}'' becomes
%    ``\tcode{void(*)(int,int*)}''.
%\end{example}
%\begin{note}
%    A top-level qualifier in a function parameter declaration does not affect
%    the function type but still affects the type of the function parameter
%    variable within the function.
%\end{note}
%\begin{example}
%    \begin{codeblock}
%        template <class T> void f(T t);
%        template <class X> void g(const X x);
%        template <class Z> void h(Z, Z*);
%
%        int main() {
%            // \#1: function type is \tcode{f(int)}, \tcode{t} is non \tcode{const}
%            f<int>(1);
%
%            // \#2: function type is \tcode{f(int)}, \tcode{t} is \tcode{const}
%            f<const int>(1);
%
%            // \#3: function type is \tcode{g(int)}, \tcode{x} is \tcode{const}
%            g<int>(1);
%
%            // \#4: function type is \tcode{g(int)}, \tcode{x} is \tcode{const}
%            g<const int>(1);
%
%            // \#5: function type is \tcode{h(int, const int*)}
%            h<const int>(1,0);
%        }
%    \end{codeblock}
%\end{example}
%
%\pnum
%\begin{note}
%    \tcode{f<int>(1)} and \tcode{f<const int>(1)} call distinct functions
%    even though both of the functions called have the same function type.
%\end{note}
%
%\pnum
%The resulting substituted and adjusted function type is used as
%the type of the function template for template argument
%deduction.  If a template argument has not been deduced and its
%corresponding template parameter has a default argument, the
%template argument is determined by substituting the template
%arguments determined for preceding template parameters into the
%default argument. If the substitution results in an invalid type,
%as described above, type deduction fails.
%
%\begin{example}
%    \begin{codeblock}
%        template <class T, class U = double>
%        void f(T t = 0, U u = 0);
%
%        void g() {
%            f(1, 'c');        // \tcode{f<int,char>(1,'c')}
%            f(1);             // \tcode{f<int,double>(1,0)}
%            f();              // error: \tcode{T} cannot be deduced
%            f<int>();         // \tcode{f<int,double>(0,0)}
%            f<int,char>();    // \tcode{f<int,char>(0,0)}
%        }
%    \end{codeblock}
%\end{example}
%
%When all template arguments have been deduced or obtained from
%default template arguments, all uses of template parameters in
%the template parameter list of the template
%are replaced with the corresponding deduced
%or default argument values. If the substitution results in an
%invalid type, as described above, type deduction fails.
%If the function template has associated constraints\iref{temp.constr.decl},
%those constraints are checked for satisfaction\iref{temp.constr.constr}.
%If the constraints are not satisfied, type deduction fails.
%In the context of a function call, if type deduction has not yet failed, then
%for those function parameters for which the function call has arguments,
%Each function parameter with a type that was non-dependent
%before substitution of any explicitly-specified template arguments
%is checked against its corresponding argument;
%if the corresponding argument cannot be implicitly converted
%to the parameter type, type deduction fails.
%\begin{note}
%    Overload resolution will check the other parameters, including
%    parameters with dependent types in which
%    no template parameters participate in template argument deduction and
%    parameters that became non-dependent due to substitution of
%    explicitly-specified template arguments.
%\end{note}
%If type deduction has not yet failed, then
%all uses of template parameters in the function type are
%replaced with the corresponding deduced or default argument values.
%If the substitution results in an invalid type, as described above,
%type deduction fails.
%\begin{example}
%    \begin{codeblock}
%        template <class T> struct Z {
%            typedef typename T::x xx;
%        };
%        template <class T> concept C = requires { typename T::A; };
%        template <C T> typename Z<T>::xx f(void *, T);          // \#1
%        template <class T> void f(int, T);                      // \#2
%        struct A {} a;
%        struct ZZ {
%            template <class T, class = typename Z<T>::xx> operator T *();
%            operator int();
%        };
%        int main() {
%            ZZ zz;
%            f(1, a);              // OK, deduction fails for \#1 because there is no conversion from \tcode{int} to \tcode{void*}
%            f(zz, 42);            // OK, deduction fails for \#1 because \tcode{C<int>} is not satisfied
%        }
%    \end{codeblock}
%\end{example}
%
%\pnum
%At certain points in the template argument deduction process it is necessary
%to take a function type that makes use of template parameters and replace those
%template parameters with the corresponding template arguments. This is done at
%the beginning of template argument deduction when any explicitly specified
%template arguments are substituted into the function type, and again at the end
%of template argument deduction when any template arguments that were deduced or
%obtained from default arguments are substituted.
%
%\pnum
%The substitution occurs in all types and expressions that are used in the function
%type and in template parameter declarations. The expressions include not only
%constant expressions such as those that appear in array bounds or as nontype
%template arguments but also general expressions (i.e., non-constant expressions)
%inside \tcode{sizeof}, \tcode{decltype}, and other contexts that allow non-constant
%expressions. The substitution proceeds in lexical order and stops when
%a condition that causes deduction to fail is encountered.
%If substitution into different declarations of the same function template would
%cause template instantiations to occur in a different order or not at all,
%the program is ill-formed; no diagnostic required.
%\begin{note}
%    The equivalent substitution in exception specifications is
%    done only when the \grammarterm{noexcept-specifier} is instantiated,
%    at which point a program is ill-formed
%    if the substitution results in an invalid type or expression.
%\end{note}
%\begin{example}
%    \begin{codeblock}
%        template <class T> struct A { using X = typename T::X; };
%        template <class T> typename T::X f(typename A<T>::X);
%        template <class T> void f(...) { }
%        template <class T> auto g(typename A<T>::X) -> typename T::X;
%        template <class T> void g(...) { }
%        template <class T> typename T::X h(typename A<T>::X);
%        template <class T> auto h(typename A<T>::X) -> typename T::X;   // redeclaration
%        template <class T> void h(...) { }
%
%        void x() {
%            f<int>(0);        // OK, substituting return type causes deduction to fail
%            g<int>(0);        // error, substituting parameter type instantiates \tcode{A<int>}
%            h<int>(0);        // ill-formed, no diagnostic required
%        }
%    \end{codeblock}
%\end{example}
%
%\pnum
%If a substitution results in an invalid type or expression, type deduction fails. An
%invalid type or expression is one that would be ill-formed, with a diagnostic
%required, if written using the substituted arguments.
%\begin{note}
%    If no
%    diagnostic is required, the program is still ill-formed. Access checking is done
%    as part of the substitution
%    process.
%\end{note}
%Only invalid types and expressions in the immediate context of
%the function type,
%its template parameter types,
%and its \grammarterm{explicit-specifier}
%can result in a deduction failure.
%\begin{note}
%    The substitution into types and expressions can result
%    in effects such as the instantiation of class template specializations and/or
%    function template specializations, the generation of implicitly-defined functions,
%    etc. Such effects are not in the ``immediate context'' and can result in the
%    program being ill-formed.
%\end{note}
%
%\pnum
%A \grammarterm{lambda-expression} appearing in a function type
%or a template parameter is not considered part of the immediate context
%for the purposes of template argument deduction.
%\begin{note}
%    The intent is to avoid requiring implementations to deal with
%    substitution failure involving arbitrary statements.
%    \begin{example}
%        \begin{codeblock}
%            template <class T>
%            auto f(T) -> decltype([]() { T::invalid; } ());
%            void f(...);
%            f(0);               // error: invalid expression not part of the immediate context
%
%            template <class T, std::size_t = sizeof([]() { T::invalid; })>
%            void g(T);
%            void g(...);
%            g(0);               // error: invalid expression not part of the immediate context
%
%            template <class T>
%            auto h(T) -> decltype([x = T::invalid]() { });
%            void h(...);
%            h(0);               // error: invalid expression not part of the immediate context
%
%            template <class T>
%            auto i(T) -> decltype([]() -> typename T::invalid { });
%            void i(...);
%            i(0);               // error: invalid expression not part of the immediate context
%
%            template <class T>
%            auto j(T t) -> decltype([](auto x) -> decltype(x.invalid) { } (t));   // \#1
%            void j(...);                                                            // \#2
%            j(0);               // deduction fails on \#1, calls \#2
%        \end{codeblock}
%    \end{example}
%\end{note}
%
%\pnum
%\begin{example}
%    \begin{codeblock}
%        struct X { };
%        struct Y {
%            Y(X) {}
%        };
%
%        template <class T> auto f(T t1, T t2) -> decltype(t1 + t2);     // \#1
%        X f(Y, Y);                                                      // \#2
%
%        X x1, x2;
%        X x3 = f(x1, x2);   // deduction fails on \#1 (cannot add \tcode{X+X}), calls \#2
%    \end{codeblock}
%\end{example}
%
%\pnum
%\begin{note}
%    Type deduction can fail for the following reasons:
%    \begin{itemize}
%        \item Attempting to instantiate a pack expansion containing multiple packs of differing lengths.
%        \item
%        Attempting to create an array with an element type that is \tcode{void}, a
%        function type, or a reference type, or attempting
%        to create an array with a size that is zero or negative.
%        \begin{example}
%            \begin{codeblock}
%                template <class T> int f(T[5]);
%                int I = f<int>(0);
%                int j = f<void>(0);             // invalid array
%            \end{codeblock}
%        \end{example}
%        \item
%        Attempting to use a type that is not a class or enumeration type in a qualified name.
%        \begin{example}
%            \begin{codeblock}
%                template <class T> int f(typename T::B*);
%                int i = f<int>(0);
%            \end{codeblock}
%        \end{example}
%        \item
%        Attempting to use a type in a \grammarterm{nested-name-specifier} of a
%        \grammarterm{qualified-id} when
%        that type does not contain the specified member, or
%        \begin{itemize}
%            \item
%            the specified member is not a type where a type is required, or
%            \item
%            the specified member is not a template where a template is required, or
%            \item
%            the specified member is not a non-type where a non-type is required.
%        \end{itemize}
%        \begin{example}
%            \begin{codeblock}
%                template <int I> struct X { };
%                template <template <class T> class> struct Z { };
%                template <class T> void f(typename T::Y*) {}
%                template <class T> void g(X<T::N>*) {}
%                template <class T> void h(Z<T::TT>*) {}
%                struct A {};
%                struct B { int Y; };
%                struct C {
%                    typedef int N;
%                };
%                struct D {
%                    typedef int TT;
%                };
%
%                int main() {
%                    // Deduction fails in each of these cases:
%                    f<A>(0);          // \tcode{A} does not contain a member \tcode{Y}
%                    f<B>(0);          // The \tcode{Y} member of \tcode{B} is not a type
%                    g<C>(0);          // The \tcode{N} member of \tcode{C} is not a non-type
%                    h<D>(0);          // The \tcode{TT} member of \tcode{D} is not a template
%                }
%            \end{codeblock}
%        \end{example}
%        \item
%        Attempting to create a pointer to reference type.
%        \item
%        Attempting to create a reference to \tcode{void}.
%        \item
%        Attempting to create ``pointer to member of \tcode{T}'' when \tcode{T} is not a
%        class type.
%        \begin{example}
%            \begin{codeblock}
%                template <class T> int f(int T::*);
%                int i = f<int>(0);
%            \end{codeblock}
%        \end{example}
%        \item
%        Attempting to give an invalid type to a non-type template parameter.
%        \begin{example}
%            \begin{codeblock}
%                template <class T, T> struct S {};
%                template <class T> int f(S<T, T()>*);
%                struct X {};
%                int i0 = f<X>(0);
%            \end{codeblock}
%        \end{example}
%
%        \item
%        Attempting to perform an invalid conversion in either a template
%        argument expression, or an expression used in the function
%        declaration.
%        \begin{example}
%            \begin{codeblock}
%                template <class T, T*> int f(int);
%                int i2 = f<int,1>(0);           // can't conv \tcode{1} to \tcode{int*}
%            \end{codeblock}
%        \end{example}
%
%        \item
%        Attempting to create a function type in which a parameter has a type
%        of \tcode{void}, or in which the return type is a function type
%        or array type.
%    \end{itemize}
%\end{note}
%
%\pnum
%\begin{example}
%    In the following example,
%    assuming a \tcode{signed char}
%    cannot represent the value 1000,
%    a narrowing conversion\iref{dcl.init.list}
%    would be required
%    to convert the \grammarterm{template-argument}
%    of type \tcode{int} to \tcode{signed char},
%    therefore substitution fails for the
%    second template\iref{temp.arg.nontype}.
%
%    \begin{codeblock}
%        template <int> int f(int);
%        template <signed char> int f(int);
%        int i1 = f<1000>(0);            // OK
%        int i2 = f<1>(0);               // ambiguous; not narrowing
%    \end{codeblock}
%\end{example}


\rSec3[temp.deduct.call]{Deducing template arguments from a function call}

//...

For a function parameter pack \removed{that occurs at the end
of the \grammarterm{parameter-declaration-list}},\added{the size of the pack \tcode{N} is deduced to be
the number of arguments in the call, minus the number of arguments preceding \tcode{A},
and minus one for each function template parameter following P that has no default argument.}

\changed{d}{D}eduction is performed for \changed{each remaining}{the next N} argument\added{s} of the call,
taking the type \tcode{P}
of the \grammarterm{declarator-id} of the function parameter pack
as the corresponding function template parameter type.
Each deduction deduces template arguments for subsequent positions in
the template parameter packs expanded by the function parameter pack.
When a function parameter pack appears in a non-deduced
context\iref{temp.deduct.type}, the type of that pack is
never deduced.


%\added{Let N be the number of remaining non-defaulted function template parameters, and K be the number of remaining arguments of the call.}
%
%For a function parameter pack \removed{that occurs at the end
%    of the \grammarterm{parameter-declaration-list}},
%deduction is performed for \changed{each remaining}{the next K-N} argument\added{s} of the call,
%taking the type \tcode{P}
%of the \grammarterm{declarator-id} of the function parameter pack
%as the corresponding function template parameter type.
%Each deduction deduces template arguments for subsequent positions in
%the template parameter packs expanded by the function parameter pack.
%When a function parameter pack appears in a non-deduced
%context\iref{temp.deduct.type}, the type of that pack is
%never deduced.

\begin{example}
\begin{removedblock}
\begin{codeblock}
    template<class ... Types> void f(Types& ...);
    template<class T1, class ... Types> void g(T1, Types ...);
    template<class T1, class ... Types> void g1(Types ..., T1);

    void h(int x, float& y) {
        const int z = x;
        f(x, y, z);                   // \tcode{Types} deduced as \tcode{int}, \tcode{float}, \tcode{const int}
        g(x, y, z);                   // \tcode{T1} deduced as \tcode{int}; \tcode{Types} deduced as \tcode{float}, \tcode{int}
        g1(x, y, z);                  // error: \tcode{Types} is not deduced
        g1<int, int, int>(x, y, z);   // OK, no deduction occurs
    }
\end{codeblock}
\end{removedblock}

\begin{addedblock}
\begin{codeblock}
template<class ... Types> void f(Types& ...);
template<class T1, class ... Types> void g(T1, Types ...);
template<class T1, class ... Types> void g1(Types ..., T1);
template<class ... Types> void g2(int, Types ..., int = 0);
template<class ... Types, class T2, class... OtherTypes> void g3(Types ..., T2, OtherTypes...);

void h(int x, float& y) {
    const int z = x;
    f(x, y, z);  // \tcode{Types} deduced as \tcode{int}, \tcode{float}, \tcode{const int}
    g(x, y, z);  // \tcode{T1} deduced as \tcode{int}; \tcode{Types} deduced as \tcode{float}, \tcode{int}
    g1(x, y, z); // \tcode{Types} deduced as \tcode{float}, \tcode{int}, \tcode{T1} deduced as \tcode{int};
    g2(x, x);    // \tcode{Types} deduced as \tcode{int};
    g2(x);       // \tcode{sizeof...(Types) == 0};
    g3(x, y, z); // error: \tcode{Types} is not deduced, \tcode{OtherTypes} is not deduced
    g3<int, int, int>(x, y, z);   // OK, no deduction occurs
}
\end{codeblock}
\end{addedblock}

\end{example}

\rSec3[temp.deduct.partial]{Deducing template arguments during partial ordering}

\pnum
Two sets of types are used to determine the partial ordering.  For each of
the templates involved there is the original function type and the
transformed function type.
\begin{note}
    The creation of the transformed type is described in [temp.func.order].
\end{note}
The deduction process uses the
transformed type as the argument template and the original type of the
other template as the parameter template.  This process is done twice
for each type involved in the partial ordering comparison: once using
the transformed template-1 as the argument template and template-2 as
the parameter template and again using the transformed template-2 as
the argument template and template-1 as the parameter template.

\pnum
The types used to determine the ordering depend on the context in which
the partial ordering is done:

\begin{itemize}
    \item
    In the context of a function call, the types used are those function parameter types
    for which the function call has arguments.
    \item
    In the context of a call to a conversion function, the return types of
    the conversion function templates are used.
    \item
    In other contexts\iref{temp.func.order} the function template's function
    type is used.
\end{itemize}

\pnum
Each type nominated above from the parameter template and the corresponding \added{$i^\text{th}$} type from the
argument template are used as the types of \tcode{P} and \tcode{A}.
\begin{addedblock}
Let \tcode{PCount} be the numbers of types of the parameter template, and \tcode{ACount} be the numbers of type of the argument template.
\end{addedblock}

Before the partial ordering is done, certain transformations are performed
on the types used for partial ordering:

\begin{itemize}
    \item
    If
    \tcode{P}
    is a reference type,
    \tcode{P}
    is replaced by the type referred to.
    \item
    If
    \tcode{A}
    is a reference type,
    \tcode{A}
    is replaced by the type referred to.
\end{itemize}

\pnum
If both
\tcode{P}
and
\tcode{A}
were reference types (before being replaced with the type referred to
above), determine which of the two types (if any) is more cv-qualified
than the other; otherwise the types are considered to be equally
cv-qualified for partial ordering purposes. The result of this
determination will be used below.

\pnum
Remove any top-level cv-qualifiers:
\begin{itemize}
    \item
    If
    \tcode{P}
    is a cv-qualified type,
    \tcode{P}
    is replaced by the cv-unqualified version of
    \tcode{P}.
    \item
    If
    \tcode{A}
    is a cv-qualified type,
    \tcode{A}
    is replaced by the cv-unqualified version of
    \tcode{A}.
\end{itemize}

\pnum
Using the resulting types
\tcode{P}
and
\tcode{A},
the deduction is then done as described in~\ref{temp.deduct.type}.
If \tcode{P} is a function parameter pack, the type \tcode{A} of \changed{each remaining}{The next $ACount-Pcount-i$}
parameter type\added{s} of the argument template is compared with the type \tcode{P} of
the \grammarterm{declarator-id} of the function parameter pack. Each comparison
deduces template arguments for subsequent positions in the template parameter
packs expanded by the function parameter pack.
Similarly, if \tcode{A} was transformed from a function parameter pack,
it is compared with \changed{each remaining}{The next $ACount-Pcount-i$} parameter types of the parameter template.
If deduction succeeds for a given type,
the type from the argument template is considered to be at least as specialized
as the type from the parameter template.
\begin{example}
    \begin{codeblock}
    template<class... Args>           void f(Args... args);         // \#1
    template<class T1, class... Args> void f(T1 a1, Args... args);  // \#2
    template<class T1, class T2>      void f(T1 a1, T2 a2);         // \#3

    f();                // calls \#1
    f(1, 2, 3);         // calls \#2
    f(1, 2);            // calls \#3; non-variadic template \#3 is more specialized
    // than the variadic templates \#1 and \#2
    \end{codeblock}
\end{example}


%\begin{addedblock}
%    Let N be the number of remaining non-defaulted function template parameters and K be the number of remaining parameter types.
%\end{addedblock}
%
%If \tcode{P} is a function parameter pack,
%the type \tcode{A} of each \changed{remaining}{of the next {K-N}}
%parameter types of the argument template is compared with the type \tcode{P} of
%the \grammarterm{declarator-id} of the function parameter pack. Each comparison
%deduces template arguments for subsequent positions in the template parameter
%packs expanded by the function parameter pack.
%Similarly, if \tcode{A} was transformed from a function parameter pack,
%it is compared with each  \changed{remaining}{of the next {K-N}} parameter type of the parameter template.
%
%
%If deduction succeeds for a given type,
%the type from the argument template is considered to be at least as specialized
%as the type from the parameter template.


\begin{example}
    \begin{codeblock}
        template<class... Args>              void f(Args... args);        // \#1
        template<class T1, class... Args>    void f(T1 a1, Args... args); // \#2
        template<class T1, class T2>         void f(T1 a1, T2 a2);        // \#3

        f();                // calls \#1
        f(1, 2, 3);         // calls \#2
        f(1, 2);            // calls \#3; non-variadic template \#3 is more specialized
        // than the variadic templates \#1 and \#2
    \end{codeblock}
\end{example}

If, for a given type, the
types are identical after the transformations above
and both \tcode{P} and \tcode{A} were reference types (before being replaced with the
type referred to above):
\begin{itemize}
    \item if the type from the argument template was an lvalue reference and the type
    from the parameter template was not,
    the parameter type is not considered to be
    at least as specialized as the argument type; otherwise,
    \item if the type from
    the argument template is more cv-qualified than the type from the
    parameter template (as described above),
    the parameter type is not considered to be
    at least as specialized as the argument type.
\end{itemize}

\pnum
\indextext{at least as specialized as|see{more specialized}}%
Function template \tcode{F}
is \defnx{at least as specialized as}{more specialized}
function template \tcode{G} if,
for each pair of types used to determine the ordering,
the type from \tcode{F}
is at least as specialized as
the type from \tcode{G}.
\tcode{F}
is \defnx{more specialized than}{more specialized!function template}
\tcode{G} if
\tcode{F}
is at least as specialized as
\tcode{G} and
\tcode{G}
is not at least as specialized as
\tcode{F}.

\pnum
If, after considering the above, function template \tcode{F}
is at least as specialized as function template \tcode{G} and vice-versa, and
if \tcode{G} has a \removed{trailing} function parameter pack
for which \tcode{F} does not have a corresponding parameter, and
if \tcode{F} does not have a \removed{trailing} function parameter pack,
then \tcode{F} is more specialized than \tcode{G}.

\pnum
In most cases,
deduction fails if not all template parameters have values,
but for partial ordering purposes a template
parameter may remain without a value provided it is not used in the
types being used for partial ordering.
\begin{note}
    A template parameter used in a non-deduced context is considered used.
\end{note}
\begin{example}
    \begin{codeblock}
        template <class T> T f(int);            // \#1
        template <class T, class U> T f(U);     // \#2
        void g() {
            f<int>(1);                            // calls \#1
        }
    \end{codeblock}
\end{example}

\pnum
\begin{note}
    Partial ordering of function templates containing
    template parameter packs is independent of the number of deduced arguments
    for those template parameter packs.
\end{note}
\begin{example}
    \begin{codeblock}
        template<class ...> struct Tuple { };
        template<class ... Types> void g(Tuple<Types ...>);                 // \#1
        template<class T1, class ... Types> void g(Tuple<T1, Types ...>);   // \#2
        template<class T1, class ... Types> void g(Tuple<T1, Types& ...>);  // \#3

        g(Tuple<>());                   // calls \#1
        g(Tuple<int, float>());         // calls \#2
        g(Tuple<int, float&>());        // calls \#3
        g(Tuple<int>());                // calls \#3
    \end{codeblock}
\end{example}


\rSec3[temp.deduct.type]{Deducing template arguments from a type}

\pnum
The non-deduced contexts are:

\indextext{context!non-deduced}%
\begin{itemize}
\item
The
\grammarterm{nested-name-specifier}
of a type that was specified using a
\grammarterm{qualified-id}.
\item
The \grammarterm{expression} of a \grammarterm{decltype-specifier}.
\item
A non-type template argument or an array bound in which a subexpression
references a template parameter.
\item
A template parameter used in the parameter type of a function parameter that
has a default argument that is being used in the call for which argument
deduction is being done.
\item
A function parameter for which the associated argument is an
overload set\iref{over.over}, and one or more of the following apply:
\begin{itemize}
    \item
    more than one function matches the function parameter type (resulting in
    an ambiguous deduction), or
    \item
    no function matches the function parameter type, or
    \item
    the overload set supplied as an argument contains one or more function templates.
\end{itemize}
\item A function parameter for which the associated argument is an initializer
list\iref{dcl.init.list} but the parameter does not have
a type for which deduction from an initializer list is specified\iref{temp.deduct.call}.
\begin{example}
    \begin{codeblock}
        template<class T> void g(T);
        g({1,2,3});                 // error: no argument deduced for \tcode{T}
    \end{codeblock}
\end{example}
\item A function parameter pack that \changed{does not occur at the end of}{is not the only parameter pack in} the \grammarterm{parameter-declaration-list}.
\end{itemize}

// ...

\pnum

%% \added{Let N be the number of remaining template parameters and K be the numbers of remaining types.}

%If \tcode{P} has a form that contains \tcode{<T>}
%or \tcode{<i>}, then each argument $\mathtt{P}_i$ of the
%respective template argument list of \tcode{P} is compared with the
%corresponding argument $\mathtt{A}_i$ of the corresponding
%template argument list of \tcode{A}. If the template argument list
%of \tcode{P} contains a pack expansion that is not the \changed{last template argument}{only pack expansion}, the entire template argument list is a non-deduced
%context. If $\texttt{P}_i$ is a pack expansion, then the pattern
%of $\texttt{P}_i$ is compared with \changed{each remaining}{the N following} argument\added{s} in the
%template argument list of \tcode{A} \added{where N is such that the deduced parameter list has the same size as the expanded list of arguments}. Each comparison deduces
%template arguments for subsequent positions in the template parameter
%packs expanded by $\texttt{P}_i$.
%During partial ordering\iref{temp.deduct.partial}, if $\texttt{A}_i$ was
%originally a pack expansion:
%\begin{itemize}
%    \item if \tcode{P} does not contain a template argument corresponding to
%    $\texttt{A}_i$ then $\texttt{A}_i$ is ignored;
%
%    \item otherwise, if $\texttt{P}_i$ is not a pack expansion, template argument
%    deduction fails.
%\end{itemize}
%\begin{example}
%    \begin{codeblock}
%        template<class T1, class... Z> class S;                                 // \#1
%        template<class T1, class... Z> class S<T1, const Z&...> { };            // \#2
%        template<class T1, class T2>   class S<T1, const T2&> { };              // \#3
%        S<int, const int&> s;           // both \#2 and \#3 match; \#3 is more specialized
%
%        template<class T, class... U>            struct A { };                  // \#1
%        template<class T1, class T2, class... U> struct A<T1, T2*, U...> { };   // \#2
%        template<class T1, class T2>             struct A<T1, T2> { };          // \#3
%        template struct A<int, int*>;   // selects \#2
%    \end{codeblock}
%\end{example}


\section{Feature test macros}

\ednote{Add a new macro in \tcode{[tab:cpp.predefined.ft]} : \tcode{__cpp_non_trailing_function_pack} set to the date of adoption}.


\section{Acknowledgments}

Thanks to Sy Brand and Michael Wong for their work on P0478 and encouragements.
Tony Van Eerd and Ólafur Waage for proofreading this paper and offering their feedbacks, and Jens Maurer for reviewing the wording.


\bibliographystyle{plain}
\bibliography{wg21}


\renewcommand{\section}[2]{}%
\begin{thebibliography}{9}

    \bibitem[N4885]{N4885}
    Thomas Köppe
    \emph{Working Draft, Standard for Programming Language C++}\newline
    \url{https://wg21.link/N4885}

\end{thebibliography}



\end{document}
