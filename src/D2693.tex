% !TeX program = luatex
% !TEX encoding = UTF-8


\documentclass{wg21}

\title{Formatting \tcode{thread::id} and \tcode{stacktrace}}
\docnumber{P2693R0}
\audience{EWG}
\author{Corentin Jabot}{corentin.jabot@gmail.com}
\authortwo{Victor Zverovich}{victor.zverovich@gmail.com}

\usepackage{color, colortbl}
\begin{document}
\maketitle

\section{Abstract}

This paper provides wording in reply to NB comments suggesting to adopt \paper{P1636R2} (Formatters for library types)
and to add formatters for \tcode{std::stacktrace}.

\section{History}

LEWG approved \paper{P1636R2} in 2019 (Cologne) for C++20.
The paper was subsequently reviewed in 2021 by LWG, who requested wording changes.

SG16 had significant concerns with the formatting of \tcode{filesystem::path} and asked for that formatter to be removed.

Another paper \paper{P2197R0} explored different options for formatting \tcode{std::complex} but was not pursued, nor was it warmly
received by LEWG when presented (summer 2020 telecon).

\paper{P1636R2} has been stuck in need of a revision and attempts to contact the author have failed.


\section{Design}

This paper provides wording for
\begin{itemize}
\item \tcode{std::thread::id}
\item \tcode{std::basic_stacktrace}
\item \tcode{std::stacktrace_entry}
\end{itemize}

Note that \paper{P1636R2} additionally proposed to support

\begin{itemize}
\item \tcode{std::complex}
\item \tcode{std::bitset}
\item \tcode{std::error_code}
\item \tcode{std::unique_ptr}
\item \tcode{std::shared_ptr}
\end{itemize}

Which we decided not to pursue as part of this NB comment resolution.

\subsection{Why do we need \tcode{thread::id} formatting in C++23?}

Two reasons. First, it is very commonly used by loggers. But most importantly that information is not exposed by any other means
than an \tcode{ostream} \tcode{<<} overload. There is no accessor of any kind, so the only well-defined way to extract a \tcode{thread::id}
is to use streams

\begin{colorblock}
std::ostringstream ss;
ss << thread.get_id();
std::print("called a nice API on thread {}", ss.str());
\end{colorblock}

Note that a quick search on GitHub reveals that users, when their expectations are subverted,  will hm... find a way, and won't let such thing as well-definedness stop them. They will, for example, exploit the amazing flexibility of \tcode{printf}
to get what they want:

\begin{colorblock}
printf("[Thread %d Profiling: %ld microseconds] ",
    std::this_thread::get_id(), microseconds); // UB
\end{colorblock}

By properly supporting \tcode{thread::id} in \tcode{format}, we can avoid the proliferation of undefined, non portable, and dangerous code.

\subsection{Why not std::complex?}

Formatting of \tcode{std::complex} is more... complex.
In particular, \paper{P1636R2} (which, to be fair, was approved by LEWG) proposed to use the same notation as \tcode{ostream},
Given the wide variance in how \tcode{complex} are formatted in other programming languages and the interaction with locales
(including the need to support the \tcode{L} specifier), it seems wise to punt this question to a later C++ version.

\begin{tabular}{|c|c|}
    \hline
    ostream & (1.0, 2.0) \\
    \paper{P1636R2} & (1.0, 2.0) \\
    \hline
    \paper{P2197R0} & (1.0, 2.0i) \\
    \hline
    Rust & 1+2i \\
    \hline
    Python & 1+2j \\
    \hline
    Julia &  1.0+ 2.0im \\
    \hline
    Nim & (1.0, 2.0) \\
    \hline
    R &  1+2i \\
    \hline
    Swift & (1.0, 2.0) \\
    \hline
    C\# & 1 + 2i\\
    \hline
\end{tabular}

\subsection{Why not \tcode{std::error_code}/\tcode{bitset}/smart pointers?}

These don't seem sufficiently useful to be processed as part of NB comments.

Moreover, there were plans to remove smart pointer formatters from P1636 for consistency with raw pointers which are intentionally not formattable by default.

\subsection{Stacktrace}

We propose adding formatters for \tcode{std::stacktrace} and \tcode{std::stacktrace_entry} in addition to existing \tcode{std::to_string} overloads such that the following would be equivalent:

\begin{colorblock}
to_string(a_stacktrace);
std::format("{}", a_stacktrace);
\end{colorblock}

Do we really need both?
Beside arithmetic types, the only other type to have a \tcode{std::to_string} method is \tcode{std::bitset}.

\subsubsection{Should \tcode{stacktrace} be formatted as a range?}

This would add a lot of complexity to something that would probably never be used.
The range behavior can be opt-in this way instead:

\begin{colorblock}
std::format("{}", std::views::all(a_stacktrace));
\end{colorblock}

\section{Wording}


\rSec2[stacktrace.syn]{Header \tcode{<stacktrace>} synopsis}\
\begin{codeblock}
#include <compare>              // see \ref{compare.syn}

namespace std {
    // \ref{stacktrace.entry}, class \tcode{stacktrace_entry}
    class stacktrace_entry;

    // \ref{stacktrace.basic}, class template \tcode{basic_stacktrace}
    template<class Allocator>
    class basic_stacktrace;

    // \tcode{basic_stacktrace} \grammarterm{typedef-name}s
    using stacktrace = basic_stacktrace<allocator<stacktrace_entry>>;

    // \ref{stacktrace.basic.nonmem}, non-member functions
    template<class Allocator>
    void swap(basic_stacktrace<Allocator>& a, basic_stacktrace<Allocator>& b)
    noexcept(noexcept(a.swap(b)));

    string to_string(const stacktrace_entry& f);

    template<class Allocator>
    string to_string(const basic_stacktrace<Allocator>& st);

    template<class charT, class traits>
    basic_ostream<charT, traits>&
    operator<<(basic_ostream<charT, traits>& os, const stacktrace_entry& f);

    template<class charT, class traits, class Allocator>
    basic_ostream<charT, traits>&
    operator<<(basic_ostream<charT, traits>& os, const basic_stacktrace<Allocator>& st);

\end{codeblock}
\begin{addedblock}
\begin{codeblock}
    struct formatter<stacktrace_entry>

    template<class Allocator>
    struct formatter<basic_stacktrace<Allocator>>;
\end{codeblock}
\end{addedblock}
\begin{codeblock}
    namespace pmr {
        using stacktrace = basic_stacktrace<polymorphic_allocator<stacktrace_entry>>;
    }

    // \ref{stacktrace.basic.hash}, hash support
    template<class T> struct hash;
    template<> struct hash<stacktrace_entry>;
    template<class Allocator> struct hash<basic_stacktrace<Allocator>>;
}
\end{codeblock}

\indexlibrarymember{operator<<}{basic_stacktrace}%
\begin{itemdecl}
    template<class charT, class traits, class Allocator>
    basic_ostream<charT, traits>&
    operator<<(basic_ostream<charT, traits>& os, const basic_stacktrace<Allocator>& st);
\end{itemdecl}

\begin{itemdescr}
    \pnum
    \effects
    Equivalent to: \tcode{return os << to_string(st);}
\end{itemdescr}

\textcolor{noteclr}{[...]}

\begin{addedblock}

\rSec3[stacktrace.format]{Formatting support}

\begin{itemdecl}
struct formatter<stacktrace_entry>;
\end{itemdecl}

\begin{itemdescr}
The \tcode{parse} member function of this formatter takes an empty format specification.

A \tcode{stacktrace_entry} object is formatted as if by passing it to \tcode{to_string} and copying the returned string through the output iterator of the context.
\end{itemdescr}



\begin{itemdecl}
template<class Allocator>
struct formatter<basic_stacktrace<Allocator>>;
\end{itemdecl}

\begin{itemdescr}
The \tcode{parse} member function of this formatter takes an empty format specification.

A \tcode{basic_stacktrace} object is formatted as if by passing it to \tcode{to_string} and copying the returned string through the output iterator of the context.
\end{itemdescr}

\end{addedblock}


\rSec3[stacktrace.basic.hash]{Hash support}

\begin{itemdecl}
    template<> struct hash<stacktrace_entry>;
    template<class Allocator> struct hash<basic_stacktrace<Allocator>>;
\end{itemdecl}

\begin{itemdescr}
    \pnum
    The specializations are enabled\iref{unord.hash}.
\end{itemdescr}

\rSec1[thread]{Concurrency support library}

\rSec3[thread.thread.id]{Class \tcode{thread::id}}

\indexlibraryglobal{thread::id}%
\indexlibrarymember{thread}{id}%
\begin{codeblock}
    namespace std {
        class thread::id {
            public:
            id() noexcept;
        };

        bool operator==(thread::id x, thread::id y) noexcept;
        strong_ordering operator<=>(thread::id x, thread::id y) noexcept;

        template<class charT, class traits>
        basic_ostream<charT, traits>&
        operator<<(basic_ostream<charT, traits>& out, thread::id id);

        // hash support
        template<class T> struct hash;
        template<> struct hash<thread::id>;

        @\added{template<class charT>}@
        @\added{struct formatter<thread::id, charT>;}@
    }
\end{codeblock}

\pnum
An object of type \tcode{thread::id} provides a unique identifier for
each thread of execution and a single distinct value for all \tcode{thread}
objects that do not represent a thread of
execution\iref{thread.thread.class}. Each thread of execution has an
associated \tcode{thread::id} object that is not equal to the
\tcode{thread::id} object of any other thread of execution and that is not
equal to the \tcode{thread::id} object of any \tcode{thread} object that
does not represent threads of execution.

\begin{addedblock}
The \defn{text representation} of an object of type \tcode{thread::id}
is an unspecified value such that, for two objects of type \tcode{thread::id} \tcode{x} and \tcode{y},
if \tcode{x == y} the \tcode{thread::id} objects have the same text
representation and if \tcode{x != y} the \tcode{thread::id} objects have
distinct text representations.
\end{addedblock}

\pnum
\tcode{thread::id} is a trivially copyable class\iref{class.prop}.
The library may reuse the value of a \tcode{thread::id} of a terminated thread that can no longer be joined.


\begin{itemdescr}
    \pnum
    Let $P(\tcode{x}, \tcode{y})$ be
    an unspecified total ordering over \tcode{thread::id}
    as described in \ref{alg.sorting}.

    \pnum
    \returns
    \tcode{strong_ordering::less} if $P(\tcode{x}, \tcode{y})$ is \tcode{true}.
    Otherwise, \tcode{strong_ordering::greater}
    if $P(\tcode{y}, \tcode{x})$ is \tcode{true}.
    Otherwise, \tcode{strong_ordering::equal}.
\end{itemdescr}

\indexlibrarymember{operator<<}{thread::id}%
\begin{itemdecl}
    template<class charT, class traits>
    basic_ostream<charT, traits>&
    operator<< (basic_ostream<charT, traits>& out, thread::id id);
\end{itemdecl}

\begin{itemdescr}
    \pnum
    \effects
    Inserts \changed{an unspecified}{the} text representation of \tcode{id} into
    \tcode{out}. \removed{For two objects of type \tcode{thread::id} \tcode{x} and \tcode{y},
    if \tcode{x == y} the \tcode{thread::id} objects have the same text
    representation and if \tcode{x != y} the \tcode{thread::id} objects have
    distinct text representations.}

    \pnum
    \returns
    \tcode{out}.
\end{itemdescr}

\begin{addedblock}

\begin{itemdecl}
template<class charT>
class formatter<thread::id, charT>;
\end{itemdecl}

\begin{itemdescr}
\tcode{formatter<thread::id, charT>} meets the \defn{Formatter} requirements ([formatter.requirements]). The \tcode{parse} member functions of this formatter interpret the format specification as a \grammarterm{thread-id-format-spec} according to the following syntax:

\grammarterm{thread-id-format-spec}:
  \grammarterm{fill-and-align_{opt}} \grammarterm{width_{opt}}

The productions \grammarterm{fill-and-align} and \grammarterm{width} are described in [format.string]. If the \grammarterm{align} option is omitted it is defaulted to \tcode{>}.

A \tcode{thread::id} object is formatted by writing the \defn{text representation} to the output with \grammarterm{thread-id-format-spec} applied.
\end{itemdescr}

\end{addedblock}

\subsection{Feature test macro}

\ednote{define \tcode{__cpp_lib_formatters} set to the date of adoption in \tcode{<version>}, \tcode{<stacktrace>} and \tcode{<thread>}}.

\section{Acknowledgments}

Thanks to Lars Gullik Bjønnes for their initial work on P1636!

\section{References}

\renewcommand{\section}[2]{}%
\bibliographystyle{plain}
\bibliography{wg21, extra}

\begin{thebibliography}{9}


\bibitem[N4885]{N4885}
Thomas Köppe
\emph{Working Draft, Standard for Programming Language C++}\newline
\url{https://wg21.link/N4885}


\end{thebibliography}

\end{document}
