% !TeX program = luatex
% !TEX encoding = UTF-8


\documentclass{wg21}

\title{A Simple Approach to Universal Template Parameters}
\docnumber{P2989R0}
\audience{EWG}
\author{Corentin Jabot}{corentin.jabot@gmail.com}
\authortwo{Gašper Ažman}{gasper.azman@gmail.com}
\usepackage[dvipsnames]{xcolor}
\usepackage{graphicx}



\def\changemargin#1#2{\list{}{\rightmargin#2\leftmargin#1}\item[]}
\let\endchangemargin=\endlist

\begin{document}
\maketitle

\section{Abstract}
In \paper{P1985R3}, we explore universal template parameters (a template parameter that can be a type, non-type, or template name).
We proposed allowing the use of universal template parameters in all contexts, which would have added a large amount of complexity to both the specification and implementations
for limited benefits. In this paper, we propose a much simpler approach to the same feature.

\section{Difference with \paper{P1985R3}}

This paper is really a follow up on \paper{P1985R3}.
The main difference is that we are only proposing that universal template parameters can only be used as template arguments, to side step parsing ambiguities, syntax disambiguators etc
(more detail below).

Beside that, this is still very much the same feature and the same design (there are not that many ways to design this feature).
We present a few design questions not considered before and report on early implementation efforts in Clang.


\section{Design}

The design is centered around as few pieces as possible.
\begin{itemize}
\item A universal template parameter can be declared in the template head of a function, class, or variable template.
\item The name of a universal parameter (which is found by unqualified lookup) can be used as a template argument. And nowhere else.
\item universal template parameters can be packs
\item universal template parameters cannot be defaulted.
\item Forwarding a universal template to another template, as well as partial specializations are the main way to handle universal template parameters
\end{itemize}

\subsection{Partial ordering of universal template parameters}

Anything that can be a template argument can be the argument of a universal template parameter. Hence the name Universal Template Parameter.

Consequently, Universal Template Parameters are less specialized than any other sort of template they are compared with.
This allows specializing an entity based on template parameter kind.

\begin{colorblock}
template <universal template>
constexpr bool is_variable = false;   // #1
template <auto a>
constexpr bool is_variable<a> = true; // #2
\end{colorblock}

Here, \tcode{\#2} is more specialized than \tcode{\#1} because a universal template parameter is less specialized than a non-type parameter.

\subsection{Universal Template Parameter Deduction}

The \emph{kind} of a universal parameter is determined by the \emph{kind} of the corresponding template argument.
IE, if the template argument is a type, the corresponding template parameter will be a type template parameter.

If the template argument denotes a template name (or concept name), the corresponding universal template parameter
is deduced to be a template (or concept) template parameter with the same template head.

\begin{colorblock}
template <universal template T>
struct S;

using A = S<std::pair>; // T is compared against template <typename T, typename U> typename
\end{colorblock}

If the template argument is an expression, the corresponding universal template parameter becomes a non-type template parameter.

\subsection{NTTP Deduction}

This brings an interesting question: When replacing a universal template parameter with a non-type argument, what type should we deduce against?
We have a few options:
\begin{itemize}
\item Deducing the exact type -  Similar to \tcode{decltype(auto)}. While seemingly simple at first, this is not particularly useful as xvalues would be deduced as \tcode{T\&\&}, which are not structural types and therefore
are not valid template arguments.
\item Deducing a non-reference type - similar to \tcode{auto}. This would make it impossible to"forward" a reference to another template. As passing template arguments to other entities is the main motivation for this feature, this is also not a viable option.
\item Deducing either a (suitably cv qualified) lvalue reference type if the argument is an lvalue, and a non-reference type otherwise. This allows passing references to other templates while preventing the creation of rvalue references. This is therefore the option we propose.
% TODO example
\end{itemize}

\subsection{Concepts}

Because concepts cannot be specialized, and because we do not have a good motivation for it, we do not propose to support universal template parameter
In the template heads of concepts. universal template parameters also cannot be constrained with anything resembling a \grammarterm{type-constraint}.
However, universal template parameter names can appear in template arguments in \tcode{requires} clauses.
It is therefore possible to constrain an entity to accept only specific kinds of universal template parameters.

% todo example

\subsection{Function template}


We are not proposing to allow passing an overload set or a function template as a universal template argument.
This might be something to consider in the future but there certainly would be implementation challenges.
Overall this is an orthogonal feature that has less to do with universal template argument than it has to the countless
"overload set as first class objects" (\paper{P1170R0}) and "customization point objects" (\paper{P2547R1}) papers.


\subsection{Library Support: Universal template parameters with \tcode{is}}

We propose a set of library traits to accompany the core language feature.


\begin{colorblock}
template <universal template T>
inline constexpr bool is_typename_v = false;
template <universal template U>
inline constexpr bool is_nttp_v = false;
template <universal template U>
inline constexpr bool is_template_v = false;
template <universal template U>
inline constexpr bool is_type_template_v = false;
template <universal template U>
inline constexpr bool is_var_template_v = false;
template <universal template U>
inline constexpr bool is_concept_v = false;
\end{colorblock}

These type traits can be specialized as follow:

\begin{colorblock}
template <typename U>
inline constexpr bool is_typename_v<U> = true;

template <auto U>
inline constexpr bool is_nttp_v<U> = true;

template <template<universal template....> universal template U>
constexpr bool is_template_v<U> = true;

template <template<universal template....> typename U>
inline constexpr bool is_type_template_v<U> = true;

template <template<universal template....> auto U>
inline constexpr bool is_var_template_v<U> = true;

template <template<universal template....> concept U>
inline constexpr bool is_concept_v<U> = true;
\end{colorblock}
The final wording will most likely include support for the non-\tcode{_v} version of these traits.

We would be remiss not to propose \tcode{std::is_specialization_of} (\paper{P2098R1}) here:

\begin{colorblock}
template<universal template T, universal template Primary >
requires is_var_template_v<Primary> || requires is_type_template_v<Primary>
inline constexpr bool is_specialization_of_v = false;

template<
    template<universal template...> typename Primary,
    universal template... Args
>
inline constexpr bool is_specialization_of_v<Primary<Args...>, Primary>  = true;

template<
    template<universal template...> auto Primary,
    universal template... Args
>
inline constexpr bool is_specialization_of_v<Primary<Args...>, Primary>  = true;
\end{colorblock}

Unlike \paper{P2098R1} which was rejected for not being universal enough (the technology did not exist at the time, this implementation) not only supports checking
specializations for any class templates, including those having template parameters that are not types, but also specialization of variable templates.

\subsection{In which we mention reflection}

One of the things we do not propose in this paper is to force the interpretation of a universal template as a specific kind "in place" - in contrast with \paper{P1985R3}.
IE, to use a universal template parameter, you have to pass it to another template which is then specialized for templates, NTTP, template names, etc.
Naming a universal template parameter anywhere except as a template argument is ill-formed.

There are a few reasons for that.
First for types and NTTP it is easy enough to write \tcode{as_type} and \tcode{as_value} respectively as library functions (not proposed).

\begin{colorblock}
template<universal template>
struct as_type;
template<typename T>
struct as_type<T> { using type = T; };
template <universal template T>
using as_type_t = as_type<T>::type;

template<universal template U>
constexpr auto as_value_v = delete;
template<decltype(auto) V>
constexpr decltype(auto) as_value_v<V> = V;
\end{colorblock}

These would allow extracting a type/value from a UTTP

\begin{colorblock}
template <universal template U>
constexpr auto test() {
    if constexpr(is_typename_v<U>) {
        using a =  as_type<int>::type; // ok
        return a{42};
    }
    else if constexpr(is_nttp_v<U>) {
        decltype(auto) v = as_value_v<U>;
        return v;
    }
}
static_assert(test<int>() == 42);
static_assert(test<24>() == 24);
\end{colorblock}

The second reason is that use cases are limited.
The reason we need universal template parameters is to be able to handle entities of different shapes in generic contexts.

However, the main reason is that the concern is somewhat orthogonal and not specific to this proposal.
Enter reflection.
This is something we mentioned in \paper{P1985R3}, and not just because we love to talk about reflection in every paper.

Reflection has this feature which is now called splicing although terminology changed over the years when you can convert a reflection of an entity
back into that entity. And of course, because it is possible to reflect on anything (in a theoretical future), then splicing a reflection
can produce any kind of entity (I apologize to Core for my liberal use of "entity" through this paper).

Quoting from \paper{P1240R2}

\begin{quoteblock}
\begin{colorblock}
struct S { struct I { }; };
template<int N> struct X;
auto refl = ˆS;
auto tmpl = ˆX;
void f() {
    typename[:refl:] * x; // Okay: declares x to be a pointer to S.
    [:refl:] * x; // Error: attempt to multiply int by x.
    [:refl:]::I i; // Okay: splice as part of a nested-name-specifier.
    typename[:refl:]{}; // Okay: default-constructs an S temporary.
    using T = [:refl:]; // Okay: operand must be a type.
    struct C: [:refl:] {}; // Okay: base classes are types.
    template[:tmpl:]<0>; // Okay: names the specialization.
    [:tmpl:] < 0 > x; // Error: attempt to compare X with 0.
}
\end{colorblock}
\end{quoteblock}

This set of examples is rather illustrative of what we need to solve in general.
Both splices and UTP are dependent expressions and need some form of parsing disambiguator to be usable
in arbitrary contexts, like other dependent names (i.e. member of classes templates).

So both \paper{P1985R3} and reflection had similar needs for disambiguating new interesting names and
both papers try to come up with rules for cleverly avoiding the need for a disambiguator syntax.
As shown in \paper{P1985R3}, this has further ramifications if we allow some form of aliases of universal template parameters, and/or some form of general aliasing that would allow more entities to become dependent
and increase the need for disambiguation syntaxes.
And of course, work is done on member packs and pack aliases, which adds a layer of consideration to these disambiguation syntaxes.

And all of that has of course a non-trivial impact on C++ parsers, so progressing one step at a time seems reasonable.

This is why we focus solely on allowing universal template parameters and once we have the basis right, we can expand to
allow UTP in more places, if we find a compelling use case for it.

Both reflection and "UTP everywhere" should have a consistent syntax for disambiguators and where we can and can't omit them.

\subsection{Syntax}

Speaking of syntax... we need a syntax for Universal Template Parameters.
The set of possible syntax is infinite and it is unlikely productive to do an exhaustive search,

\begin{itemize}

\item \tcode{universal template Foo} works (ie is not ambiguous)

\item \tcode{template auto} which is the syntax used by circle and \paper{P1985R0} reused \tcode{auto} in inconsistent ways,
which many people - including us - are not the biggest fan of.

\item \tcode{template} as an isolated keyword is perfectly fine, but people have expressed opposition to that.

\item \tcode{__any} or \tcode{__universal} would also work, but C++ doesn't do that (unlike C).

\item \tcode{universal_template} or similar would also work but again we don't do underscore often (of course, \tcode{co_yield} and co do)

\item \tcode{anytmplarg} or any such weird enough keyword would also work. (we can't pick something non-weird, it would be ambiguous)

\item \tcode{register} would like you to know it is available and down for anything.

\end{itemize}

We are proposing \tcode{universal template}. People seem to not hate it. It does what it says on the tin.
We are of course \textbf{not} proposing to make \tcode{universal} a keyword.
In a template parameter declaration, \tcode{universal template} would have a special meaning, so the meaning of \tcode{universal} would be
contextual, like it is the case for \tcode{module}, \tcode{final}, \tcode{override}, etc.


\begin{colorblock}
template <
        universal template, // template
        universal foo       // type constraint or NTTP
>
\end{colorblock}

Putting some of these options together show they are quite similar.

\begin{table}
\tiny
\hspace{-30pt}
\begin{tabular}{c | c | c}
\begin{colorblock}
template <
  universal template,
  template <universal template> auto C
>
struct s;
\end{colorblock}
&
\begin{colorblock}
template <
  anytmplarg,
  template <anytmplarg> auto C
>
struct s;
\end{colorblock}
&
\begin{colorblock}
template <
  template auto,
  template <template auto> auto C
>
struct s;
\end{colorblock}
\end{tabular}
\end{table}


The feature we are proposing allows us to express ideas that can't be expressed without this proposal, so it is important,
but we expect these things will mostly be used by select generic facilities. Terseness is not a goal here.

The \tcode{template auto} syntax is perhaps less cromulent than the two other option ilustrated above as, in addition to overloading
auto with a novel meaning (one that has ery little to do with variables), it makes it harder (for a human) to distinguish
variable templates, template variable template parameters, template template parameters with a variable template, universal templates,
template template parameters with a universal template, etc.

\begin{colorblock}
template <
   auto, // variable
   template auto, // universal parameter
   template <auto> auto, // variable template
   template <template auto> auto // variable template with a universal template parameter.
>
struct S;
\end{colorblock}


We should clarify that none of the option proposed here pose anty challenge for the compiler. It's only what will feel more intuitive to
developers with time. And as long as we don't do something too outlandish, any sytax we pick will become familiar with time.


\section{Status of this proposal and implementation}

The design presented in this paper roughly matches the implementation of Universal template parameters, albeit with a different syntax.
We started a prototype implementation in Clang to re-demonstrate implementability and find out interesting design questions we might otherwise
miss.

The clang implementation does not support packs or template template parameters, but it does support passing type and non-type template arguments
as arguments to universal parameters, and partial ordering/specialization is also generally working.

Introducing a new kind of template parameters/template arguments certainly requires the equivalent of a few weeks of work for a production-ready implementation,
but as far as Clang is concerned there are no fundamental limitations that would make this proposal impossible or unreasonable to implement.


Our aim and hope is to agree on general design and syntax in Kona and come back in the next few months with a more complete implementation and wording.
We should also progress \paper{P2841R0} first as we intend for concept and variable template to be valid universal template parameters,
and so there is a better order of operation here, especially for ease of specification.

\section{Wording}

TBD!

\section{Acknowledgments}

\bibliographystyle{plain}
\bibliography{wg21}


\renewcommand{\section}[2]{}%

\begin{thebibliography}{9}


\end{thebibliography}
\end{document}
