\documentclass{wg21}

\usepackage{xcolor}
\usepackage{soul}
\usepackage{ulem}
\usepackage{fullpage}
\usepackage{parskip}
\usepackage{csquotes}
\usepackage{listings}
\usepackage{minted}
\usepackage{enumitem}
\usepackage{minted}


\lstdefinestyle{base}{
  language=c++,
  breaklines=false,
  basicstyle=\ttfamily\color{black},
  moredelim=**[is][\color{green!50!black}]{@}{@},
  escapeinside={(*@}{@*)}
}

\newcommand{\cc}[1]{\mintinline{c++}{#1}}
\newminted[cpp]{c++}{}


\title{A SFINAE-friendly trait to determine the extent of statically sized containers}
\docnumber{P1419R0}
\audience{LEWG}
\author{Corentin Jabot}{corentin.jabot@gmail.com}
\authortwo{Casey Carter}{Casey@carter.net}

\begin{document}
\maketitle

\section{Abstract}

We propose \tcode{ranges::static_extent}, a SFINAE friendly replacement of \tcode{std::extent} compatible with all statically sized containers.

\section{Tony tables}

Here is a (simplified) wording for span without and with this proposal

\begin{center}
\begin{tabular}{l|l}
Before & After\\ \hline

\begin{minipage}[t]{0.5\textwidth}

\begin{minted}[fontsize=\footnotesize]{cpp}

template<class ElementType, 
	ptrdiff_t Extent = dynamic_extent>
class span {

template<size_t N>
constexpr span(element_type (&arr)[N]);
template<size_t N>
constexpr span(array<value_type, N>& arr);
template<size_t N>
constexpr span(const array<value_type, N>& arr);
template<ContiguousRange R>
constexpr span(R&& cont);

//...
}
\end{minted}
\end{minipage}
&
\begin{minipage}[t]{0.5\textwidth}
\begin{minted}[fontsize=\footnotesize]{cpp}

template<class ElementType, 
ptrdiff_t Extent = dynamic_extent>
class span {

template <ranges::ContiguousRange R>
requires Extent == dynamic_extent 
   || ranges::static_extent_v<R> == dynamic_extent
constexpr span(R&& r);

//...
};
\end{minted}
\end{minipage}
\\\\ \hline

\end{tabular}
\end{center}


\section{Motivation}

This paper is an offshoot of \cite{P1394}. While writing the wording and the implementation of span constructors, it become clear that
a trait to determine the extent of a type would simplify both the wording and the implementation of \tcode{std::span} and any code 
dealing with types with static extent.

\tcode{std::extent} suffers from a few shortcomings that make it ill suited for the task:

\begin{itemize}
	\item It only supports raw arrays
	\item \tcode{extent<T>::value} is well-formed for non-array types which means it can't be used in SFINAE contexts
	\item Because it returns 0 for types with no static extent, types with a static extent of 0 and types with no static extent would not be valid.
\end{itemize}

\section{Proposal}

We propose a new type trait \tcode{std::ranges::static_extent} to supersede \tcode{std::extent} such that:

\begin{itemize}
	\item \tcode{ranges::static_extent<T>::value} is well formed if and only if the type has a static extent.
	\item \tcode{ranges::static_extent} can be specialized for non array types such as  \tcode{std::array}, \tcode{std::span}, \tcode{std::mdspan}
	and user defined types;
\end{itemize}


\section{Proposed wording}

\begin{quote}
\begin{codeblock}

namespace ranges {
	template<class T, unsigned I = 0>
	struct static_extent;
	template <class T, unsigned I>
	struct static_extent<T[], I> : std::extent<T[], I> {};
	template<class T, std::size_t N, unsigned I>
	struct static_extent<T[N], I> : std::extent<T[N], I> {}
	template <class T, std::size_t N>
	struct static_extent<std::array<T, N>> : std::integral_constant<size_t, N> {};
	template <class T, std::size_t N>
	struct static_extent<std::span<T, N>> : std::integral_constant<size_t, N> {};
	
	template<class T, unsigned I = 0>
	inline constexpr size_t static_extent_v = static_extent<T, I>::value;
};

\end{codeblock}
\end{quote}


\begin{itemdecl}
template<class T, unsigned I = 0>
struct static_extent;
\end{itemdecl}
\begin{itemdescr}
	\pnum
	If T is an array, the member \tcode{value} shall be equal to \tcode{std::extent_v<T[], I>}.
	Otherwise, unless this trait is specialized there shall be no member \tcode{value}.
	
	Pursuant to [namespace.std], a program may specialize \tcode{static_extent} for statically sized types satisfying the requirements of Ranges
	such that, given an instance c of type T:
		\begin{itemize}
			\item If \tcode{I} equals 0 then \tcode{range::size(c)} shall always be equal to \tcode{static_extent<T>::value}
			\item Otherwise, range::size(c[I]) shall always be equal to \tcode{static_extent<T, I>::value}
		\end{itemize}
	\pagebreak
	\begin{example}
	\begin{codeblock}
	// the following assertions hold:
	static_assert (static_extent_v<int[2]> == 2);
	static_assert (static_extent_v<int[2][4], 1> == 4);
	static_assert (static_extent_v<int[][4], 1> == 4);
	static_assert (static_extent_v<std::span<int, 5>> == 5);
	static_assert (static_extent_v<std::array<int, 1>> == 1);
	// the following expression are ill formed
	(static_extent_v<int>);
	(static_extent_v<std::vector<int>>);
	(static_extent_v<std::span<int>>);
	(static_extent_v<std::array<int>, 1>);		
\end{codeblock}
	\end{example}
\end{itemdescr}

\subsection{\tcode{std::dynamic_extent}}

For consistency, 
we propose to move \tcode{std::dynamic_extent} from the header \tcode{<span>} \linebreak
to \tcode{std::ranges::dynamic_extent} in the header \tcode{<ranges>}

\section{Future work}

\tcode{std::span} should be modified to benefits of the changes proposed here.

\section{References}
\renewcommand{\section}[2]{}%
\begin{thebibliography}{9}
	\bibitem[P1394]{P1394}
	Corentin Jabot
	\emph{Range constructor for std::span}\newline
	\url{https://wg21.link/P1394}
\end{thebibliography}
\end{document}

\end{document}